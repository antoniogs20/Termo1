\documentclass[11pt]{article}

    \usepackage[breakable]{tcolorbox}
    \usepackage{parskip} % Stop auto-indenting (to mimic markdown behaviour)
    
    \usepackage{iftex}
    \ifPDFTeX
    	\usepackage[T1]{fontenc}
    	\usepackage{mathpazo}
    \else
    	\usepackage{fontspec}
    \fi

    % Basic figure setup, for now with no caption control since it's done
    % automatically by Pandoc (which extracts ![](path) syntax from Markdown).
    \usepackage{graphicx}
    % Maintain compatibility with old templates. Remove in nbconvert 6.0
    \let\Oldincludegraphics\includegraphics
    % Ensure that by default, figures have no caption (until we provide a
    % proper Figure object with a Caption API and a way to capture that
    % in the conversion process - todo).
    \usepackage{caption}
    \DeclareCaptionFormat{nocaption}{}
    \captionsetup{format=nocaption,aboveskip=0pt,belowskip=0pt}

    \usepackage[Export]{adjustbox} % Used to constrain images to a maximum size
    \adjustboxset{max size={0.9\linewidth}{0.9\paperheight}}
    \usepackage{float}
    \floatplacement{figure}{H} % forces figures to be placed at the correct location
    \usepackage{xcolor} % Allow colors to be defined
    \usepackage{enumerate} % Needed for markdown enumerations to work
    \usepackage{geometry} % Used to adjust the document margins
    \usepackage{amsmath} % Equations
    \usepackage{amssymb} % Equations
    \usepackage{textcomp} % defines textquotesingle
    % Hack from http://tex.stackexchange.com/a/47451/13684:
    \AtBeginDocument{%
        \def\PYZsq{\textquotesingle}% Upright quotes in Pygmentized code
    }
    \usepackage{upquote} % Upright quotes for verbatim code
    \usepackage{eurosym} % defines \euro
    \usepackage[mathletters]{ucs} % Extended unicode (utf-8) support
    \usepackage{fancyvrb} % verbatim replacement that allows latex
    \usepackage{grffile} % extends the file name processing of package graphics 
                         % to support a larger range
    \makeatletter % fix for grffile with XeLaTeX
    \def\Gread@@xetex#1{%
      \IfFileExists{"\Gin@base".bb}%
      {\Gread@eps{\Gin@base.bb}}%
      {\Gread@@xetex@aux#1}%
    }
    \makeatother

    % The hyperref package gives us a pdf with properly built
    % internal navigation ('pdf bookmarks' for the table of contents,
    % internal cross-reference links, web links for URLs, etc.)
    \usepackage{hyperref}
    % The default LaTeX title has an obnoxious amount of whitespace. By default,
    % titling removes some of it. It also provides customization options.
    \usepackage{titling}
    \usepackage{longtable} % longtable support required by pandoc >1.10
    \usepackage{booktabs}  % table support for pandoc > 1.12.2
    \usepackage[inline]{enumitem} % IRkernel/repr support (it uses the enumerate* environment)
    \usepackage[normalem]{ulem} % ulem is needed to support strikethroughs (\sout)
                                % normalem makes italics be italics, not underlines
    \usepackage{mathrsfs}
    

    
    % Colors for the hyperref package
    \definecolor{urlcolor}{rgb}{0,.145,.698}
    \definecolor{linkcolor}{rgb}{.71,0.21,0.01}
    \definecolor{citecolor}{rgb}{.12,.54,.11}

    % ANSI colors
    \definecolor{ansi-black}{HTML}{3E424D}
    \definecolor{ansi-black-intense}{HTML}{282C36}
    \definecolor{ansi-red}{HTML}{E75C58}
    \definecolor{ansi-red-intense}{HTML}{B22B31}
    \definecolor{ansi-green}{HTML}{00A250}
    \definecolor{ansi-green-intense}{HTML}{007427}
    \definecolor{ansi-yellow}{HTML}{DDB62B}
    \definecolor{ansi-yellow-intense}{HTML}{B27D12}
    \definecolor{ansi-blue}{HTML}{208FFB}
    \definecolor{ansi-blue-intense}{HTML}{0065CA}
    \definecolor{ansi-magenta}{HTML}{D160C4}
    \definecolor{ansi-magenta-intense}{HTML}{A03196}
    \definecolor{ansi-cyan}{HTML}{60C6C8}
    \definecolor{ansi-cyan-intense}{HTML}{258F8F}
    \definecolor{ansi-white}{HTML}{C5C1B4}
    \definecolor{ansi-white-intense}{HTML}{A1A6B2}
    \definecolor{ansi-default-inverse-fg}{HTML}{FFFFFF}
    \definecolor{ansi-default-inverse-bg}{HTML}{000000}

    % commands and environments needed by pandoc snippets
    % extracted from the output of `pandoc -s`
    \providecommand{\tightlist}{%
      \setlength{\itemsep}{0pt}\setlength{\parskip}{0pt}}
    \DefineVerbatimEnvironment{Highlighting}{Verbatim}{commandchars=\\\{\}}
    % Add ',fontsize=\small' for more characters per line
    \newenvironment{Shaded}{}{}
    \newcommand{\KeywordTok}[1]{\textcolor[rgb]{0.00,0.44,0.13}{\textbf{{#1}}}}
    \newcommand{\DataTypeTok}[1]{\textcolor[rgb]{0.56,0.13,0.00}{{#1}}}
    \newcommand{\DecValTok}[1]{\textcolor[rgb]{0.25,0.63,0.44}{{#1}}}
    \newcommand{\BaseNTok}[1]{\textcolor[rgb]{0.25,0.63,0.44}{{#1}}}
    \newcommand{\FloatTok}[1]{\textcolor[rgb]{0.25,0.63,0.44}{{#1}}}
    \newcommand{\CharTok}[1]{\textcolor[rgb]{0.25,0.44,0.63}{{#1}}}
    \newcommand{\StringTok}[1]{\textcolor[rgb]{0.25,0.44,0.63}{{#1}}}
    \newcommand{\CommentTok}[1]{\textcolor[rgb]{0.38,0.63,0.69}{\textit{{#1}}}}
    \newcommand{\OtherTok}[1]{\textcolor[rgb]{0.00,0.44,0.13}{{#1}}}
    \newcommand{\AlertTok}[1]{\textcolor[rgb]{1.00,0.00,0.00}{\textbf{{#1}}}}
    \newcommand{\FunctionTok}[1]{\textcolor[rgb]{0.02,0.16,0.49}{{#1}}}
    \newcommand{\RegionMarkerTok}[1]{{#1}}
    \newcommand{\ErrorTok}[1]{\textcolor[rgb]{1.00,0.00,0.00}{\textbf{{#1}}}}
    \newcommand{\NormalTok}[1]{{#1}}
    
    % Additional commands for more recent versions of Pandoc
    \newcommand{\ConstantTok}[1]{\textcolor[rgb]{0.53,0.00,0.00}{{#1}}}
    \newcommand{\SpecialCharTok}[1]{\textcolor[rgb]{0.25,0.44,0.63}{{#1}}}
    \newcommand{\VerbatimStringTok}[1]{\textcolor[rgb]{0.25,0.44,0.63}{{#1}}}
    \newcommand{\SpecialStringTok}[1]{\textcolor[rgb]{0.73,0.40,0.53}{{#1}}}
    \newcommand{\ImportTok}[1]{{#1}}
    \newcommand{\DocumentationTok}[1]{\textcolor[rgb]{0.73,0.13,0.13}{\textit{{#1}}}}
    \newcommand{\AnnotationTok}[1]{\textcolor[rgb]{0.38,0.63,0.69}{\textbf{\textit{{#1}}}}}
    \newcommand{\CommentVarTok}[1]{\textcolor[rgb]{0.38,0.63,0.69}{\textbf{\textit{{#1}}}}}
    \newcommand{\VariableTok}[1]{\textcolor[rgb]{0.10,0.09,0.49}{{#1}}}
    \newcommand{\ControlFlowTok}[1]{\textcolor[rgb]{0.00,0.44,0.13}{\textbf{{#1}}}}
    \newcommand{\OperatorTok}[1]{\textcolor[rgb]{0.40,0.40,0.40}{{#1}}}
    \newcommand{\BuiltInTok}[1]{{#1}}
    \newcommand{\ExtensionTok}[1]{{#1}}
    \newcommand{\PreprocessorTok}[1]{\textcolor[rgb]{0.74,0.48,0.00}{{#1}}}
    \newcommand{\AttributeTok}[1]{\textcolor[rgb]{0.49,0.56,0.16}{{#1}}}
    \newcommand{\InformationTok}[1]{\textcolor[rgb]{0.38,0.63,0.69}{\textbf{\textit{{#1}}}}}
    \newcommand{\WarningTok}[1]{\textcolor[rgb]{0.38,0.63,0.69}{\textbf{\textit{{#1}}}}}
    
    
    % Define a nice break command that doesn't care if a line doesn't already
    % exist.
    \def\br{\hspace*{\fill} \\* }
    % Math Jax compatibility definitions
    \def\gt{>}
    \def\lt{<}
    \let\Oldtex\TeX
    \let\Oldlatex\LaTeX
    \renewcommand{\TeX}{\textrm{\Oldtex}}
    \renewcommand{\LaTeX}{\textrm{\Oldlatex}}
    % Document parameters
    % Document title
    \title{Esperienza 1}
    
    
    
    
    
% Pygments definitions
\makeatletter
\def\PY@reset{\let\PY@it=\relax \let\PY@bf=\relax%
    \let\PY@ul=\relax \let\PY@tc=\relax%
    \let\PY@bc=\relax \let\PY@ff=\relax}
\def\PY@tok#1{\csname PY@tok@#1\endcsname}
\def\PY@toks#1+{\ifx\relax#1\empty\else%
    \PY@tok{#1}\expandafter\PY@toks\fi}
\def\PY@do#1{\PY@bc{\PY@tc{\PY@ul{%
    \PY@it{\PY@bf{\PY@ff{#1}}}}}}}
\def\PY#1#2{\PY@reset\PY@toks#1+\relax+\PY@do{#2}}

\expandafter\def\csname PY@tok@w\endcsname{\def\PY@tc##1{\textcolor[rgb]{0.73,0.73,0.73}{##1}}}
\expandafter\def\csname PY@tok@c\endcsname{\let\PY@it=\textit\def\PY@tc##1{\textcolor[rgb]{0.25,0.50,0.50}{##1}}}
\expandafter\def\csname PY@tok@cp\endcsname{\def\PY@tc##1{\textcolor[rgb]{0.74,0.48,0.00}{##1}}}
\expandafter\def\csname PY@tok@k\endcsname{\let\PY@bf=\textbf\def\PY@tc##1{\textcolor[rgb]{0.00,0.50,0.00}{##1}}}
\expandafter\def\csname PY@tok@kp\endcsname{\def\PY@tc##1{\textcolor[rgb]{0.00,0.50,0.00}{##1}}}
\expandafter\def\csname PY@tok@kt\endcsname{\def\PY@tc##1{\textcolor[rgb]{0.69,0.00,0.25}{##1}}}
\expandafter\def\csname PY@tok@o\endcsname{\def\PY@tc##1{\textcolor[rgb]{0.40,0.40,0.40}{##1}}}
\expandafter\def\csname PY@tok@ow\endcsname{\let\PY@bf=\textbf\def\PY@tc##1{\textcolor[rgb]{0.67,0.13,1.00}{##1}}}
\expandafter\def\csname PY@tok@nb\endcsname{\def\PY@tc##1{\textcolor[rgb]{0.00,0.50,0.00}{##1}}}
\expandafter\def\csname PY@tok@nf\endcsname{\def\PY@tc##1{\textcolor[rgb]{0.00,0.00,1.00}{##1}}}
\expandafter\def\csname PY@tok@nc\endcsname{\let\PY@bf=\textbf\def\PY@tc##1{\textcolor[rgb]{0.00,0.00,1.00}{##1}}}
\expandafter\def\csname PY@tok@nn\endcsname{\let\PY@bf=\textbf\def\PY@tc##1{\textcolor[rgb]{0.00,0.00,1.00}{##1}}}
\expandafter\def\csname PY@tok@ne\endcsname{\let\PY@bf=\textbf\def\PY@tc##1{\textcolor[rgb]{0.82,0.25,0.23}{##1}}}
\expandafter\def\csname PY@tok@nv\endcsname{\def\PY@tc##1{\textcolor[rgb]{0.10,0.09,0.49}{##1}}}
\expandafter\def\csname PY@tok@no\endcsname{\def\PY@tc##1{\textcolor[rgb]{0.53,0.00,0.00}{##1}}}
\expandafter\def\csname PY@tok@nl\endcsname{\def\PY@tc##1{\textcolor[rgb]{0.63,0.63,0.00}{##1}}}
\expandafter\def\csname PY@tok@ni\endcsname{\let\PY@bf=\textbf\def\PY@tc##1{\textcolor[rgb]{0.60,0.60,0.60}{##1}}}
\expandafter\def\csname PY@tok@na\endcsname{\def\PY@tc##1{\textcolor[rgb]{0.49,0.56,0.16}{##1}}}
\expandafter\def\csname PY@tok@nt\endcsname{\let\PY@bf=\textbf\def\PY@tc##1{\textcolor[rgb]{0.00,0.50,0.00}{##1}}}
\expandafter\def\csname PY@tok@nd\endcsname{\def\PY@tc##1{\textcolor[rgb]{0.67,0.13,1.00}{##1}}}
\expandafter\def\csname PY@tok@s\endcsname{\def\PY@tc##1{\textcolor[rgb]{0.73,0.13,0.13}{##1}}}
\expandafter\def\csname PY@tok@sd\endcsname{\let\PY@it=\textit\def\PY@tc##1{\textcolor[rgb]{0.73,0.13,0.13}{##1}}}
\expandafter\def\csname PY@tok@si\endcsname{\let\PY@bf=\textbf\def\PY@tc##1{\textcolor[rgb]{0.73,0.40,0.53}{##1}}}
\expandafter\def\csname PY@tok@se\endcsname{\let\PY@bf=\textbf\def\PY@tc##1{\textcolor[rgb]{0.73,0.40,0.13}{##1}}}
\expandafter\def\csname PY@tok@sr\endcsname{\def\PY@tc##1{\textcolor[rgb]{0.73,0.40,0.53}{##1}}}
\expandafter\def\csname PY@tok@ss\endcsname{\def\PY@tc##1{\textcolor[rgb]{0.10,0.09,0.49}{##1}}}
\expandafter\def\csname PY@tok@sx\endcsname{\def\PY@tc##1{\textcolor[rgb]{0.00,0.50,0.00}{##1}}}
\expandafter\def\csname PY@tok@m\endcsname{\def\PY@tc##1{\textcolor[rgb]{0.40,0.40,0.40}{##1}}}
\expandafter\def\csname PY@tok@gh\endcsname{\let\PY@bf=\textbf\def\PY@tc##1{\textcolor[rgb]{0.00,0.00,0.50}{##1}}}
\expandafter\def\csname PY@tok@gu\endcsname{\let\PY@bf=\textbf\def\PY@tc##1{\textcolor[rgb]{0.50,0.00,0.50}{##1}}}
\expandafter\def\csname PY@tok@gd\endcsname{\def\PY@tc##1{\textcolor[rgb]{0.63,0.00,0.00}{##1}}}
\expandafter\def\csname PY@tok@gi\endcsname{\def\PY@tc##1{\textcolor[rgb]{0.00,0.63,0.00}{##1}}}
\expandafter\def\csname PY@tok@gr\endcsname{\def\PY@tc##1{\textcolor[rgb]{1.00,0.00,0.00}{##1}}}
\expandafter\def\csname PY@tok@ge\endcsname{\let\PY@it=\textit}
\expandafter\def\csname PY@tok@gs\endcsname{\let\PY@bf=\textbf}
\expandafter\def\csname PY@tok@gp\endcsname{\let\PY@bf=\textbf\def\PY@tc##1{\textcolor[rgb]{0.00,0.00,0.50}{##1}}}
\expandafter\def\csname PY@tok@go\endcsname{\def\PY@tc##1{\textcolor[rgb]{0.53,0.53,0.53}{##1}}}
\expandafter\def\csname PY@tok@gt\endcsname{\def\PY@tc##1{\textcolor[rgb]{0.00,0.27,0.87}{##1}}}
\expandafter\def\csname PY@tok@err\endcsname{\def\PY@bc##1{\setlength{\fboxsep}{0pt}\fcolorbox[rgb]{1.00,0.00,0.00}{1,1,1}{\strut ##1}}}
\expandafter\def\csname PY@tok@kc\endcsname{\let\PY@bf=\textbf\def\PY@tc##1{\textcolor[rgb]{0.00,0.50,0.00}{##1}}}
\expandafter\def\csname PY@tok@kd\endcsname{\let\PY@bf=\textbf\def\PY@tc##1{\textcolor[rgb]{0.00,0.50,0.00}{##1}}}
\expandafter\def\csname PY@tok@kn\endcsname{\let\PY@bf=\textbf\def\PY@tc##1{\textcolor[rgb]{0.00,0.50,0.00}{##1}}}
\expandafter\def\csname PY@tok@kr\endcsname{\let\PY@bf=\textbf\def\PY@tc##1{\textcolor[rgb]{0.00,0.50,0.00}{##1}}}
\expandafter\def\csname PY@tok@bp\endcsname{\def\PY@tc##1{\textcolor[rgb]{0.00,0.50,0.00}{##1}}}
\expandafter\def\csname PY@tok@fm\endcsname{\def\PY@tc##1{\textcolor[rgb]{0.00,0.00,1.00}{##1}}}
\expandafter\def\csname PY@tok@vc\endcsname{\def\PY@tc##1{\textcolor[rgb]{0.10,0.09,0.49}{##1}}}
\expandafter\def\csname PY@tok@vg\endcsname{\def\PY@tc##1{\textcolor[rgb]{0.10,0.09,0.49}{##1}}}
\expandafter\def\csname PY@tok@vi\endcsname{\def\PY@tc##1{\textcolor[rgb]{0.10,0.09,0.49}{##1}}}
\expandafter\def\csname PY@tok@vm\endcsname{\def\PY@tc##1{\textcolor[rgb]{0.10,0.09,0.49}{##1}}}
\expandafter\def\csname PY@tok@sa\endcsname{\def\PY@tc##1{\textcolor[rgb]{0.73,0.13,0.13}{##1}}}
\expandafter\def\csname PY@tok@sb\endcsname{\def\PY@tc##1{\textcolor[rgb]{0.73,0.13,0.13}{##1}}}
\expandafter\def\csname PY@tok@sc\endcsname{\def\PY@tc##1{\textcolor[rgb]{0.73,0.13,0.13}{##1}}}
\expandafter\def\csname PY@tok@dl\endcsname{\def\PY@tc##1{\textcolor[rgb]{0.73,0.13,0.13}{##1}}}
\expandafter\def\csname PY@tok@s2\endcsname{\def\PY@tc##1{\textcolor[rgb]{0.73,0.13,0.13}{##1}}}
\expandafter\def\csname PY@tok@sh\endcsname{\def\PY@tc##1{\textcolor[rgb]{0.73,0.13,0.13}{##1}}}
\expandafter\def\csname PY@tok@s1\endcsname{\def\PY@tc##1{\textcolor[rgb]{0.73,0.13,0.13}{##1}}}
\expandafter\def\csname PY@tok@mb\endcsname{\def\PY@tc##1{\textcolor[rgb]{0.40,0.40,0.40}{##1}}}
\expandafter\def\csname PY@tok@mf\endcsname{\def\PY@tc##1{\textcolor[rgb]{0.40,0.40,0.40}{##1}}}
\expandafter\def\csname PY@tok@mh\endcsname{\def\PY@tc##1{\textcolor[rgb]{0.40,0.40,0.40}{##1}}}
\expandafter\def\csname PY@tok@mi\endcsname{\def\PY@tc##1{\textcolor[rgb]{0.40,0.40,0.40}{##1}}}
\expandafter\def\csname PY@tok@il\endcsname{\def\PY@tc##1{\textcolor[rgb]{0.40,0.40,0.40}{##1}}}
\expandafter\def\csname PY@tok@mo\endcsname{\def\PY@tc##1{\textcolor[rgb]{0.40,0.40,0.40}{##1}}}
\expandafter\def\csname PY@tok@ch\endcsname{\let\PY@it=\textit\def\PY@tc##1{\textcolor[rgb]{0.25,0.50,0.50}{##1}}}
\expandafter\def\csname PY@tok@cm\endcsname{\let\PY@it=\textit\def\PY@tc##1{\textcolor[rgb]{0.25,0.50,0.50}{##1}}}
\expandafter\def\csname PY@tok@cpf\endcsname{\let\PY@it=\textit\def\PY@tc##1{\textcolor[rgb]{0.25,0.50,0.50}{##1}}}
\expandafter\def\csname PY@tok@c1\endcsname{\let\PY@it=\textit\def\PY@tc##1{\textcolor[rgb]{0.25,0.50,0.50}{##1}}}
\expandafter\def\csname PY@tok@cs\endcsname{\let\PY@it=\textit\def\PY@tc##1{\textcolor[rgb]{0.25,0.50,0.50}{##1}}}

\def\PYZbs{\char`\\}
\def\PYZus{\char`\_}
\def\PYZob{\char`\{}
\def\PYZcb{\char`\}}
\def\PYZca{\char`\^}
\def\PYZam{\char`\&}
\def\PYZlt{\char`\<}
\def\PYZgt{\char`\>}
\def\PYZsh{\char`\#}
\def\PYZpc{\char`\%}
\def\PYZdl{\char`\$}
\def\PYZhy{\char`\-}
\def\PYZsq{\char`\'}
\def\PYZdq{\char`\"}
\def\PYZti{\char`\~}
% for compatibility with earlier versions
\def\PYZat{@}
\def\PYZlb{[}
\def\PYZrb{]}
\makeatother


    % For linebreaks inside Verbatim environment from package fancyvrb. 
    \makeatletter
        \newbox\Wrappedcontinuationbox 
        \newbox\Wrappedvisiblespacebox 
        \newcommand*\Wrappedvisiblespace {\textcolor{red}{\textvisiblespace}} 
        \newcommand*\Wrappedcontinuationsymbol {\textcolor{red}{\llap{\tiny$\m@th\hookrightarrow$}}} 
        \newcommand*\Wrappedcontinuationindent {3ex } 
        \newcommand*\Wrappedafterbreak {\kern\Wrappedcontinuationindent\copy\Wrappedcontinuationbox} 
        % Take advantage of the already applied Pygments mark-up to insert 
        % potential linebreaks for TeX processing. 
        %        {, <, #, %, $, ' and ": go to next line. 
        %        _, }, ^, &, >, - and ~: stay at end of broken line. 
        % Use of \textquotesingle for straight quote. 
        \newcommand*\Wrappedbreaksatspecials {% 
            \def\PYGZus{\discretionary{\char`\_}{\Wrappedafterbreak}{\char`\_}}% 
            \def\PYGZob{\discretionary{}{\Wrappedafterbreak\char`\{}{\char`\{}}% 
            \def\PYGZcb{\discretionary{\char`\}}{\Wrappedafterbreak}{\char`\}}}% 
            \def\PYGZca{\discretionary{\char`\^}{\Wrappedafterbreak}{\char`\^}}% 
            \def\PYGZam{\discretionary{\char`\&}{\Wrappedafterbreak}{\char`\&}}% 
            \def\PYGZlt{\discretionary{}{\Wrappedafterbreak\char`\<}{\char`\<}}% 
            \def\PYGZgt{\discretionary{\char`\>}{\Wrappedafterbreak}{\char`\>}}% 
            \def\PYGZsh{\discretionary{}{\Wrappedafterbreak\char`\#}{\char`\#}}% 
            \def\PYGZpc{\discretionary{}{\Wrappedafterbreak\char`\%}{\char`\%}}% 
            \def\PYGZdl{\discretionary{}{\Wrappedafterbreak\char`\$}{\char`\$}}% 
            \def\PYGZhy{\discretionary{\char`\-}{\Wrappedafterbreak}{\char`\-}}% 
            \def\PYGZsq{\discretionary{}{\Wrappedafterbreak\textquotesingle}{\textquotesingle}}% 
            \def\PYGZdq{\discretionary{}{\Wrappedafterbreak\char`\"}{\char`\"}}% 
            \def\PYGZti{\discretionary{\char`\~}{\Wrappedafterbreak}{\char`\~}}% 
        } 
        % Some characters . , ; ? ! / are not pygmentized. 
        % This macro makes them "active" and they will insert potential linebreaks 
        \newcommand*\Wrappedbreaksatpunct {% 
            \lccode`\~`\.\lowercase{\def~}{\discretionary{\hbox{\char`\.}}{\Wrappedafterbreak}{\hbox{\char`\.}}}% 
            \lccode`\~`\,\lowercase{\def~}{\discretionary{\hbox{\char`\,}}{\Wrappedafterbreak}{\hbox{\char`\,}}}% 
            \lccode`\~`\;\lowercase{\def~}{\discretionary{\hbox{\char`\;}}{\Wrappedafterbreak}{\hbox{\char`\;}}}% 
            \lccode`\~`\:\lowercase{\def~}{\discretionary{\hbox{\char`\:}}{\Wrappedafterbreak}{\hbox{\char`\:}}}% 
            \lccode`\~`\?\lowercase{\def~}{\discretionary{\hbox{\char`\?}}{\Wrappedafterbreak}{\hbox{\char`\?}}}% 
            \lccode`\~`\!\lowercase{\def~}{\discretionary{\hbox{\char`\!}}{\Wrappedafterbreak}{\hbox{\char`\!}}}% 
            \lccode`\~`\/\lowercase{\def~}{\discretionary{\hbox{\char`\/}}{\Wrappedafterbreak}{\hbox{\char`\/}}}% 
            \catcode`\.\active
            \catcode`\,\active 
            \catcode`\;\active
            \catcode`\:\active
            \catcode`\?\active
            \catcode`\!\active
            \catcode`\/\active 
            \lccode`\~`\~ 	
        }
    \makeatother

    \let\OriginalVerbatim=\Verbatim
    \makeatletter
    \renewcommand{\Verbatim}[1][1]{%
        %\parskip\z@skip
        \sbox\Wrappedcontinuationbox {\Wrappedcontinuationsymbol}%
        \sbox\Wrappedvisiblespacebox {\FV@SetupFont\Wrappedvisiblespace}%
        \def\FancyVerbFormatLine ##1{\hsize\linewidth
            \vtop{\raggedright\hyphenpenalty\z@\exhyphenpenalty\z@
                \doublehyphendemerits\z@\finalhyphendemerits\z@
                \strut ##1\strut}%
        }%
        % If the linebreak is at a space, the latter will be displayed as visible
        % space at end of first line, and a continuation symbol starts next line.
        % Stretch/shrink are however usually zero for typewriter font.
        \def\FV@Space {%
            \nobreak\hskip\z@ plus\fontdimen3\font minus\fontdimen4\font
            \discretionary{\copy\Wrappedvisiblespacebox}{\Wrappedafterbreak}
            {\kern\fontdimen2\font}%
        }%
        
        % Allow breaks at special characters using \PYG... macros.
        \Wrappedbreaksatspecials
        % Breaks at punctuation characters . , ; ? ! and / need catcode=\active 	
        \OriginalVerbatim[#1,codes*=\Wrappedbreaksatpunct]%
    }
    \makeatother

    % Exact colors from NB
    \definecolor{incolor}{HTML}{303F9F}
    \definecolor{outcolor}{HTML}{D84315}
    \definecolor{cellborder}{HTML}{CFCFCF}
    \definecolor{cellbackground}{HTML}{F7F7F7}
    
    % prompt
    \makeatletter
    \newcommand{\boxspacing}{\kern\kvtcb@left@rule\kern\kvtcb@boxsep}
    \makeatother
    \newcommand{\prompt}[4]{
        \ttfamily\llap{{\color{#2}[#3]:\hspace{3pt}#4}}\vspace{-\baselineskip}
    }
    

    
    % Prevent overflowing lines due to hard-to-break entities
    \sloppy 
    % Setup hyperref package
    \hypersetup{
      breaklinks=true,  % so long urls are correctly broken across lines
      colorlinks=true,
      urlcolor=urlcolor,
      linkcolor=linkcolor,
      citecolor=citecolor,
      }
    % Slightly bigger margins than the latex defaults
    
    \geometry{verbose,tmargin=1in,bmargin=1in,lmargin=1in,rmargin=1in}
    
    

\begin{document}
    
    \maketitle
    
    

    
    \begin{tcolorbox}[breakable, size=fbox, boxrule=1pt, pad at break*=1mm,colback=cellbackground, colframe=cellborder]
\prompt{In}{incolor}{2}{\boxspacing}
\begin{Verbatim}[commandchars=\\\{\}]
\PY{k+kn}{import} \PY{n+nn}{numpy} \PY{k}{as} \PY{n+nn}{np}
\PY{k+kn}{import} \PY{n+nn}{matplotlib}\PY{n+nn}{.}\PY{n+nn}{pyplot} \PY{k}{as} \PY{n+nn}{plt}
\PY{k+kn}{from} \PY{n+nn}{scipy}\PY{n+nn}{.}\PY{n+nn}{optimize} \PY{k}{import} \PY{n}{curve\PYZus{}fit}
\PY{k+kn}{from} \PY{n+nn}{pylab} \PY{k}{import} \PY{o}{*}
\PY{k+kn}{from} \PY{n+nn}{io} \PY{k}{import} \PY{n}{StringIO}
\end{Verbatim}
\end{tcolorbox}

    \hypertarget{esperienza-1-termometria-e-calorimetria}{%
\section{Esperienza 1: Termometria e
calorimetria}\label{esperienza-1-termometria-e-calorimetria}}

\textbf{Data}: 21 Ottobre 2019 \textbf{Gruppo}: (V) Ivan , Antonio
Gonzalez, Pietro

    \hypertarget{materiale}{%
\subsection{Materiale}\label{materiale}}

\begin{longtable}[]{@{}rll@{}}
\toprule
Strumenti & Divisione & Portata\tabularnewline
\midrule
\endhead
2 Termometri a mercurio & \(0.2^\circ C\) &
\(100^\circ C\)\tabularnewline
Bilancia & \(0.1 g\) & --\tabularnewline
Calorimetri & -- & \(1 l\)\tabularnewline
Cronometro & \(0.01 s\) &\tabularnewline
\bottomrule
\end{longtable}

    \hypertarget{costante-di-tempo-del-termometro}{%
\subsection{1.1 Costante di tempo del
termometro}\label{costante-di-tempo-del-termometro}}

\hypertarget{relazioni-di-base-per-il-processo-ideale}{%
\subsubsection{Relazioni di base per il processo
ideale}\label{relazioni-di-base-per-il-processo-ideale}}

La relazione che lega la risposta del termometro al tempo è data dalla
seguente formula: \[T(t)=T_{f}+(T_{amb}-T_{f} )e ^{-\frac{t}{\tau}}\]
dove: - \(T(t)\) rappresenta la temperatura mostrata sul termometro
all'istante di tempo \(t\); - \(T_{amb}\) rappresenta la temperatura
riportata sul termometro prima che questo venga inserito nel bagno di
acqua cala, ovvero la temperatura al tempo \(t=0\); - \(T_{f}\)
rappresenta la temperatura dell'acqua calda; - \(\tau\) rappresenta la
costante di tempo del termometro che vogliamo stimare;

\hypertarget{procedimento-di-misura}{%
\subsubsection{Procedimento di misura}\label{procedimento-di-misura}}

\begin{enumerate}
\def\labelenumi{\arabic{enumi}.}
\tightlist
\item
  Nel primo calorimetro versiamo una quantità di acqua alla temperatura
  di \(\sim 54^\circ C\), che rappresenta la nostra \(T_{f}\); nel
  secondo una quantità di acqua a temperatura ambiente;
\item
  Immergiamo il termometro nel bagno di acqua a temperatura ambiente e
  aspettiamo che termalizzi con l'acqua stessa; a termalizzazione
  avvenuta registriamo la temperatura segnata dal termometro come
  \(T_{amb}\)
\item
  Immergiamo il termometro nel calorimetro con l'acqua calda e
  registriamo la temperatura segnta ad intervalli di tempo fissati
  (\(0.5s\));
\end{enumerate}

Effettuiamo l'analisi dei dati servendoci di Python.

    \hypertarget{dati}{%
\subsubsection{Dati}\label{dati}}

\texttt{t}: tempo (s)\\
\texttt{T}: temperatura ( \(^\circ C\) )\\
\texttt{DT}: risoluzione del termometro (distanza tra due tacche)\\
\texttt{sT}: incertezza (deviazione standard) su T

    Eseguiamo un fit sui dati raccolti attraversp la funzione curve\_fit
della libreria scipy. curve\_fit riceve in ingresso una funzione che
restituisce un esponenziale, un vettore contente gli istanti di tempo ad
intervalli di \(0.5s\) (asse \(x\)), un vettore contenente le
temperature registrate (asse \(y\)), e un vettore con lo stesso numero
di elementi dei vettori precedenti e contente l'incertezza associata a
ogni misura di temperatura. Assumendo una distribuizione uniforme
nell'intervallo delle divisioni stimiaimo tale incertezza come
\(\frac{DT}{\sqrt{12}}\). L'ultimo parametro di curve\_fit viene
dichiarato TRUE in modo da calcolare le incertezze assolute.

    \hypertarget{video-1-1s}{%
\paragraph{Video 1, 1s}\label{video-1-1s}}

    \begin{tcolorbox}[breakable, size=fbox, boxrule=1pt, pad at break*=1mm,colback=cellbackground, colframe=cellborder]
\prompt{In}{incolor}{15}{\boxspacing}
\begin{Verbatim}[commandchars=\\\{\}]
\PY{n}{s} \PY{o}{=} \PY{n+nb}{open}\PY{p}{(}\PY{l+s+s2}{\PYZdq{}}\PY{l+s+s2}{Datos12.txt}\PY{l+s+s2}{\PYZdq{}}\PY{p}{)}\PY{o}{.}\PY{n}{read}\PY{p}{(}\PY{p}{)}\PY{o}{.}\PY{n}{replace}\PY{p}{(}\PY{l+s+s2}{\PYZdq{}}\PY{l+s+s2}{,}\PY{l+s+s2}{\PYZdq{}}\PY{p}{,} \PY{l+s+s2}{\PYZdq{}}\PY{l+s+s2}{.}\PY{l+s+s2}{\PYZdq{}}\PY{p}{)}
\PY{n}{t}\PY{p}{,}\PY{n}{T} \PY{o}{=} \PY{n}{transpose}\PY{p}{(}\PY{n}{loadtxt}\PY{p}{(}\PY{n}{StringIO}\PY{p}{(}\PY{n}{s}\PY{p}{)}\PY{p}{)}\PY{p}{)}
\PY{n}{DT}\PY{o}{=} \PY{l+m+mf}{0.2}
\PY{n}{dS}\PY{o}{=}\PY{n}{DT}\PY{o}{/}\PY{n}{sqrt}\PY{p}{(}\PY{l+m+mi}{12}\PY{p}{)}\PY{o}{*}\PY{n}{np}\PY{o}{.}\PY{n}{ones\PYZus{}like}\PY{p}{(}\PY{n}{T}\PY{p}{)}
\PY{k}{def} \PY{n+nf}{fit\PYZus{}func}\PY{p}{(}\PY{n}{t}\PY{p}{,} \PY{n}{tau}\PY{p}{,}\PY{n}{TF} \PY{p}{,}\PY{n}{DeltaT}\PY{p}{)}\PY{p}{:}
    \PY{k}{return} \PY{n}{TF} \PY{o}{\PYZhy{}} \PY{n}{DeltaT}\PY{o}{*}\PY{n}{exp}\PY{p}{(}\PY{o}{\PYZhy{}}\PY{n}{t}\PY{o}{/}\PY{n}{tau}\PY{p}{)}
\PY{n}{params}\PY{p}{,}\PY{n}{pcov12} \PY{o}{=} \PY{n}{curve\PYZus{}fit}\PY{p}{(}\PY{n}{fit\PYZus{}func}\PY{p}{,} \PY{n}{t}\PY{p}{,} \PY{n}{T}\PY{p}{,}\PY{n}{sigma}\PY{o}{=}\PY{n}{dS}\PY{p}{,} \PY{n}{absolute\PYZus{}sigma}\PY{o}{=}\PY{k+kc}{True}\PY{p}{)}
\PY{n}{tau12}\PY{p}{,} \PY{n}{T0}\PY{p}{,} \PY{n}{DeltaT} \PY{o}{=} \PY{n}{params}
\PY{n}{plot}\PY{p}{(}\PY{n}{t}\PY{p}{,} \PY{n}{T}\PY{p}{,} \PY{l+s+s2}{\PYZdq{}}\PY{l+s+s2}{o}\PY{l+s+s2}{\PYZdq{}}\PY{p}{)}
\PY{n}{plot}\PY{p}{(}\PY{n}{t}\PY{p}{,} \PY{n}{fit\PYZus{}func}\PY{p}{(}\PY{n}{t}\PY{p}{,} \PY{n}{tau12}\PY{p}{,} \PY{n}{T0}\PY{p}{,} \PY{n}{DeltaT}\PY{p}{)}\PY{p}{)}
\PY{n}{xlabel}\PY{p}{(}\PY{l+s+s2}{\PYZdq{}}\PY{l+s+s2}{t (s)}\PY{l+s+s2}{\PYZdq{}}\PY{p}{)}
\PY{n}{ylabel}\PY{p}{(}\PY{l+s+s2}{\PYZdq{}}\PY{l+s+s2}{T (\PYZdl{}\PYZca{}}\PY{l+s+s2}{\PYZbs{}}\PY{l+s+s2}{circ\PYZdl{}C)}\PY{l+s+s2}{\PYZdq{}}\PY{p}{)}
\PY{n}{plt}\PY{o}{.}\PY{n}{show}\PY{p}{(}\PY{p}{)}
\PY{n+nb}{print}\PY{p}{(}\PY{n}{params}\PY{p}{)}
\end{Verbatim}
\end{tcolorbox}

    \begin{center}
    \adjustimage{max size={0.9\linewidth}{0.9\paperheight}}{./Esperienza_files/./Esperienza_7_0.pdf}
    \end{center}
    { \hspace*{\fill} \\}
    
    \begin{Verbatim}[commandchars=\\\{\}]
[10.15223516 54.40966347 24.39024659]
    \end{Verbatim}

    \begin{tcolorbox}[breakable, size=fbox, boxrule=1pt, pad at break*=1mm,colback=cellbackground, colframe=cellborder]
\prompt{In}{incolor}{16}{\boxspacing}
\begin{Verbatim}[commandchars=\\\{\}]
\PY{n+nb}{print}\PY{p}{(}\PY{l+s+s2}{\PYZdq{}}\PY{l+s+s2}{tau = }\PY{l+s+si}{\PYZpc{}.2f}\PY{l+s+s2}{ +/\PYZhy{} }\PY{l+s+si}{\PYZpc{}.2f}\PY{l+s+s2}{ s}\PY{l+s+s2}{\PYZdq{}} \PY{o}{\PYZpc{}} \PY{p}{(}\PY{n}{tau12}\PY{p}{,} \PY{n}{sqrt}\PY{p}{(}\PY{n}{pcov12}\PY{p}{[}\PY{l+m+mi}{0}\PY{p}{,}\PY{l+m+mi}{0}\PY{p}{]}\PY{p}{)}\PY{p}{)}\PY{p}{)}
\end{Verbatim}
\end{tcolorbox}

    \begin{Verbatim}[commandchars=\\\{\}]
tau = 10.15 +/- 0.04 s
    \end{Verbatim}

    \hypertarget{video-2-1-s}{%
\paragraph{Video 2, 1 s}\label{video-2-1-s}}

    \begin{tcolorbox}[breakable, size=fbox, boxrule=1pt, pad at break*=1mm,colback=cellbackground, colframe=cellborder]
\prompt{In}{incolor}{17}{\boxspacing}
\begin{Verbatim}[commandchars=\\\{\}]
\PY{n}{s} \PY{o}{=} \PY{n+nb}{open}\PY{p}{(}\PY{l+s+s2}{\PYZdq{}}\PY{l+s+s2}{Datos22.txt}\PY{l+s+s2}{\PYZdq{}}\PY{p}{)}\PY{o}{.}\PY{n}{read}\PY{p}{(}\PY{p}{)}\PY{o}{.}\PY{n}{replace}\PY{p}{(}\PY{l+s+s2}{\PYZdq{}}\PY{l+s+s2}{,}\PY{l+s+s2}{\PYZdq{}}\PY{p}{,} \PY{l+s+s2}{\PYZdq{}}\PY{l+s+s2}{.}\PY{l+s+s2}{\PYZdq{}}\PY{p}{)}
\PY{n}{t}\PY{p}{,}\PY{n}{T} \PY{o}{=} \PY{n}{transpose}\PY{p}{(}\PY{n}{loadtxt}\PY{p}{(}\PY{n}{StringIO}\PY{p}{(}\PY{n}{s}\PY{p}{)}\PY{p}{)}\PY{p}{)}
\PY{n}{DT}\PY{o}{=} \PY{l+m+mf}{0.2}
\PY{n}{dS}\PY{o}{=}\PY{n}{DT}\PY{o}{/}\PY{n}{sqrt}\PY{p}{(}\PY{l+m+mi}{12}\PY{p}{)}\PY{o}{*}\PY{n}{np}\PY{o}{.}\PY{n}{ones\PYZus{}like}\PY{p}{(}\PY{n}{T}\PY{p}{)}
\PY{k}{def} \PY{n+nf}{fit\PYZus{}func}\PY{p}{(}\PY{n}{t}\PY{p}{,} \PY{n}{tau}\PY{p}{,}\PY{n}{TF} \PY{p}{,}\PY{n}{DeltaT}\PY{p}{)}\PY{p}{:}
    \PY{k}{return} \PY{n}{TF} \PY{o}{\PYZhy{}} \PY{n}{DeltaT}\PY{o}{*}\PY{n}{exp}\PY{p}{(}\PY{o}{\PYZhy{}}\PY{n}{t}\PY{o}{/}\PY{n}{tau}\PY{p}{)}
\PY{n}{params}\PY{p}{,}\PY{n}{pcov22} \PY{o}{=} \PY{n}{curve\PYZus{}fit}\PY{p}{(}\PY{n}{fit\PYZus{}func}\PY{p}{,} \PY{n}{t}\PY{p}{,} \PY{n}{T}\PY{p}{,}\PY{n}{sigma}\PY{o}{=}\PY{n}{dS}\PY{p}{,} \PY{n}{absolute\PYZus{}sigma}\PY{o}{=}\PY{k+kc}{True}\PY{p}{)}
\PY{n}{tau22}\PY{p}{,} \PY{n}{T0}\PY{p}{,} \PY{n}{DeltaT} \PY{o}{=} \PY{n}{params}
\PY{n}{plot}\PY{p}{(}\PY{n}{t}\PY{p}{,} \PY{n}{T}\PY{p}{,} \PY{l+s+s2}{\PYZdq{}}\PY{l+s+s2}{o}\PY{l+s+s2}{\PYZdq{}}\PY{p}{)}
\PY{n}{plot}\PY{p}{(}\PY{n}{t}\PY{p}{,} \PY{n}{fit\PYZus{}func}\PY{p}{(}\PY{n}{t}\PY{p}{,} \PY{n}{tau22}\PY{p}{,} \PY{n}{T0}\PY{p}{,} \PY{n}{DeltaT}\PY{p}{)}\PY{p}{)}
\PY{n}{xlabel}\PY{p}{(}\PY{l+s+s2}{\PYZdq{}}\PY{l+s+s2}{t (s)}\PY{l+s+s2}{\PYZdq{}}\PY{p}{)}
\PY{n}{ylabel}\PY{p}{(}\PY{l+s+s2}{\PYZdq{}}\PY{l+s+s2}{T (\PYZdl{}\PYZca{}}\PY{l+s+s2}{\PYZbs{}}\PY{l+s+s2}{circ\PYZdl{}C)}\PY{l+s+s2}{\PYZdq{}}\PY{p}{)}
\PY{n}{plt}\PY{o}{.}\PY{n}{show}\PY{p}{(}\PY{p}{)}
\end{Verbatim}
\end{tcolorbox}

    \begin{center}
    \adjustimage{max size={0.9\linewidth}{0.9\paperheight}}{./Esperienza_files/./Esperienza_10_0.pdf}
    \end{center}
    { \hspace*{\fill} \\}
    
    \begin{tcolorbox}[breakable, size=fbox, boxrule=1pt, pad at break*=1mm,colback=cellbackground, colframe=cellborder]
\prompt{In}{incolor}{18}{\boxspacing}
\begin{Verbatim}[commandchars=\\\{\}]
\PY{n+nb}{print}\PY{p}{(}\PY{l+s+s2}{\PYZdq{}}\PY{l+s+s2}{tau = }\PY{l+s+si}{\PYZpc{}.2f}\PY{l+s+s2}{ +/\PYZhy{} }\PY{l+s+si}{\PYZpc{}.2f}\PY{l+s+s2}{ s}\PY{l+s+s2}{\PYZdq{}} \PY{o}{\PYZpc{}} \PY{p}{(}\PY{n}{tau22}\PY{p}{,} \PY{n}{sqrt}\PY{p}{(}\PY{n}{pcov22}\PY{p}{[}\PY{l+m+mi}{0}\PY{p}{,}\PY{l+m+mi}{0}\PY{p}{]}\PY{p}{)}\PY{p}{)}\PY{p}{)}
\end{Verbatim}
\end{tcolorbox}

    \begin{Verbatim}[commandchars=\\\{\}]
tau = 9.70 +/- 0.02 s
    \end{Verbatim}

    \begin{tcolorbox}[breakable, size=fbox, boxrule=1pt, pad at break*=1mm,colback=cellbackground, colframe=cellborder]
\prompt{In}{incolor}{19}{\boxspacing}
\begin{Verbatim}[commandchars=\\\{\}]
\PY{n}{N2} \PY{o}{=} \PY{n}{tau12}\PY{o}{/}\PY{n}{pcov12}\PY{p}{[}\PY{l+m+mi}{0}\PY{p}{,}\PY{l+m+mi}{0}\PY{p}{]}\PY{o}{+}\PY{n}{tau22}\PY{o}{/}\PY{n}{pcov22}\PY{p}{[}\PY{l+m+mi}{0}\PY{p}{,}\PY{l+m+mi}{0}\PY{p}{]}
\PY{n}{D2}\PY{o}{=}\PY{l+m+mi}{1}\PY{o}{/}\PY{n}{pcov12}\PY{p}{[}\PY{l+m+mi}{0}\PY{p}{,}\PY{l+m+mi}{0}\PY{p}{]}\PY{o}{+}\PY{l+m+mi}{1}\PY{o}{/}\PY{n}{pcov22}\PY{p}{[}\PY{l+m+mi}{0}\PY{p}{,}\PY{l+m+mi}{0}\PY{p}{]}
\PY{n}{P2}\PY{o}{=}\PY{l+m+mi}{1}\PY{o}{/}\PY{n}{sqrt}\PY{p}{(}\PY{n}{D2}\PY{p}{)}
\PY{n+nb}{print}\PY{p}{(}\PY{l+s+s2}{\PYZdq{}}\PY{l+s+s2}{tau = }\PY{l+s+si}{\PYZpc{}.2f}\PY{l+s+s2}{ +/\PYZhy{} }\PY{l+s+si}{\PYZpc{}.2f}\PY{l+s+s2}{ s}\PY{l+s+s2}{\PYZdq{}} \PY{o}{\PYZpc{}} \PY{p}{(}\PY{n}{N2}\PY{o}{/}\PY{n}{D2}\PY{p}{,}\PY{n}{P2} \PY{p}{)}\PY{p}{)}
\PY{c+c1}{\PYZsh{}\PYZsh{}\PYZsh{}\PYZsh{}\PYZsh{}Tau  videos a 1.0s\PYZsh{}\PYZsh{}\PYZsh{}\PYZsh{}\PYZsh{}\PYZsh{}}
\end{Verbatim}
\end{tcolorbox}

    \begin{Verbatim}[commandchars=\\\{\}]
tau = 9.83 +/- 0.02 s
    \end{Verbatim}

    \hypertarget{calcolo-attraverso-la-legge-di-raffreddamento-di-newton}{%
\subsubsection{Calcolo attraverso la legge di raffreddamento di
Newton}\label{calcolo-attraverso-la-legge-di-raffreddamento-di-newton}}

E' possibile calcolare analiticamente il valore di \(\tau\) attraverso
la legge di raffreddamento di Newton \[ \tau=\frac{C}{hA}\] dove - C é
la capacitá termica del mercurio, che possiamo calolare come attraverso
il calore specifico e la massa, il primo noto, la seconda ricavata dalla
relazione \(v\rho = m\). Sapendo che \(\rho = 1.3\cdot 10^4 Kg/m^3\) e,
assumendo una forma cilindrica per il bulbo, dati l'altezza \$ l = 11 mm
\$ e il diametro \(d= 6 mm\) del bulbo, si ottiene che \[ C=cv\rho\] - h
é il coefficiente di convezione in acqua statica e vale \$ 750 W/m\^{}2K
\$; - A é la superficie del bulbo

    \begin{tcolorbox}[breakable, size=fbox, boxrule=1pt, pad at break*=1mm,colback=cellbackground, colframe=cellborder]
\prompt{In}{incolor}{20}{\boxspacing}
\begin{Verbatim}[commandchars=\\\{\}]
\PY{n}{c} \PY{o}{=} \PY{l+m+mi}{140}
\PY{n}{h} \PY{o}{=} \PY{l+m+mi}{750}
\PY{n}{r} \PY{o}{=} \PY{l+m+mf}{0.003}
\PY{n}{l} \PY{o}{=} \PY{l+m+mf}{0.011}
\PY{n}{rho} \PY{o}{=} \PY{l+m+mf}{1.3e4}

\PY{n}{tau} \PY{o}{=} \PY{n}{c}\PY{o}{*}\PY{n}{r}\PY{o}{*}\PY{o}{*}\PY{l+m+mi}{2}\PY{o}{*}\PY{n}{np}\PY{o}{.}\PY{n}{pi}\PY{o}{*}\PY{n}{l}\PY{o}{*}\PY{n}{rho}\PY{o}{/}\PY{p}{(}\PY{n}{h}\PY{o}{*}\PY{l+m+mi}{2}\PY{o}{*}\PY{n}{r}\PY{o}{*}\PY{n}{np}\PY{o}{.}\PY{n}{pi}\PY{o}{*}\PY{n}{l}\PY{p}{)}
\PY{n+nb}{print}\PY{p}{(}\PY{l+s+s2}{\PYZdq{}}\PY{l+s+s2}{tau = }\PY{l+s+si}{\PYZpc{}.1f}\PY{l+s+s2}{ s }\PY{l+s+s2}{\PYZdq{}} \PY{o}{\PYZpc{}} \PY{p}{(}\PY{n}{tau}\PY{p}{)}\PY{p}{)}
\end{Verbatim}
\end{tcolorbox}

    \begin{Verbatim}[commandchars=\\\{\}]
tau = 3.6 s
    \end{Verbatim}

    Confrontando i valori ottenuti dalle misure con il valore di \(\tau\)
calcolato analiticamente notiamo che questi valori sono decisamente poco
consistenti. Alla luce di questa inconsistenza ipotiziamo che il
termometro con il quale abbiamo effettuato le misure non sia un
termometro a mercurio.

    \hypertarget{calcolo-calore-specifico}{%
\subsection{1.2 Calcolo calore
specifico}\label{calcolo-calore-specifico}}

    \hypertarget{relazione-di-base}{%
\subsubsection{Relazione di base}\label{relazione-di-base}}

\ldots{}.

\hypertarget{procedimento-di-misura}{%
\subsubsection{Procedimento di misura}\label{procedimento-di-misura}}

\ldots{}.

\hypertarget{dati}{%
\subsubsection{Dati}\label{dati}}

\ldots{}.

    \begin{tcolorbox}[breakable, size=fbox, boxrule=1pt, pad at break*=1mm,colback=cellbackground, colframe=cellborder]
\prompt{In}{incolor}{22}{\boxspacing}
\begin{Verbatim}[commandchars=\\\{\}]
\PY{n}{Meq} \PY{o}{=} \PY{l+m+mf}{25.} 
\PY{n}{DMeq} \PY{o}{=} \PY{l+m+mf}{5.}
\PY{n}{Ma} \PY{o}{=} \PY{n}{np}\PY{o}{.}\PY{n}{array}\PY{p}{(}\PY{p}{[}\PY{l+m+mf}{202.}\PY{p}{,} \PY{l+m+mf}{202.8}\PY{p}{,} \PY{l+m+mf}{200.7}\PY{p}{]}\PY{p}{)}
\PY{n}{DMa} \PY{o}{=} \PY{l+m+mf}{0.1}\PY{o}{/}\PY{n}{np}\PY{o}{.}\PY{n}{sqrt}\PY{p}{(}\PY{l+m+mi}{12}\PY{p}{)}\PY{o}{*}\PY{n}{np}\PY{o}{.}\PY{n}{ones\PYZus{}like}\PY{p}{(}\PY{n}{Ma}\PY{p}{)}
\PY{n}{ca} \PY{o}{=} \PY{l+m+mf}{1.}
\PY{n}{Ta} \PY{o}{=} \PY{n}{np}\PY{o}{.}\PY{n}{array}\PY{p}{(}\PY{p}{[}\PY{l+m+mf}{42.2}\PY{p}{,} \PY{l+m+mf}{40.8}\PY{p}{,} \PY{l+m+mf}{39.6}\PY{p}{]}\PY{p}{)}
\PY{n}{DTa} \PY{o}{=} \PY{l+m+mf}{0.2}\PY{o}{/}\PY{n}{np}\PY{o}{.}\PY{n}{sqrt}\PY{p}{(}\PY{l+m+mi}{12}\PY{p}{)}\PY{o}{*}\PY{n}{np}\PY{o}{.}\PY{n}{ones\PYZus{}like}\PY{p}{(}\PY{n}{Ta}\PY{p}{)}
\PY{n}{Te} \PY{o}{=} \PY{n}{np}\PY{o}{.}\PY{n}{array}\PY{p}{(}\PY{p}{[}\PY{l+m+mf}{40.8}\PY{p}{,} \PY{l+m+mf}{39.}\PY{p}{,} \PY{l+m+mf}{38.2}\PY{p}{]}\PY{p}{)}
\PY{n}{DTe} \PY{o}{=} \PY{l+m+mf}{0.2}\PY{o}{/}\PY{n}{np}\PY{o}{.}\PY{n}{sqrt}\PY{p}{(}\PY{l+m+mi}{12}\PY{p}{)}\PY{o}{*}\PY{n}{np}\PY{o}{.}\PY{n}{ones\PYZus{}like}\PY{p}{(}\PY{n}{Te}\PY{p}{)}
\PY{n}{Mm} \PY{o}{=} \PY{n}{np}\PY{o}{.}\PY{n}{array}\PY{p}{(}\PY{p}{[}\PY{l+m+mf}{79.}\PY{p}{,} \PY{l+m+mf}{73.1}\PY{p}{,} \PY{l+m+mf}{194.4}\PY{p}{]}\PY{p}{)}
\PY{n}{DMm} \PY{o}{=} \PY{l+m+mf}{0.1}\PY{o}{/}\PY{n}{np}\PY{o}{.}\PY{n}{sqrt}\PY{p}{(}\PY{l+m+mi}{12}\PY{p}{)}\PY{o}{*}\PY{n}{np}\PY{o}{.}\PY{n}{ones\PYZus{}like}\PY{p}{(}\PY{n}{Mm}\PY{p}{)}
\PY{n}{Tm} \PY{o}{=} \PY{l+m+mf}{28.2}\PY{o}{*}\PY{n}{np}\PY{o}{.}\PY{n}{ones\PYZus{}like}\PY{p}{(}\PY{n}{Ta}\PY{p}{)}
\PY{n}{DTm} \PY{o}{=} \PY{l+m+mf}{0.2}\PY{o}{/}\PY{n}{np}\PY{o}{.}\PY{n}{sqrt}\PY{p}{(}\PY{l+m+mi}{12}\PY{p}{)}\PY{o}{*}\PY{n}{np}\PY{o}{.}\PY{n}{ones\PYZus{}like}\PY{p}{(}\PY{n}{Tm}\PY{p}{)}
\end{Verbatim}
\end{tcolorbox}

    \hypertarget{calcolo}{%
\paragraph{Calcolo}\label{calcolo}}

    \begin{tcolorbox}[breakable, size=fbox, boxrule=1pt, pad at break*=1mm,colback=cellbackground, colframe=cellborder]
\prompt{In}{incolor}{23}{\boxspacing}
\begin{Verbatim}[commandchars=\\\{\}]
\PY{n}{cm} \PY{o}{=} \PY{p}{(}\PY{n}{Meq} \PY{o}{+} \PY{n}{Ma}\PY{p}{)}\PY{o}{*}\PY{n}{ca}\PY{o}{*}\PY{p}{(}\PY{n}{Ta}\PY{o}{\PYZhy{}}\PY{n}{Te}\PY{p}{)}\PY{o}{/}\PY{p}{(}\PY{n}{Mm}\PY{o}{*}\PY{p}{(}\PY{n}{Te}\PY{o}{\PYZhy{}}\PY{n}{Tm}\PY{p}{)}\PY{p}{)}
\PY{n}{Dcm\PYZus{}Meq} \PY{o}{=} \PY{n}{ca}\PY{o}{*}\PY{p}{(}\PY{n}{Ta} \PY{o}{\PYZhy{}} \PY{n}{Te}\PY{p}{)}\PY{o}{/}\PY{p}{(}\PY{n}{Mm}\PY{o}{*}\PY{p}{(}\PY{n}{Te}\PY{o}{\PYZhy{}}\PY{n}{Tm}\PY{p}{)}\PY{p}{)}
\PY{n}{Dcm\PYZus{}Ma} \PY{o}{=} \PY{n}{ca}\PY{o}{*}\PY{p}{(}\PY{n}{Ta} \PY{o}{\PYZhy{}} \PY{n}{Te}\PY{p}{)}\PY{o}{/}\PY{p}{(}\PY{n}{Mm}\PY{o}{*}\PY{p}{(}\PY{n}{Te}\PY{o}{\PYZhy{}}\PY{n}{Tm}\PY{p}{)}\PY{p}{)}
\PY{n}{Dcm\PYZus{}Ta} \PY{o}{=} \PY{p}{(}\PY{n}{Meq}\PY{o}{+}\PY{n}{Ma}\PY{p}{)}\PY{o}{*}\PY{n}{ca}\PY{o}{/}\PY{p}{(}\PY{n}{Mm}\PY{o}{*}\PY{p}{(}\PY{n}{Te}\PY{o}{\PYZhy{}}\PY{n}{Tm}\PY{p}{)}\PY{p}{)}
\PY{n}{Dcm\PYZus{}Te} \PY{o}{=} \PY{p}{(}\PY{n}{Meq} \PY{o}{+} \PY{n}{Ma}\PY{p}{)}\PY{o}{*}\PY{n}{ca}\PY{o}{/}\PY{p}{(}\PY{n}{Mm}\PY{p}{)}\PY{o}{*}\PY{p}{(}\PY{n}{Tm} \PY{o}{+} \PY{n}{Ta}\PY{p}{)}\PY{o}{/}\PY{p}{(}\PY{p}{(}\PY{n}{Te}\PY{o}{\PYZhy{}} \PY{n}{Tm}\PY{p}{)}\PY{o}{*}\PY{o}{*}\PY{l+m+mi}{2}\PY{p}{)}
\PY{n}{Dcm\PYZus{}Mm} \PY{o}{=} \PY{o}{\PYZhy{}} \PY{p}{(}\PY{n}{Meq} \PY{o}{+} \PY{n}{Ma}\PY{p}{)}\PY{o}{*}\PY{n}{ca}\PY{o}{*}\PY{p}{(}\PY{n}{Ta}\PY{o}{\PYZhy{}}\PY{n}{Te}\PY{p}{)}\PY{o}{/}\PY{p}{(}\PY{p}{(}\PY{n}{Te}\PY{o}{\PYZhy{}}\PY{n}{Tm}\PY{p}{)}\PY{o}{*}\PY{n}{Mm}\PY{o}{*}\PY{o}{*}\PY{l+m+mi}{2}\PY{p}{)}
\PY{n}{Dcm\PYZus{}Tm} \PY{o}{=} \PY{p}{(}\PY{n}{Meq} \PY{o}{+} \PY{n}{Ma}\PY{p}{)}\PY{o}{*}\PY{n}{ca}\PY{o}{*}\PY{p}{(}\PY{n}{Ta}\PY{o}{\PYZhy{}}\PY{n}{Te}\PY{p}{)}\PY{o}{/}\PY{p}{(}\PY{n}{Mm}\PY{o}{*}\PY{p}{(}\PY{n}{Te} \PY{o}{\PYZhy{}} \PY{n}{Tm}\PY{p}{)}\PY{o}{*}\PY{o}{*}\PY{l+m+mi}{2}\PY{p}{)}
\PY{n}{Dcm} \PY{o}{=} \PY{n}{np}\PY{o}{.}\PY{n}{sqrt}\PY{p}{(}\PY{n}{Dcm\PYZus{}Meq}\PY{o}{*}\PY{o}{*}\PY{l+m+mi}{2}\PY{o}{*}\PY{n}{DMeq}\PY{o}{*}\PY{o}{*}\PY{l+m+mi}{2} \PY{o}{+} \PY{n}{Dcm\PYZus{}Ma}\PY{o}{*}\PY{o}{*}\PY{l+m+mi}{2}\PY{o}{*}\PY{n}{DMa}\PY{o}{*}\PY{o}{*}\PY{l+m+mi}{2} \PY{o}{+} \PY{n}{Dcm\PYZus{}Ta}\PY{o}{*}\PY{o}{*}\PY{l+m+mi}{2}\PY{o}{*}
      \PY{n}{DTa}\PY{o}{*}\PY{o}{*}\PY{l+m+mi}{2} \PY{o}{+} \PY{n}{Dcm\PYZus{}Te}\PY{o}{*}\PY{o}{*}\PY{l+m+mi}{2}\PY{o}{*}\PY{n}{DTe}\PY{o}{*}\PY{o}{*}\PY{l+m+mi}{2} \PY{o}{+} \PY{n}{Dcm\PYZus{}Mm}\PY{o}{*}\PY{o}{*}\PY{l+m+mi}{2}\PY{o}{*}\PY{n}{DMm}\PY{o}{*}\PY{o}{*}\PY{l+m+mi}{2} \PY{o}{+} \PY{n}{Dcm\PYZus{}Tm}\PY{o}{*}\PY{o}{*}\PY{l+m+mi}{2}\PY{o}{*}\PY{n}{DTm}\PY{o}{*}\PY{o}{*}\PY{l+m+mi}{2}\PY{p}{)}
\PY{n+nb}{print}\PY{p}{(}\PY{l+s+s2}{\PYZdq{}}\PY{l+s+s2}{Calore specifico }\PY{l+s+si}{\PYZpc{}.2f}\PY{l+s+s2}{ }\PY{l+s+si}{\PYZpc{}.2f}\PY{l+s+s2}{ }\PY{l+s+si}{\PYZpc{}.2f}\PY{l+s+s2}{\PYZdq{}} \PY{o}{\PYZpc{}}\PY{p}{(}\PY{n}{cm}\PY{p}{[}\PY{l+m+mi}{0}\PY{p}{]}\PY{p}{,} \PY{n}{cm}\PY{p}{[}\PY{l+m+mi}{1}\PY{p}{]}\PY{p}{,} \PY{n}{cm}\PY{p}{[}\PY{l+m+mi}{2}\PY{p}{]}\PY{p}{)}\PY{p}{)}

\PY{n+nb}{print}\PY{p}{(}\PY{l+s+s2}{\PYZdq{}}\PY{l+s+s2}{Incertezze }\PY{l+s+si}{\PYZpc{}.2f}\PY{l+s+s2}{ }\PY{l+s+si}{\PYZpc{}.2f}\PY{l+s+s2}{ }\PY{l+s+si}{\PYZpc{}.2f}\PY{l+s+s2}{\PYZdq{}} \PY{o}{\PYZpc{}}\PY{p}{(}\PY{n}{Dcm}\PY{p}{[}\PY{l+m+mi}{0}\PY{p}{]}\PY{p}{,} \PY{n}{Dcm}\PY{p}{[}\PY{l+m+mi}{1}\PY{p}{]}\PY{p}{,} \PY{n}{Dcm}\PY{p}{[}\PY{l+m+mi}{2}\PY{p}{]}\PY{p}{)}\PY{p}{)}
\end{Verbatim}
\end{tcolorbox}

    \begin{Verbatim}[commandchars=\\\{\}]
Calore specifico 0.32 0.52 0.16
Incertezze 0.08 0.11 0.05
    \end{Verbatim}

    \hypertarget{conclusioni}{%
\subsubsection{Conclusioni}\label{conclusioni}}

\ldots{}

    \hypertarget{calcolo-del-calore-latente-di-fusione-del-ghiaccio}{%
\subsection{1.3 Calcolo del calore latente di fusione del
ghiaccio}\label{calcolo-del-calore-latente-di-fusione-del-ghiaccio}}

\hypertarget{relazione-di-base}{%
\subsubsection{Relazione di base}\label{relazione-di-base}}

\ldots{}.

\hypertarget{procedimento-di-misura}{%
\subsubsection{Procedimento di misura}\label{procedimento-di-misura}}

\ldots{}.

\hypertarget{dati}{%
\subsubsection{Dati}\label{dati}}

\ldots{}.

    \begin{tcolorbox}[breakable, size=fbox, boxrule=1pt, pad at break*=1mm,colback=cellbackground, colframe=cellborder]
\prompt{In}{incolor}{24}{\boxspacing}
\begin{Verbatim}[commandchars=\\\{\}]
\PY{n}{Meq} \PY{o}{=} \PY{l+m+mi}{25} 
\PY{n}{Ma} \PY{o}{=} \PY{l+m+mf}{200.5}
\PY{n}{DMa} \PY{o}{=} \PY{l+m+mf}{0.1}\PY{o}{/}\PY{n}{np}\PY{o}{.}\PY{n}{sqrt}\PY{p}{(}\PY{l+m+mi}{12}\PY{p}{)}
\PY{n}{Mg} \PY{o}{=} \PY{l+m+mf}{70.7}
\PY{n}{DMg} \PY{o}{=} \PY{l+m+mf}{0.1}\PY{o}{/}\PY{n}{np}\PY{o}{.}\PY{n}{sqrt}\PY{p}{(}\PY{l+m+mi}{12}\PY{p}{)}
\PY{n}{Ta} \PY{o}{=} \PY{l+m+mf}{40.}
\PY{n}{DTa} \PY{o}{=} \PY{l+m+mf}{0.2}\PY{o}{/}\PY{n}{np}\PY{o}{.}\PY{n}{sqrt}\PY{p}{(}\PY{l+m+mi}{12}\PY{p}{)}
\PY{n}{Teq}\PY{o}{=} \PY{l+m+mf}{13.2}
\PY{n}{DTeq} \PY{o}{=} \PY{l+m+mf}{0.2}\PY{o}{/}\PY{n}{np}\PY{o}{.}\PY{n}{sqrt}\PY{p}{(}\PY{l+m+mi}{12}\PY{p}{)}
\PY{n}{Tf}\PY{o}{=}\PY{l+m+mf}{0.}
\end{Verbatim}
\end{tcolorbox}

    \hypertarget{calcolo}{%
\paragraph{Calcolo}\label{calcolo}}

    \begin{tcolorbox}[breakable, size=fbox, boxrule=1pt, pad at break*=1mm,colback=cellbackground, colframe=cellborder]
\prompt{In}{incolor}{25}{\boxspacing}
\begin{Verbatim}[commandchars=\\\{\}]
\PY{n}{L} \PY{o}{=} \PY{p}{(}\PY{p}{(}\PY{n}{Meq}\PY{o}{+}\PY{n}{Ma}\PY{p}{)}\PY{o}{*}\PY{n}{ca}\PY{o}{*}\PY{p}{(}\PY{n}{Ta}\PY{o}{\PYZhy{}}\PY{n}{Teq}\PY{p}{)}\PY{o}{\PYZhy{}}\PY{n}{ca}\PY{o}{*}\PY{n}{Mg}\PY{o}{*}\PY{p}{(}\PY{n}{Teq}\PY{o}{\PYZhy{}}\PY{n}{Tf}\PY{p}{)}\PY{p}{)}\PY{o}{/}\PY{n}{Mg}
\PY{n}{DL\PYZus{}Meq} \PY{o}{=} \PY{n}{ca}\PY{o}{*}\PY{p}{(}\PY{n}{Ta}\PY{o}{\PYZhy{}}\PY{n}{Teq}\PY{p}{)}\PY{o}{/}\PY{n}{Mg}
\PY{n}{DL\PYZus{}Ma} \PY{o}{=} \PY{n}{ca}\PY{o}{*}\PY{p}{(}\PY{n}{Ta}\PY{o}{\PYZhy{}}\PY{n}{Teq}\PY{p}{)}\PY{o}{/}\PY{n}{Mg}
\PY{n}{DL\PYZus{}Ta} \PY{o}{=} \PY{n}{ca}\PY{o}{*}\PY{p}{(}\PY{n}{Meq}\PY{o}{+}\PY{n}{Ma}\PY{p}{)}\PY{o}{/}\PY{n}{Mg}
\PY{n}{DL\PYZus{}Teq} \PY{o}{=} \PY{o}{\PYZhy{}} \PY{n}{ca}\PY{o}{*}\PY{p}{(}\PY{n}{Meq} \PY{o}{+} \PY{n}{Ma} \PY{o}{+} \PY{n}{Mg}\PY{p}{)}\PY{o}{/}\PY{n}{Mg}
\PY{n}{DL\PYZus{}Mg} \PY{o}{=} \PY{o}{\PYZhy{}}\PY{p}{(}\PY{p}{(}\PY{n}{Meq}\PY{o}{+}\PY{n}{Ma}\PY{p}{)}\PY{o}{*}\PY{n}{ca}\PY{o}{*}\PY{p}{(}\PY{n}{Ta}\PY{o}{\PYZhy{}}\PY{n}{Teq}\PY{p}{)}\PY{p}{)}\PY{o}{/}\PY{n}{Mg}\PY{o}{*}\PY{o}{*}\PY{l+m+mi}{2}
\PY{n}{DL} \PY{o}{=} \PY{n}{np}\PY{o}{.}\PY{n}{sqrt}\PY{p}{(}\PY{n}{DL\PYZus{}Meq}\PY{o}{*}\PY{o}{*}\PY{l+m+mi}{2}\PY{o}{*}\PY{n}{DMeq}\PY{o}{*}\PY{o}{*}\PY{l+m+mi}{2}\PY{o}{+}\PY{n}{DL\PYZus{}Ma}\PY{o}{*}\PY{o}{*}\PY{l+m+mi}{2}\PY{o}{*}\PY{n}{DMa}\PY{o}{*}\PY{o}{*}\PY{l+m+mi}{2}\PY{o}{+}\PY{n}{DL\PYZus{}Ta}\PY{o}{*}\PY{o}{*}\PY{l+m+mi}{2}\PY{o}{*}
     \PY{n}{DTa}\PY{o}{*}\PY{o}{*}\PY{l+m+mi}{2}\PY{o}{+}\PY{n}{DL\PYZus{}Teq}\PY{o}{*}\PY{o}{*}\PY{l+m+mi}{2}\PY{o}{*}\PY{n}{DTeq}\PY{o}{*}\PY{o}{*}\PY{l+m+mi}{2}\PY{o}{+}\PY{n}{DL\PYZus{}Mg}\PY{o}{*}\PY{o}{*}\PY{l+m+mi}{2}\PY{o}{*}\PY{n}{DMg}\PY{o}{*}\PY{o}{*}\PY{l+m+mi}{2}\PY{p}{)}
\PY{n+nb}{print}\PY{p}{(}\PY{l+s+s2}{\PYZdq{}}\PY{l+s+s2}{Lambda = }\PY{l+s+si}{\PYZpc{}.1f}\PY{l+s+s2}{ cal/g}\PY{l+s+s2}{\PYZdq{}} \PY{o}{\PYZpc{}}\PY{p}{(}\PY{n}{L}\PY{p}{)}\PY{p}{)}
\PY{n+nb}{print}\PY{p}{(}\PY{l+s+s2}{\PYZdq{}}\PY{l+s+s2}{Incertezza = }\PY{l+s+si}{\PYZpc{}.1f}\PY{l+s+s2}{ cal/g}\PY{l+s+s2}{\PYZdq{}} \PY{o}{\PYZpc{}}\PY{p}{(}\PY{n}{DL}\PY{p}{)}\PY{p}{)}
\end{Verbatim}
\end{tcolorbox}

    \begin{Verbatim}[commandchars=\\\{\}]
Lambda = 72.3 cal/g
Incertezza = 1.9 cal/g
    \end{Verbatim}

    \hypertarget{concluzione}{%
\subsubsection{Concluzione}\label{concluzione}}

\ldots{}..

    \hypertarget{verifica-dellequivalente-in-acqua-del-calorimetro}{%
\subsection{1.4 Verifica dell'equivalente in acqua del
calorimetro}\label{verifica-dellequivalente-in-acqua-del-calorimetro}}

\hypertarget{relazione-di-base}{%
\subsubsection{Relazione di base}\label{relazione-di-base}}

\ldots{}.

\hypertarget{procedimento-di-misura}{%
\subsubsection{Procedimento di misura}\label{procedimento-di-misura}}

\ldots{}.

\hypertarget{dati}{%
\subsubsection{Dati}\label{dati}}

\ldots{}.

    \begin{tcolorbox}[breakable, size=fbox, boxrule=1pt, pad at break*=1mm,colback=cellbackground, colframe=cellborder]
\prompt{In}{incolor}{26}{\boxspacing}
\begin{Verbatim}[commandchars=\\\{\}]
\PY{n}{Ma1} \PY{o}{=} \PY{l+m+mf}{201.2}
\PY{n}{DMa1}\PY{o}{=} \PY{l+m+mf}{0.1}\PY{o}{/}\PY{n}{np}\PY{o}{.}\PY{n}{sqrt}\PY{p}{(}\PY{l+m+mi}{12}\PY{p}{)}
\PY{n}{Ma2} \PY{o}{=} \PY{l+m+mf}{31.2}
\PY{n}{DMa2}\PY{o}{=} \PY{l+m+mf}{0.1}\PY{o}{/}\PY{n}{np}\PY{o}{.}\PY{n}{sqrt}\PY{p}{(}\PY{l+m+mi}{12}\PY{p}{)}
\PY{n}{Ta1} \PY{o}{=} \PY{l+m+mf}{40.4}
\PY{n}{DTa1}\PY{o}{=} \PY{l+m+mf}{0.2}\PY{o}{/}\PY{n}{np}\PY{o}{.}\PY{n}{sqrt}\PY{p}{(}\PY{l+m+mi}{12}\PY{p}{)}
\PY{n}{Ta2} \PY{o}{=} \PY{l+m+mf}{23.6}
\PY{n}{DTa2}\PY{o}{=} \PY{l+m+mf}{0.2}\PY{o}{/}\PY{n}{np}\PY{o}{.}\PY{n}{sqrt}\PY{p}{(}\PY{l+m+mi}{12}\PY{p}{)}
\PY{n}{Teq} \PY{o}{=} \PY{l+m+mf}{37.6}
\PY{n}{DTeq}\PY{o}{=} \PY{l+m+mf}{0.2}\PY{o}{/}\PY{n}{np}\PY{o}{.}\PY{n}{sqrt}\PY{p}{(}\PY{l+m+mi}{12}\PY{p}{)}
\end{Verbatim}
\end{tcolorbox}

    \hypertarget{calcoli}{%
\paragraph{Calcoli}\label{calcoli}}

    \begin{tcolorbox}[breakable, size=fbox, boxrule=1pt, pad at break*=1mm,colback=cellbackground, colframe=cellborder]
\prompt{In}{incolor}{27}{\boxspacing}
\begin{Verbatim}[commandchars=\\\{\}]
\PY{n}{Meq} \PY{o}{=} \PY{n}{Ma2}\PY{o}{*}\PY{p}{(}\PY{n}{Teq} \PY{o}{\PYZhy{}} \PY{n}{Ta2}\PY{p}{)}\PY{o}{/}\PY{p}{(}\PY{n}{Ta1} \PY{o}{\PYZhy{}} \PY{n}{Teq}\PY{p}{)} \PY{o}{\PYZhy{}} \PY{n}{Ma1}
\PY{n}{DM\PYZus{}Ma1} \PY{o}{=} \PY{o}{\PYZhy{}}\PY{l+m+mi}{1}
\PY{n}{DM\PYZus{}Ma2} \PY{o}{=} \PY{p}{(}\PY{n}{Teq} \PY{o}{\PYZhy{}} \PY{n}{Ta2}\PY{p}{)}\PY{o}{/}\PY{p}{(}\PY{n}{Ta1} \PY{o}{\PYZhy{}} \PY{n}{Teq}\PY{p}{)}
\PY{n}{DM\PYZus{}Ta1} \PY{o}{=} \PY{o}{\PYZhy{}}\PY{n}{Ma2}\PY{o}{*}\PY{p}{(}\PY{n}{Teq} \PY{o}{\PYZhy{}} \PY{n}{Ta2}\PY{p}{)}\PY{o}{/}\PY{p}{(}\PY{n}{Ta1} \PY{o}{\PYZhy{}} \PY{n}{Teq}\PY{p}{)}\PY{o}{*}\PY{o}{*}\PY{l+m+mi}{2}
\PY{n}{DM\PYZus{}Ta2} \PY{o}{=} \PY{o}{\PYZhy{}}\PY{n}{Ma2}\PY{o}{/}\PY{p}{(}\PY{n}{Ta1} \PY{o}{\PYZhy{}} \PY{n}{Teq}\PY{p}{)}
\PY{n}{DM\PYZus{}Teq} \PY{o}{=}  \PY{n}{Ma2}\PY{o}{/}\PY{p}{(}\PY{n}{Ta1} \PY{o}{\PYZhy{}} \PY{n}{Teq}\PY{p}{)}
\PY{n}{DM} \PY{o}{=} \PY{n}{np}\PY{o}{.}\PY{n}{sqrt}\PY{p}{(}\PY{n}{DM\PYZus{}Teq}\PY{o}{*}\PY{o}{*}\PY{l+m+mi}{2}\PY{o}{*}\PY{n}{DTeq}\PY{o}{*}\PY{o}{*}\PY{l+m+mi}{2}\PY{o}{+}\PY{n}{DM\PYZus{}Ma1}\PY{o}{*}\PY{o}{*}\PY{l+m+mi}{2}\PY{o}{*}\PY{n}{DMa1}\PY{o}{*}\PY{o}{*}\PY{l+m+mi}{2}\PY{o}{+}\PY{n}{DM\PYZus{}Ta1}\PY{o}{*}\PY{o}{*}\PY{l+m+mi}{2}\PY{o}{*}\PY{n}{DTa1}\PY{o}{*}\PY{o}{*}\PY{l+m+mi}{2}\PY{o}{+}\PY{n}{DM\PYZus{}Ta2}\PY{o}{*}\PY{o}{*}\PY{l+m+mi}{2}\PY{o}{*}
     \PY{n}{DTa2}\PY{o}{*}\PY{o}{*}\PY{l+m+mi}{2}\PY{o}{+}\PY{n}{DM\PYZus{}Ma2}\PY{o}{*}\PY{o}{*}\PY{l+m+mi}{2}\PY{o}{*}\PY{n}{DMa2}\PY{o}{*}\PY{o}{*}\PY{l+m+mi}{2}\PY{p}{)}
\PY{n+nb}{print}\PY{p}{(}\PY{n}{Meq}\PY{p}{)}
\PY{n+nb}{print}\PY{p}{(}\PY{n}{DM}\PY{p}{)}
\end{Verbatim}
\end{tcolorbox}

    \begin{Verbatim}[commandchars=\\\{\}]
-45.19999999999982
3.346096314247056
    \end{Verbatim}


    % Add a bibliography block to the postdoc
    
    
    
\end{document}
