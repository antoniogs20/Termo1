\documentclass[11pt]{article}

    \usepackage[breakable]{tcolorbox}
    \usepackage{parskip} % Stop auto-indenting (to mimic markdown behaviour)
    
    \usepackage{iftex}
    \ifPDFTeX
    	\usepackage[T1]{fontenc}
    	\usepackage{mathpazo}
    \else
    	\usepackage{fontspec}
    \fi

    % Basic figure setup, for now with no caption control since it's done
    % automatically by Pandoc (which extracts ![](path) syntax from Markdown).
    \usepackage{graphicx}
    % Maintain compatibility with old templates. Remove in nbconvert 6.0
    \let\Oldincludegraphics\includegraphics
    % Ensure that by default, figures have no caption (until we provide a
    % proper Figure object with a Caption API and a way to capture that
    % in the conversion process - todo).
    \usepackage{caption}
    \DeclareCaptionFormat{nocaption}{}
    \captionsetup{format=nocaption,aboveskip=0pt,belowskip=0pt}

    \usepackage[Export]{adjustbox} % Used to constrain images to a maximum size
    \adjustboxset{max size={0.9\linewidth}{0.9\paperheight}}
    \usepackage{float}
    \floatplacement{figure}{H} % forces figures to be placed at the correct location
    \usepackage{xcolor} % Allow colors to be defined
    \usepackage{enumerate} % Needed for markdown enumerations to work
    \usepackage{geometry} % Used to adjust the document margins
    \usepackage{amsmath} % Equations
    \usepackage{amssymb} % Equations
    \usepackage{textcomp} % defines textquotesingle
    \usepackage{hyperref}
    % Hack from http://tex.stackexchange.com/a/47451/13684:
    \AtBeginDocument{%
        \def\PYZsq{\textquotesingle}% Upright quotes in Pygmentized code
    }
    \usepackage{upquote} % Upright quotes for verbatim code
    \usepackage{eurosym} % defines \euro
    \usepackage[mathletters]{ucs} % Extended unicode (utf-8) support
    \usepackage{fancyvrb} % verbatim replacement that allows latex
    \usepackage{grffile} % extends the file name processing of package graphics 
                         % to support a larger range
    \makeatletter % fix for grffile with XeLaTeX
    \def\Gread@@xetex#1{%
      \IfFileExists{"\Gin@base".bb}%
      {\Gread@eps{\Gin@base.bb}}%
      {\Gread@@xetex@aux#1}%
    }
    \makeatother

    % The hyperref package gives us a pdf with properly built
    % internal navigation ('pdf bookmarks' for the table of contents,
    % internal cross-reference links, web links for URLs, etc.)
    \usepackage{hyperref}
    % The default LaTeX title has an obnoxious amount of whitespace. By default,
    % titling removes some of it. It also provides customization options.
    \usepackage{titling}
    \usepackage{longtable} % longtable support required by pandoc >1.10
    \usepackage{booktabs}  % table support for pandoc > 1.12.2
    \usepackage[inline]{enumitem} % IRkernel/repr support (it uses the enumerate* environment)
    \usepackage[normalem]{ulem} % ulem is needed to support strikethroughs (\sout)
                                % normalem makes italics be italics, not underlines
    \usepackage{mathrsfs}
    

    
    % Colors for the hyperref package
    \definecolor{urlcolor}{rgb}{0,.145,.698}
    \definecolor{linkcolor}{rgb}{.71,0.21,0.01}
    \definecolor{citecolor}{rgb}{.12,.54,.11}

    % ANSI colors
    \definecolor{ansi-black}{HTML}{3E424D}
    \definecolor{ansi-black-intense}{HTML}{282C36}
    \definecolor{ansi-red}{HTML}{E75C58}
    \definecolor{ansi-red-intense}{HTML}{B22B31}
    \definecolor{ansi-green}{HTML}{00A250}
    \definecolor{ansi-green-intense}{HTML}{007427}
    \definecolor{ansi-yellow}{HTML}{DDB62B}
    \definecolor{ansi-yellow-intense}{HTML}{B27D12}
    \definecolor{ansi-blue}{HTML}{208FFB}
    \definecolor{ansi-blue-intense}{HTML}{0065CA}
    \definecolor{ansi-magenta}{HTML}{D160C4}
    \definecolor{ansi-magenta-intense}{HTML}{A03196}
    \definecolor{ansi-cyan}{HTML}{60C6C8}
    \definecolor{ansi-cyan-intense}{HTML}{258F8F}
    \definecolor{ansi-white}{HTML}{C5C1B4}
    \definecolor{ansi-white-intense}{HTML}{A1A6B2}
    \definecolor{ansi-default-inverse-fg}{HTML}{FFFFFF}
    \definecolor{ansi-default-inverse-bg}{HTML}{000000}

    % commands and environments needed by pandoc snippets
    % extracted from the output of `pandoc -s`
    \providecommand{\tightlist}{%
      \setlength{\itemsep}{0pt}\setlength{\parskip}{0pt}}
    \DefineVerbatimEnvironment{Highlighting}{Verbatim}{commandchars=\\\{\}}
    % Add ',fontsize=\small' for more characters per line
    \newenvironment{Shaded}{}{}
    \newcommand{\KeywordTok}[1]{\textcolor[rgb]{0.00,0.44,0.13}{\textbf{{#1}}}}
    \newcommand{\DataTypeTok}[1]{\textcolor[rgb]{0.56,0.13,0.00}{{#1}}}
    \newcommand{\DecValTok}[1]{\textcolor[rgb]{0.25,0.63,0.44}{{#1}}}
    \newcommand{\BaseNTok}[1]{\textcolor[rgb]{0.25,0.63,0.44}{{#1}}}
    \newcommand{\FloatTok}[1]{\textcolor[rgb]{0.25,0.63,0.44}{{#1}}}
    \newcommand{\CharTok}[1]{\textcolor[rgb]{0.25,0.44,0.63}{{#1}}}
    \newcommand{\StringTok}[1]{\textcolor[rgb]{0.25,0.44,0.63}{{#1}}}
    \newcommand{\CommentTok}[1]{\textcolor[rgb]{0.38,0.63,0.69}{\textit{{#1}}}}
    \newcommand{\OtherTok}[1]{\textcolor[rgb]{0.00,0.44,0.13}{{#1}}}
    \newcommand{\AlertTok}[1]{\textcolor[rgb]{1.00,0.00,0.00}{\textbf{{#1}}}}
    \newcommand{\FunctionTok}[1]{\textcolor[rgb]{0.02,0.16,0.49}{{#1}}}
    \newcommand{\RegionMarkerTok}[1]{{#1}}
    \newcommand{\ErrorTok}[1]{\textcolor[rgb]{1.00,0.00,0.00}{\textbf{{#1}}}}
    \newcommand{\NormalTok}[1]{{#1}}
    
    % Additional commands for more recent versions of Pandoc
    \newcommand{\ConstantTok}[1]{\textcolor[rgb]{0.53,0.00,0.00}{{#1}}}
    \newcommand{\SpecialCharTok}[1]{\textcolor[rgb]{0.25,0.44,0.63}{{#1}}}
    \newcommand{\VerbatimStringTok}[1]{\textcolor[rgb]{0.25,0.44,0.63}{{#1}}}
    \newcommand{\SpecialStringTok}[1]{\textcolor[rgb]{0.73,0.40,0.53}{{#1}}}
    \newcommand{\ImportTok}[1]{{#1}}
    \newcommand{\DocumentationTok}[1]{\textcolor[rgb]{0.73,0.13,0.13}{\textit{{#1}}}}
    \newcommand{\AnnotationTok}[1]{\textcolor[rgb]{0.38,0.63,0.69}{\textbf{\textit{{#1}}}}}
    \newcommand{\CommentVarTok}[1]{\textcolor[rgb]{0.38,0.63,0.69}{\textbf{\textit{{#1}}}}}
    \newcommand{\VariableTok}[1]{\textcolor[rgb]{0.10,0.09,0.49}{{#1}}}
    \newcommand{\ControlFlowTok}[1]{\textcolor[rgb]{0.00,0.44,0.13}{\textbf{{#1}}}}
    \newcommand{\OperatorTok}[1]{\textcolor[rgb]{0.40,0.40,0.40}{{#1}}}
    \newcommand{\BuiltInTok}[1]{{#1}}
    \newcommand{\ExtensionTok}[1]{{#1}}
    \newcommand{\PreprocessorTok}[1]{\textcolor[rgb]{0.74,0.48,0.00}{{#1}}}
    \newcommand{\AttributeTok}[1]{\textcolor[rgb]{0.49,0.56,0.16}{{#1}}}
    \newcommand{\InformationTok}[1]{\textcolor[rgb]{0.38,0.63,0.69}{\textbf{\textit{{#1}}}}}
    \newcommand{\WarningTok}[1]{\textcolor[rgb]{0.38,0.63,0.69}{\textbf{\textit{{#1}}}}}
    
    
    % Define a nice break command that doesn't care if a line doesn't already
    % exist.
    \def\br{\hspace*{\fill} \\* }
    % Math Jax compatibility definitions
    \def\gt{>}
    \def\lt{<}
    \let\Oldtex\TeX
    \let\Oldlatex\LaTeX
    \renewcommand{\TeX}{\textrm{\Oldtex}}
    \renewcommand{\LaTeX}{\textrm{\Oldlatex}}
    % Document parameters
    % Document title
    \title{esp1\_ivan}
    
    
    
    
    
% Pygments definitions
\makeatletter
\def\PY@reset{\let\PY@it=\relax \let\PY@bf=\relax%
    \let\PY@ul=\relax \let\PY@tc=\relax%
    \let\PY@bc=\relax \let\PY@ff=\relax}
\def\PY@tok#1{\csname PY@tok@#1\endcsname}
\def\PY@toks#1+{\ifx\relax#1\empty\else%
    \PY@tok{#1}\expandafter\PY@toks\fi}
\def\PY@do#1{\PY@bc{\PY@tc{\PY@ul{%
    \PY@it{\PY@bf{\PY@ff{#1}}}}}}}
\def\PY#1#2{\PY@reset\PY@toks#1+\relax+\PY@do{#2}}

\expandafter\def\csname PY@tok@w\endcsname{\def\PY@tc##1{\textcolor[rgb]{0.73,0.73,0.73}{##1}}}
\expandafter\def\csname PY@tok@c\endcsname{\let\PY@it=\textit\def\PY@tc##1{\textcolor[rgb]{0.25,0.50,0.50}{##1}}}
\expandafter\def\csname PY@tok@cp\endcsname{\def\PY@tc##1{\textcolor[rgb]{0.74,0.48,0.00}{##1}}}
\expandafter\def\csname PY@tok@k\endcsname{\let\PY@bf=\textbf\def\PY@tc##1{\textcolor[rgb]{0.00,0.50,0.00}{##1}}}
\expandafter\def\csname PY@tok@kp\endcsname{\def\PY@tc##1{\textcolor[rgb]{0.00,0.50,0.00}{##1}}}
\expandafter\def\csname PY@tok@kt\endcsname{\def\PY@tc##1{\textcolor[rgb]{0.69,0.00,0.25}{##1}}}
\expandafter\def\csname PY@tok@o\endcsname{\def\PY@tc##1{\textcolor[rgb]{0.40,0.40,0.40}{##1}}}
\expandafter\def\csname PY@tok@ow\endcsname{\let\PY@bf=\textbf\def\PY@tc##1{\textcolor[rgb]{0.67,0.13,1.00}{##1}}}
\expandafter\def\csname PY@tok@nb\endcsname{\def\PY@tc##1{\textcolor[rgb]{0.00,0.50,0.00}{##1}}}
\expandafter\def\csname PY@tok@nf\endcsname{\def\PY@tc##1{\textcolor[rgb]{0.00,0.00,1.00}{##1}}}
\expandafter\def\csname PY@tok@nc\endcsname{\let\PY@bf=\textbf\def\PY@tc##1{\textcolor[rgb]{0.00,0.00,1.00}{##1}}}
\expandafter\def\csname PY@tok@nn\endcsname{\let\PY@bf=\textbf\def\PY@tc##1{\textcolor[rgb]{0.00,0.00,1.00}{##1}}}
\expandafter\def\csname PY@tok@ne\endcsname{\let\PY@bf=\textbf\def\PY@tc##1{\textcolor[rgb]{0.82,0.25,0.23}{##1}}}
\expandafter\def\csname PY@tok@nv\endcsname{\def\PY@tc##1{\textcolor[rgb]{0.10,0.09,0.49}{##1}}}
\expandafter\def\csname PY@tok@no\endcsname{\def\PY@tc##1{\textcolor[rgb]{0.53,0.00,0.00}{##1}}}
\expandafter\def\csname PY@tok@nl\endcsname{\def\PY@tc##1{\textcolor[rgb]{0.63,0.63,0.00}{##1}}}
\expandafter\def\csname PY@tok@ni\endcsname{\let\PY@bf=\textbf\def\PY@tc##1{\textcolor[rgb]{0.60,0.60,0.60}{##1}}}
\expandafter\def\csname PY@tok@na\endcsname{\def\PY@tc##1{\textcolor[rgb]{0.49,0.56,0.16}{##1}}}
\expandafter\def\csname PY@tok@nt\endcsname{\let\PY@bf=\textbf\def\PY@tc##1{\textcolor[rgb]{0.00,0.50,0.00}{##1}}}
\expandafter\def\csname PY@tok@nd\endcsname{\def\PY@tc##1{\textcolor[rgb]{0.67,0.13,1.00}{##1}}}
\expandafter\def\csname PY@tok@s\endcsname{\def\PY@tc##1{\textcolor[rgb]{0.73,0.13,0.13}{##1}}}
\expandafter\def\csname PY@tok@sd\endcsname{\let\PY@it=\textit\def\PY@tc##1{\textcolor[rgb]{0.73,0.13,0.13}{##1}}}
\expandafter\def\csname PY@tok@si\endcsname{\let\PY@bf=\textbf\def\PY@tc##1{\textcolor[rgb]{0.73,0.40,0.53}{##1}}}
\expandafter\def\csname PY@tok@se\endcsname{\let\PY@bf=\textbf\def\PY@tc##1{\textcolor[rgb]{0.73,0.40,0.13}{##1}}}
\expandafter\def\csname PY@tok@sr\endcsname{\def\PY@tc##1{\textcolor[rgb]{0.73,0.40,0.53}{##1}}}
\expandafter\def\csname PY@tok@ss\endcsname{\def\PY@tc##1{\textcolor[rgb]{0.10,0.09,0.49}{##1}}}
\expandafter\def\csname PY@tok@sx\endcsname{\def\PY@tc##1{\textcolor[rgb]{0.00,0.50,0.00}{##1}}}
\expandafter\def\csname PY@tok@m\endcsname{\def\PY@tc##1{\textcolor[rgb]{0.40,0.40,0.40}{##1}}}
\expandafter\def\csname PY@tok@gh\endcsname{\let\PY@bf=\textbf\def\PY@tc##1{\textcolor[rgb]{0.00,0.00,0.50}{##1}}}
\expandafter\def\csname PY@tok@gu\endcsname{\let\PY@bf=\textbf\def\PY@tc##1{\textcolor[rgb]{0.50,0.00,0.50}{##1}}}
\expandafter\def\csname PY@tok@gd\endcsname{\def\PY@tc##1{\textcolor[rgb]{0.63,0.00,0.00}{##1}}}
\expandafter\def\csname PY@tok@gi\endcsname{\def\PY@tc##1{\textcolor[rgb]{0.00,0.63,0.00}{##1}}}
\expandafter\def\csname PY@tok@gr\endcsname{\def\PY@tc##1{\textcolor[rgb]{1.00,0.00,0.00}{##1}}}
\expandafter\def\csname PY@tok@ge\endcsname{\let\PY@it=\textit}
\expandafter\def\csname PY@tok@gs\endcsname{\let\PY@bf=\textbf}
\expandafter\def\csname PY@tok@gp\endcsname{\let\PY@bf=\textbf\def\PY@tc##1{\textcolor[rgb]{0.00,0.00,0.50}{##1}}}
\expandafter\def\csname PY@tok@go\endcsname{\def\PY@tc##1{\textcolor[rgb]{0.53,0.53,0.53}{##1}}}
\expandafter\def\csname PY@tok@gt\endcsname{\def\PY@tc##1{\textcolor[rgb]{0.00,0.27,0.87}{##1}}}
\expandafter\def\csname PY@tok@err\endcsname{\def\PY@bc##1{\setlength{\fboxsep}{0pt}\fcolorbox[rgb]{1.00,0.00,0.00}{1,1,1}{\strut ##1}}}
\expandafter\def\csname PY@tok@kc\endcsname{\let\PY@bf=\textbf\def\PY@tc##1{\textcolor[rgb]{0.00,0.50,0.00}{##1}}}
\expandafter\def\csname PY@tok@kd\endcsname{\let\PY@bf=\textbf\def\PY@tc##1{\textcolor[rgb]{0.00,0.50,0.00}{##1}}}
\expandafter\def\csname PY@tok@kn\endcsname{\let\PY@bf=\textbf\def\PY@tc##1{\textcolor[rgb]{0.00,0.50,0.00}{##1}}}
\expandafter\def\csname PY@tok@kr\endcsname{\let\PY@bf=\textbf\def\PY@tc##1{\textcolor[rgb]{0.00,0.50,0.00}{##1}}}
\expandafter\def\csname PY@tok@bp\endcsname{\def\PY@tc##1{\textcolor[rgb]{0.00,0.50,0.00}{##1}}}
\expandafter\def\csname PY@tok@fm\endcsname{\def\PY@tc##1{\textcolor[rgb]{0.00,0.00,1.00}{##1}}}
\expandafter\def\csname PY@tok@vc\endcsname{\def\PY@tc##1{\textcolor[rgb]{0.10,0.09,0.49}{##1}}}
\expandafter\def\csname PY@tok@vg\endcsname{\def\PY@tc##1{\textcolor[rgb]{0.10,0.09,0.49}{##1}}}
\expandafter\def\csname PY@tok@vi\endcsname{\def\PY@tc##1{\textcolor[rgb]{0.10,0.09,0.49}{##1}}}
\expandafter\def\csname PY@tok@vm\endcsname{\def\PY@tc##1{\textcolor[rgb]{0.10,0.09,0.49}{##1}}}
\expandafter\def\csname PY@tok@sa\endcsname{\def\PY@tc##1{\textcolor[rgb]{0.73,0.13,0.13}{##1}}}
\expandafter\def\csname PY@tok@sb\endcsname{\def\PY@tc##1{\textcolor[rgb]{0.73,0.13,0.13}{##1}}}
\expandafter\def\csname PY@tok@sc\endcsname{\def\PY@tc##1{\textcolor[rgb]{0.73,0.13,0.13}{##1}}}
\expandafter\def\csname PY@tok@dl\endcsname{\def\PY@tc##1{\textcolor[rgb]{0.73,0.13,0.13}{##1}}}
\expandafter\def\csname PY@tok@s2\endcsname{\def\PY@tc##1{\textcolor[rgb]{0.73,0.13,0.13}{##1}}}
\expandafter\def\csname PY@tok@sh\endcsname{\def\PY@tc##1{\textcolor[rgb]{0.73,0.13,0.13}{##1}}}
\expandafter\def\csname PY@tok@s1\endcsname{\def\PY@tc##1{\textcolor[rgb]{0.73,0.13,0.13}{##1}}}
\expandafter\def\csname PY@tok@mb\endcsname{\def\PY@tc##1{\textcolor[rgb]{0.40,0.40,0.40}{##1}}}
\expandafter\def\csname PY@tok@mf\endcsname{\def\PY@tc##1{\textcolor[rgb]{0.40,0.40,0.40}{##1}}}
\expandafter\def\csname PY@tok@mh\endcsname{\def\PY@tc##1{\textcolor[rgb]{0.40,0.40,0.40}{##1}}}
\expandafter\def\csname PY@tok@mi\endcsname{\def\PY@tc##1{\textcolor[rgb]{0.40,0.40,0.40}{##1}}}
\expandafter\def\csname PY@tok@il\endcsname{\def\PY@tc##1{\textcolor[rgb]{0.40,0.40,0.40}{##1}}}
\expandafter\def\csname PY@tok@mo\endcsname{\def\PY@tc##1{\textcolor[rgb]{0.40,0.40,0.40}{##1}}}
\expandafter\def\csname PY@tok@ch\endcsname{\let\PY@it=\textit\def\PY@tc##1{\textcolor[rgb]{0.25,0.50,0.50}{##1}}}
\expandafter\def\csname PY@tok@cm\endcsname{\let\PY@it=\textit\def\PY@tc##1{\textcolor[rgb]{0.25,0.50,0.50}{##1}}}
\expandafter\def\csname PY@tok@cpf\endcsname{\let\PY@it=\textit\def\PY@tc##1{\textcolor[rgb]{0.25,0.50,0.50}{##1}}}
\expandafter\def\csname PY@tok@c1\endcsname{\let\PY@it=\textit\def\PY@tc##1{\textcolor[rgb]{0.25,0.50,0.50}{##1}}}
\expandafter\def\csname PY@tok@cs\endcsname{\let\PY@it=\textit\def\PY@tc##1{\textcolor[rgb]{0.25,0.50,0.50}{##1}}}

\def\PYZbs{\char`\\}
\def\PYZus{\char`\_}
\def\PYZob{\char`\{}
\def\PYZcb{\char`\}}
\def\PYZca{\char`\^}
\def\PYZam{\char`\&}
\def\PYZlt{\char`\<}
\def\PYZgt{\char`\>}
\def\PYZsh{\char`\#}
\def\PYZpc{\char`\%}
\def\PYZdl{\char`\$}
\def\PYZhy{\char`\-}
\def\PYZsq{\char`\'}
\def\PYZdq{\char`\"}
\def\PYZti{\char`\~}
% for compatibility with earlier versions
\def\PYZat{@}
\def\PYZlb{[}
\def\PYZrb{]}
\makeatother


    % For linebreaks inside Verbatim environment from package fancyvrb. 
    \makeatletter
        \newbox\Wrappedcontinuationbox 
        \newbox\Wrappedvisiblespacebox 
        \newcommand*\Wrappedvisiblespace {\textcolor{red}{\textvisiblespace}} 
        \newcommand*\Wrappedcontinuationsymbol {\textcolor{red}{\llap{\tiny$\m@th\hookrightarrow$}}} 
        \newcommand*\Wrappedcontinuationindent {3ex } 
        \newcommand*\Wrappedafterbreak {\kern\Wrappedcontinuationindent\copy\Wrappedcontinuationbox} 
        % Take advantage of the already applied Pygments mark-up to insert 
        % potential linebreaks for TeX processing. 
        %        {, <, #, %, $, ' and ": go to next line. 
        %        _, }, ^, &, >, - and ~: stay at end of broken line. 
        % Use of \textquotesingle for straight quote. 
        \newcommand*\Wrappedbreaksatspecials {% 
            \def\PYGZus{\discretionary{\char`\_}{\Wrappedafterbreak}{\char`\_}}% 
            \def\PYGZob{\discretionary{}{\Wrappedafterbreak\char`\{}{\char`\{}}% 
            \def\PYGZcb{\discretionary{\char`\}}{\Wrappedafterbreak}{\char`\}}}% 
            \def\PYGZca{\discretionary{\char`\^}{\Wrappedafterbreak}{\char`\^}}% 
            \def\PYGZam{\discretionary{\char`\&}{\Wrappedafterbreak}{\char`\&}}% 
            \def\PYGZlt{\discretionary{}{\Wrappedafterbreak\char`\<}{\char`\<}}% 
            \def\PYGZgt{\discretionary{\char`\>}{\Wrappedafterbreak}{\char`\>}}% 
            \def\PYGZsh{\discretionary{}{\Wrappedafterbreak\char`\#}{\char`\#}}% 
            \def\PYGZpc{\discretionary{}{\Wrappedafterbreak\char`\%}{\char`\%}}% 
            \def\PYGZdl{\discretionary{}{\Wrappedafterbreak\char`\$}{\char`\$}}% 
            \def\PYGZhy{\discretionary{\char`\-}{\Wrappedafterbreak}{\char`\-}}% 
            \def\PYGZsq{\discretionary{}{\Wrappedafterbreak\textquotesingle}{\textquotesingle}}% 
            \def\PYGZdq{\discretionary{}{\Wrappedafterbreak\char`\"}{\char`\"}}% 
            \def\PYGZti{\discretionary{\char`\~}{\Wrappedafterbreak}{\char`\~}}% 
        } 
        % Some characters . , ; ? ! / are not pygmentized. 
        % This macro makes them "active" and they will insert potential linebreaks 
        \newcommand*\Wrappedbreaksatpunct {% 
            \lccode`\~`\.\lowercase{\def~}{\discretionary{\hbox{\char`\.}}{\Wrappedafterbreak}{\hbox{\char`\.}}}% 
            \lccode`\~`\,\lowercase{\def~}{\discretionary{\hbox{\char`\,}}{\Wrappedafterbreak}{\hbox{\char`\,}}}% 
            \lccode`\~`\;\lowercase{\def~}{\discretionary{\hbox{\char`\;}}{\Wrappedafterbreak}{\hbox{\char`\;}}}% 
            \lccode`\~`\:\lowercase{\def~}{\discretionary{\hbox{\char`\:}}{\Wrappedafterbreak}{\hbox{\char`\:}}}% 
            \lccode`\~`\?\lowercase{\def~}{\discretionary{\hbox{\char`\?}}{\Wrappedafterbreak}{\hbox{\char`\?}}}% 
            \lccode`\~`\!\lowercase{\def~}{\discretionary{\hbox{\char`\!}}{\Wrappedafterbreak}{\hbox{\char`\!}}}% 
            \lccode`\~`\/\lowercase{\def~}{\discretionary{\hbox{\char`\/}}{\Wrappedafterbreak}{\hbox{\char`\/}}}% 
            \catcode`\.\active
            \catcode`\,\active 
            \catcode`\;\active
            \catcode`\:\active
            \catcode`\?\active
            \catcode`\!\active
            \catcode`\/\active 
            \lccode`\~`\~ 	
        }
    \makeatother

    \let\OriginalVerbatim=\Verbatim
    \makeatletter
    \renewcommand{\Verbatim}[1][1]{%
        %\parskip\z@skip
        \sbox\Wrappedcontinuationbox {\Wrappedcontinuationsymbol}%
        \sbox\Wrappedvisiblespacebox {\FV@SetupFont\Wrappedvisiblespace}%
        \def\FancyVerbFormatLine ##1{\hsize\linewidth
            \vtop{\raggedright\hyphenpenalty\z@\exhyphenpenalty\z@
                \doublehyphendemerits\z@\finalhyphendemerits\z@
                \strut ##1\strut}%
        }%
        % If the linebreak is at a space, the latter will be displayed as visible
        % space at end of first line, and a continuation symbol starts next line.
        % Stretch/shrink are however usually zero for typewriter font.
        \def\FV@Space {%
            \nobreak\hskip\z@ plus\fontdimen3\font minus\fontdimen4\font
            \discretionary{\copy\Wrappedvisiblespacebox}{\Wrappedafterbreak}
            {\kern\fontdimen2\font}%
        }%
        
        % Allow breaks at special characters using \PYG... macros.
        \Wrappedbreaksatspecials
        % Breaks at punctuation characters . , ; ? ! and / need catcode=\active 	
        \OriginalVerbatim[#1,codes*=\Wrappedbreaksatpunct]%
    }
    \makeatother

    % Exact colors from NB
    \definecolor{incolor}{HTML}{303F9F}
    \definecolor{outcolor}{HTML}{D84315}
    \definecolor{cellborder}{HTML}{CFCFCF}
    \definecolor{cellbackground}{HTML}{F7F7F7}
    
    % prompt
    \makeatletter
    \newcommand{\boxspacing}{\kern\kvtcb@left@rule\kern\kvtcb@boxsep}
    \makeatother
    \newcommand{\prompt}[4]{
        \ttfamily\llap{{\color{#2}[#3]:\hspace{3pt}#4}}\vspace{-\baselineskip}
    }
    

    
    % Prevent overflowing lines due to hard-to-break entities
    \sloppy 
    % Setup hyperref package
    \hypersetup{
      breaklinks=true,  % so long urls are correctly broken across lines
      colorlinks=true,
      urlcolor=urlcolor,
      linkcolor=linkcolor,
      citecolor=citecolor,
      }
    % Slightly bigger margins than the latex defaults
    
    \geometry{verbose,tmargin=1in,bmargin=1in,lmargin=1in,rmargin=1in}
    
    

\begin{document}
    
    \maketitle
    
    

    
    \begin{tcolorbox}[breakable, size=fbox, boxrule=1pt, pad at break*=1mm,colback=cellbackground, colframe=cellborder]
\prompt{In}{incolor}{4}{\boxspacing}
\begin{Verbatim}[commandchars=\\\{\}]
\PY{k+kn}{import} \PY{n+nn}{numpy} \PY{k}{as} \PY{n+nn}{np}
\PY{k+kn}{import} \PY{n+nn}{matplotlib}\PY{n+nn}{.}\PY{n+nn}{pyplot} \PY{k}{as} \PY{n+nn}{plt}
\PY{k+kn}{from} \PY{n+nn}{scipy}\PY{n+nn}{.}\PY{n+nn}{optimize} \PY{k}{import} \PY{n}{curve\PYZus{}fit}
\PY{k+kn}{from} \PY{n+nn}{pylab} \PY{k}{import} \PY{o}{*}
\PY{k+kn}{from} \PY{n+nn}{io} \PY{k}{import} \PY{n}{StringIO}
\PY{k+kn}{from} \PY{n+nn}{statistics} \PY{k}{import} \PY{o}{*}
\PY{k+kn}{import} \PY{n+nn}{subprocess}
\end{Verbatim}
\end{tcolorbox}

    \hypertarget{esperienza-1-termometria-e-calorimetria}{%
\section{Esperienza 1: Termometria e
calorimetria}\label{esperienza-1-termometria-e-calorimetria}}

\textbf{Data}: 21 Ottobre 2019 \textbf{Gruppo}: (V) Ivan Ingrosso ,
Antonio Gonzalez, Pietro Garofalo

    \hypertarget{materiale}{%
\subsection{Materiale}\label{materiale}}

\begin{longtable}[]{@{}rll@{}}
\toprule
Strumenti & Divisione & Portata\tabularnewline
\midrule
\endhead
2 Termometri a mercurio & \(0.2^\circ C\) &
\(100^\circ C\)\tabularnewline
Bilancia & \(0.1 g\) & --\tabularnewline
Calorimetri & -- & \(1 l\)\tabularnewline
Cronometro & \(0.01 s\) &\tabularnewline
\bottomrule
\end{longtable}

    \hypertarget{costante-di-tempo-del-termometro}{%
\subsection{1.1 Costante di tempo del
termometro}\label{costante-di-tempo-del-termometro}}

\hypertarget{relazioni-di-base-per-il-processo-ideale}{%
\subsubsection{Relazioni di base per il processo
ideale}\label{relazioni-di-base-per-il-processo-ideale}}

La relazione che lega la risposta del termometro al tempo è data dalla
seguente formula: \[T(t)=T_{f}+(T_{amb}-T_{f} )e ^{-\frac{t}{\tau}}\]
dove: - \(T(t)\) rappresenta la temperatura mostrata sul termometro
all'istante di tempo \(t\); - \(T_{amb}\) rappresenta la temperatura
riportata sul termometro prima che questo venga inserito nel bagno di
acqua cala, ovvero la temperatura al tempo \(t=0\); - \(T_{f}\)
rappresenta la temperatura dell'acqua calda; - \(\tau\) rappresenta la
costante di tempo del termometro che vogliamo stimare;

\hypertarget{procedimento-di-misura}{%
\subsubsection{Procedimento di misura}\label{procedimento-di-misura}}

\begin{enumerate}
\def\labelenumi{\arabic{enumi}.}
\tightlist
\item
  Nel primo calorimetro versiamo una quantità di acqua alla temperatura
  di \(\sim 54^\circ C\), che rappresenta la nostra \(T_{f}\); nel
  secondo una quantità di acqua a temperatura ambiente;
\item
  Immergiamo il termometro nel bagno di acqua a temperatura ambiente e
  aspettiamo che termalizzi con l'acqua stessa; a termalizzazione
  avvenuta registriamo la temperatura segnata dal termometro come
  \(T_{amb}\)
\item
  Immergiamo il termometro nel calorimetro con l'acqua calda e
  registriamo la temperatura segnta ad intervalli di tempo fissati
  (\(0.5s\)); la strategia adottata per realizzare queste misure è stata
  filmare con uno smartphone il termometro e il cronometro uno vicino
  all'altro, in modo tale che fossero ben visibili sia le misure del
  tempo che quelle delle temperature. Dal momento che abbiamo ripetuto
  l'esperimento due volte, realizzando due video differenti, riportiamo
  l'analisi dati di entrambi, combinando le stime ottenute attraverso
  una media pesata.
\end{enumerate}

\hypertarget{dati}{%
\subsubsection{Dati}\label{dati}}

\texttt{t}: tempo {[}s{]}\\
\texttt{T}: temperatura {[}\(^\circ C\) {]}\\
\texttt{DT}: risoluzione del termometro (distanza tra due tacche)
{[}\(^\circ C\) {]}\\
\texttt{sT}: incertezza (deviazione standard) su T {[} \(^\circ C\) {]}

    Eseguiamo un fit esponenziale sui dati, attraverso la funzione
curve\_fit della libreria scipy. Curve\_fit riceve in ingresso una
funzione (che, in questo caso restituisce un esponenziale), un vettore
contente gli istanti di tempo ad intervalli di \(0.5s\) (asse \(x\)), un
vettore contenente le temperature registrate (asse \(y\)), e un vettore
con lo stesso numero di elementi dei vettori precedenti e contente
l'incertezza associata a ogni misura di temperatura. Assumendo una
distribuizione uniforme nell'intervallo delle divisioni stimiaimo tale
incertezza come \(\frac{DT}{\sqrt{12}}\). L'ultimo parametro di
curve\_fit viene dichiarato TRUE in modo da calcolare le incertezze
assolute. Curve\_fit, oltre a restituire un vettore con i valori dei
parametri stimati (tra i quali a noi, in questa circostanza interessa
solo \(\tau\)), fornisce anche la matrice di covarianza tra i parametri
stessi, dalla quale possiamo ricavare la varianza, e quindi la
deviazione standard, di \(\tau\). Tale varianza rappresenta la varianza
del parametro ottenuto da un fit su dati sintetici, ovvero dati
costruiti prendendo la variabile T sulla curva di best fit e
aggiuggendole una variabile gaussiana (Come calcolare l'incertezza
generando dati sintetici sarà oggetto di una sezione del paragrafo
successivo, per cui, sull'argomento, rimandiamo a 1.2 Calcolo calore
specifico.)

    \hypertarget{video-1-1s}{%
\paragraph{Video 1, 1s}\label{video-1-1s}}

    \begin{tcolorbox}[breakable, size=fbox, boxrule=1pt, pad at break*=1mm,colback=cellbackground, colframe=cellborder]
\prompt{In}{incolor}{7}{\boxspacing}
\begin{Verbatim}[commandchars=\\\{\}]
\PY{n}{s} \PY{o}{=} \PY{n+nb}{open}\PY{p}{(}\PY{l+s+s2}{\PYZdq{}}\PY{l+s+s2}{Datos12.txt}\PY{l+s+s2}{\PYZdq{}}\PY{p}{)}\PY{o}{.}\PY{n}{read}\PY{p}{(}\PY{p}{)}\PY{o}{.}\PY{n}{replace}\PY{p}{(}\PY{l+s+s2}{\PYZdq{}}\PY{l+s+s2}{,}\PY{l+s+s2}{\PYZdq{}}\PY{p}{,} \PY{l+s+s2}{\PYZdq{}}\PY{l+s+s2}{.}\PY{l+s+s2}{\PYZdq{}}\PY{p}{)}
\PY{n}{t}\PY{p}{,}\PY{n}{T} \PY{o}{=} \PY{n}{transpose}\PY{p}{(}\PY{n}{loadtxt}\PY{p}{(}\PY{n}{StringIO}\PY{p}{(}\PY{n}{s}\PY{p}{)}\PY{p}{)}\PY{p}{)}
\PY{n}{DT}\PY{o}{=} \PY{l+m+mf}{0.2}
\PY{n}{dS}\PY{o}{=}\PY{n}{DT}\PY{o}{/}\PY{n}{sqrt}\PY{p}{(}\PY{l+m+mi}{12}\PY{p}{)}\PY{o}{*}\PY{n}{np}\PY{o}{.}\PY{n}{ones\PYZus{}like}\PY{p}{(}\PY{n}{T}\PY{p}{)}
\PY{k}{def} \PY{n+nf}{fit\PYZus{}func}\PY{p}{(}\PY{n}{t}\PY{p}{,} \PY{n}{tau}\PY{p}{,}\PY{n}{TF} \PY{p}{,}\PY{n}{DeltaT}\PY{p}{)}\PY{p}{:}
    \PY{k}{return} \PY{n}{TF} \PY{o}{\PYZhy{}} \PY{n}{DeltaT}\PY{o}{*}\PY{n}{exp}\PY{p}{(}\PY{o}{\PYZhy{}}\PY{n}{t}\PY{o}{/}\PY{n}{tau}\PY{p}{)}
\PY{n}{params}\PY{p}{,}\PY{n}{pcov12} \PY{o}{=} \PY{n}{curve\PYZus{}fit}\PY{p}{(}\PY{n}{fit\PYZus{}func}\PY{p}{,} \PY{n}{t}\PY{p}{,} \PY{n}{T}\PY{p}{,} \PY{n}{sigma}\PY{o}{=}\PY{n}{dS}\PY{p}{,} \PY{n}{absolute\PYZus{}sigma}\PY{o}{=}\PY{k+kc}{True}\PY{p}{)}
\PY{n}{tau12}\PY{p}{,} \PY{n}{T0}\PY{p}{,} \PY{n}{DeltaT} \PY{o}{=} \PY{n}{params}
\PY{n}{plot}\PY{p}{(}\PY{n}{t}\PY{p}{,} \PY{n}{T}\PY{p}{,} \PY{l+s+s2}{\PYZdq{}}\PY{l+s+s2}{o}\PY{l+s+s2}{\PYZdq{}}\PY{p}{)}
\PY{n}{plot}\PY{p}{(}\PY{n}{t}\PY{p}{,} \PY{n}{fit\PYZus{}func}\PY{p}{(}\PY{n}{t}\PY{p}{,} \PY{n}{tau12}\PY{p}{,} \PY{n}{T0}\PY{p}{,} \PY{n}{DeltaT}\PY{p}{)}\PY{p}{)}
\PY{n}{xlabel}\PY{p}{(}\PY{l+s+s2}{\PYZdq{}}\PY{l+s+s2}{t [s]}\PY{l+s+s2}{\PYZdq{}}\PY{p}{)}
\PY{n}{ylabel}\PY{p}{(}\PY{l+s+s2}{\PYZdq{}}\PY{l+s+s2}{T [\PYZdl{}\PYZca{}}\PY{l+s+s2}{\PYZbs{}}\PY{l+s+s2}{circ\PYZdl{}C]}\PY{l+s+s2}{\PYZdq{}}\PY{p}{)}
\PY{n}{plt}\PY{o}{.}\PY{n}{show}\PY{p}{(}\PY{p}{)}
\end{Verbatim}
\end{tcolorbox}

    \begin{center}
    \adjustimage{max size={0.9\linewidth}{0.9\paperheight}}{./Esperienza1_files/./Esperienza1_6_0.pdf}
    \end{center}
    { \hspace*{\fill} \\}
    
    \begin{tcolorbox}[breakable, size=fbox, boxrule=1pt, pad at break*=1mm,colback=cellbackground, colframe=cellborder]
\prompt{In}{incolor}{8}{\boxspacing}
\begin{Verbatim}[commandchars=\\\{\}]
\PY{n+nb}{print}\PY{p}{(}\PY{l+s+s2}{\PYZdq{}}\PY{l+s+s2}{Parametro stimato per tau, con incertezza: }\PY{l+s+si}{\PYZpc{}lf}\PY{l+s+s2}{ }\PY{l+s+si}{\PYZpc{}lf}\PY{l+s+s2}{\PYZdq{}} \PY{o}{\PYZpc{}}\PY{p}{(}\PY{n}{tau12}\PY{p}{,} \PY{n}{np}\PY{o}{.}\PY{n}{sqrt}\PY{p}{(}\PY{n}{pcov12}\PY{p}{[}\PY{l+m+mi}{0}\PY{p}{,}\PY{l+m+mi}{0}\PY{p}{]}\PY{p}{)}\PY{p}{)}\PY{p}{)}
\PY{n+nb}{print}\PY{p}{(}\PY{l+s+s2}{\PYZdq{}}\PY{l+s+s2}{tau = }\PY{l+s+si}{\PYZpc{}.2f}\PY{l+s+s2}{ +/\PYZhy{} }\PY{l+s+si}{\PYZpc{}.2f}\PY{l+s+s2}{ s}\PY{l+s+s2}{\PYZdq{}} \PY{o}{\PYZpc{}} \PY{p}{(}\PY{n}{tau12}\PY{p}{,} \PY{n}{sqrt}\PY{p}{(}\PY{n}{pcov12}\PY{p}{[}\PY{l+m+mi}{0}\PY{p}{,}\PY{l+m+mi}{0}\PY{p}{]}\PY{p}{)}\PY{p}{)}\PY{p}{)}
\end{Verbatim}
\end{tcolorbox}

    \begin{Verbatim}[commandchars=\\\{\}]
Parametro stimato per tau, con incertezza: 10.152235 0.037992
tau = 10.15 +/- 0.04 s
    \end{Verbatim}

    Analizzando i residui si può notare che gli scarti sono contenuti in un
intervallo che è maggiore di quello che ci si aspetterebbe in base ai
calcoli realizzati dal best fit. Inoltre, l'andamento di tali scarti non
sembra puramente casuale, come invece dovrebbe essere se il modello
utilizzato per interpolare i dati fosse corretto. Da questo osserviamo
che, con buona probabilità, l'andamento della temperatura è più
complicato di un esponenziale, motivo per cui i dati sperimentali si
discostano dalla funzione teorica con un errore che è, evidentemente,
non casuale.

    \begin{tcolorbox}[breakable, size=fbox, boxrule=1pt, pad at break*=1mm,colback=cellbackground, colframe=cellborder]
\prompt{In}{incolor}{9}{\boxspacing}
\begin{Verbatim}[commandchars=\\\{\}]
\PY{n}{Tfit} \PY{o}{=} \PY{n}{fit\PYZus{}func}\PY{p}{(}\PY{n}{t}\PY{p}{,} \PY{n}{tau12}\PY{p}{,} \PY{n}{T0}\PY{p}{,} \PY{n}{DeltaT}\PY{p}{)}
\PY{n}{res} \PY{o}{=} \PY{n}{T}\PY{o}{\PYZhy{}}\PY{n}{Tfit}
\PY{n}{plot}\PY{p}{(}\PY{n}{res}\PY{p}{)}\PY{p}{;}
\PY{n}{ylabel}\PY{p}{(}\PY{l+s+s2}{\PYZdq{}}\PY{l+s+s2}{T[\PYZdl{}\PYZca{}}\PY{l+s+s2}{\PYZbs{}}\PY{l+s+s2}{circ C\PYZdl{}]}\PY{l+s+s2}{\PYZdq{}}\PY{p}{)}\PY{p}{;}
\end{Verbatim}
\end{tcolorbox}

    \begin{center}
    \adjustimage{max size={0.9\linewidth}{0.9\paperheight}}{./Esperienza1_files/./Esperienza1_9_0.pdf}
    \end{center}
    { \hspace*{\fill} \\}
    
    \hypertarget{video-2-1-s}{%
\paragraph{Video 2, 1 s}\label{video-2-1-s}}

    \begin{tcolorbox}[breakable, size=fbox, boxrule=1pt, pad at break*=1mm,colback=cellbackground, colframe=cellborder]
\prompt{In}{incolor}{10}{\boxspacing}
\begin{Verbatim}[commandchars=\\\{\}]
\PY{n}{s} \PY{o}{=} \PY{n+nb}{open}\PY{p}{(}\PY{l+s+s2}{\PYZdq{}}\PY{l+s+s2}{Datos22.txt}\PY{l+s+s2}{\PYZdq{}}\PY{p}{)}\PY{o}{.}\PY{n}{read}\PY{p}{(}\PY{p}{)}\PY{o}{.}\PY{n}{replace}\PY{p}{(}\PY{l+s+s2}{\PYZdq{}}\PY{l+s+s2}{,}\PY{l+s+s2}{\PYZdq{}}\PY{p}{,} \PY{l+s+s2}{\PYZdq{}}\PY{l+s+s2}{.}\PY{l+s+s2}{\PYZdq{}}\PY{p}{)}
\PY{n}{t}\PY{p}{,}\PY{n}{T} \PY{o}{=} \PY{n}{transpose}\PY{p}{(}\PY{n}{loadtxt}\PY{p}{(}\PY{n}{StringIO}\PY{p}{(}\PY{n}{s}\PY{p}{)}\PY{p}{)}\PY{p}{)}
\PY{n}{DT}\PY{o}{=} \PY{l+m+mf}{0.2}
\PY{n}{dS}\PY{o}{=}\PY{n}{DT}\PY{o}{/}\PY{n}{sqrt}\PY{p}{(}\PY{l+m+mi}{12}\PY{p}{)}\PY{o}{*}\PY{n}{np}\PY{o}{.}\PY{n}{ones\PYZus{}like}\PY{p}{(}\PY{n}{T}\PY{p}{)}
\PY{k}{def} \PY{n+nf}{fit\PYZus{}func}\PY{p}{(}\PY{n}{t}\PY{p}{,} \PY{n}{tau}\PY{p}{,}\PY{n}{TF} \PY{p}{,}\PY{n}{DeltaT}\PY{p}{)}\PY{p}{:}
    \PY{k}{return} \PY{n}{TF} \PY{o}{\PYZhy{}} \PY{n}{DeltaT}\PY{o}{*}\PY{n}{exp}\PY{p}{(}\PY{o}{\PYZhy{}}\PY{n}{t}\PY{o}{/}\PY{n}{tau}\PY{p}{)}
\PY{n}{params}\PY{p}{,}\PY{n}{pcov22} \PY{o}{=} \PY{n}{curve\PYZus{}fit}\PY{p}{(}\PY{n}{fit\PYZus{}func}\PY{p}{,} \PY{n}{t}\PY{p}{,} \PY{n}{T}\PY{p}{,}\PY{n}{sigma}\PY{o}{=}\PY{n}{dS}\PY{p}{,} \PY{n}{absolute\PYZus{}sigma}\PY{o}{=}\PY{k+kc}{True}\PY{p}{)}
\PY{n}{tau22}\PY{p}{,} \PY{n}{T0}\PY{p}{,} \PY{n}{DeltaT} \PY{o}{=} \PY{n}{params}
\PY{n}{plot}\PY{p}{(}\PY{n}{t}\PY{p}{,} \PY{n}{T}\PY{p}{,} \PY{l+s+s2}{\PYZdq{}}\PY{l+s+s2}{o}\PY{l+s+s2}{\PYZdq{}}\PY{p}{)}
\PY{n}{plot}\PY{p}{(}\PY{n}{t}\PY{p}{,} \PY{n}{fit\PYZus{}func}\PY{p}{(}\PY{n}{t}\PY{p}{,} \PY{n}{tau22}\PY{p}{,} \PY{n}{T0}\PY{p}{,} \PY{n}{DeltaT}\PY{p}{)}\PY{p}{)}
\PY{n}{xlabel}\PY{p}{(}\PY{l+s+s2}{\PYZdq{}}\PY{l+s+s2}{t [s]}\PY{l+s+s2}{\PYZdq{}}\PY{p}{)}
\PY{n}{ylabel}\PY{p}{(}\PY{l+s+s2}{\PYZdq{}}\PY{l+s+s2}{T [\PYZdl{}\PYZca{}}\PY{l+s+s2}{\PYZbs{}}\PY{l+s+s2}{circ\PYZdl{}C]}\PY{l+s+s2}{\PYZdq{}}\PY{p}{)}
\PY{n}{plt}\PY{o}{.}\PY{n}{show}\PY{p}{(}\PY{p}{)}
\end{Verbatim}
\end{tcolorbox}

    \begin{center}
    \adjustimage{max size={0.9\linewidth}{0.9\paperheight}}{./Esperienza1_files/./Esperienza1_11_0.pdf}
    \end{center}
    { \hspace*{\fill} \\}
    
    \begin{tcolorbox}[breakable, size=fbox, boxrule=1pt, pad at break*=1mm,colback=cellbackground, colframe=cellborder]
\prompt{In}{incolor}{11}{\boxspacing}
\begin{Verbatim}[commandchars=\\\{\}]
\PY{n+nb}{print}\PY{p}{(}\PY{l+s+s2}{\PYZdq{}}\PY{l+s+s2}{Parametro stimato per tau, con incertezza: }\PY{l+s+si}{\PYZpc{}lf}\PY{l+s+s2}{, }\PY{l+s+si}{\PYZpc{}lf}\PY{l+s+s2}{\PYZdq{}} \PY{o}{\PYZpc{}}\PY{p}{(}\PY{n}{tau22}\PY{p}{,} \PY{n}{np}\PY{o}{.}\PY{n}{sqrt}\PY{p}{(}\PY{n}{pcov22}\PY{p}{[}\PY{l+m+mi}{0}\PY{p}{,}\PY{l+m+mi}{0}\PY{p}{]}\PY{p}{)}\PY{p}{)}\PY{p}{)}
\PY{n+nb}{print}\PY{p}{(}\PY{l+s+s2}{\PYZdq{}}\PY{l+s+s2}{tau = }\PY{l+s+si}{\PYZpc{}.2f}\PY{l+s+s2}{ +/\PYZhy{} }\PY{l+s+si}{\PYZpc{}.2f}\PY{l+s+s2}{ s}\PY{l+s+s2}{\PYZdq{}} \PY{o}{\PYZpc{}} \PY{p}{(}\PY{n}{tau22}\PY{p}{,} \PY{n}{sqrt}\PY{p}{(}\PY{n}{pcov22}\PY{p}{[}\PY{l+m+mi}{0}\PY{p}{,}\PY{l+m+mi}{0}\PY{p}{]}\PY{p}{)}\PY{p}{)}\PY{p}{)}
\end{Verbatim}
\end{tcolorbox}

    \begin{Verbatim}[commandchars=\\\{\}]
Parametro stimato per tau, con incertezza: 9.705420, 0.024353
tau = 9.71 +/- 0.02 s
    \end{Verbatim}

    \begin{tcolorbox}[breakable, size=fbox, boxrule=1pt, pad at break*=1mm,colback=cellbackground, colframe=cellborder]
\prompt{In}{incolor}{12}{\boxspacing}
\begin{Verbatim}[commandchars=\\\{\}]
\PY{n}{Tfit} \PY{o}{=} \PY{n}{fit\PYZus{}func}\PY{p}{(}\PY{n}{t}\PY{p}{,} \PY{n}{tau12}\PY{p}{,} \PY{n}{T0}\PY{p}{,} \PY{n}{DeltaT}\PY{p}{)}
\PY{n}{res} \PY{o}{=} \PY{n}{T}\PY{o}{\PYZhy{}}\PY{n}{Tfit}
\PY{n}{plot}\PY{p}{(}\PY{n}{res}\PY{p}{)}\PY{p}{;}
\PY{n}{ylabel}\PY{p}{(}\PY{l+s+s2}{\PYZdq{}}\PY{l+s+s2}{T[\PYZdl{}\PYZca{}}\PY{l+s+s2}{\PYZbs{}}\PY{l+s+s2}{circ C\PYZdl{}]}\PY{l+s+s2}{\PYZdq{}}\PY{p}{)}\PY{p}{;}
\end{Verbatim}
\end{tcolorbox}

    \begin{center}
    \adjustimage{max size={0.9\linewidth}{0.9\paperheight}}{./Esperienza1_files/./Esperienza1_13_0.pdf}
    \end{center}
    { \hspace*{\fill} \\}
    
    Combiniamo le due stime di \(\tau\) in un'unica stima attraverso la
media pesata di tali stime con le incertezze come pesi:

    \begin{tcolorbox}[breakable, size=fbox, boxrule=1pt, pad at break*=1mm,colback=cellbackground, colframe=cellborder]
\prompt{In}{incolor}{13}{\boxspacing}
\begin{Verbatim}[commandchars=\\\{\}]
\PY{n}{N2} \PY{o}{=} \PY{n}{tau12}\PY{o}{/}\PY{n}{pcov12}\PY{p}{[}\PY{l+m+mi}{0}\PY{p}{,}\PY{l+m+mi}{0}\PY{p}{]}\PY{o}{+}\PY{n}{tau22}\PY{o}{/}\PY{n}{pcov22}\PY{p}{[}\PY{l+m+mi}{0}\PY{p}{,}\PY{l+m+mi}{0}\PY{p}{]}
\PY{n}{D2}\PY{o}{=}\PY{l+m+mi}{1}\PY{o}{/}\PY{n}{pcov12}\PY{p}{[}\PY{l+m+mi}{0}\PY{p}{,}\PY{l+m+mi}{0}\PY{p}{]}\PY{o}{+}\PY{l+m+mi}{1}\PY{o}{/}\PY{n}{pcov22}\PY{p}{[}\PY{l+m+mi}{0}\PY{p}{,}\PY{l+m+mi}{0}\PY{p}{]}
\PY{n}{P2}\PY{o}{=}\PY{l+m+mi}{1}\PY{o}{/}\PY{n}{sqrt}\PY{p}{(}\PY{n}{D2}\PY{p}{)}
\PY{n+nb}{print}\PY{p}{(}\PY{l+s+s2}{\PYZdq{}}\PY{l+s+s2}{tau = }\PY{l+s+si}{\PYZpc{}.2f}\PY{l+s+s2}{ +/\PYZhy{} }\PY{l+s+si}{\PYZpc{}.2f}\PY{l+s+s2}{ s}\PY{l+s+s2}{\PYZdq{}} \PY{o}{\PYZpc{}} \PY{p}{(}\PY{n}{N2}\PY{o}{/}\PY{n}{D2}\PY{p}{,}\PY{n}{P2} \PY{p}{)}\PY{p}{)}
\PY{c+c1}{\PYZsh{}\PYZsh{}\PYZsh{}\PYZsh{}\PYZsh{}Tau  videos a 1.0s\PYZsh{}\PYZsh{}\PYZsh{}\PYZsh{}\PYZsh{}\PYZsh{}}
\end{Verbatim}
\end{tcolorbox}

    \begin{Verbatim}[commandchars=\\\{\}]
tau = 9.84 +/- 0.02 s
    \end{Verbatim}

    \hypertarget{verifica-dei-risultati-del-fit-attraverso-la-legge-di-raffreddamento-di-newton}{%
\subsubsection{Verifica dei risultati del fit attraverso la legge di
raffreddamento di
Newton}\label{verifica-dei-risultati-del-fit-attraverso-la-legge-di-raffreddamento-di-newton}}

E' possibile calcolare analiticamente il valore di \(\tau\) attraverso
la legge di raffreddamento di Newton \[ \tau=\frac{C}{hA}\] dove - C é
la capacitá termica del mercurio, che possiamo calolare attraverso il
calore specifico e la massa, il primo noto, la seconda ricavata dalla
relazione \(v\rho = m\), dove $\rho $ è la densità del mercurio.
Assumendo una forma cilindrica per il bulbo, chiamando \(l\) l'altezza e
\(d\) il diametro, si ottiene che \[ C=cv\rho\] - h é il coefficiente di
convezione in acqua statica; - A é la superficie del bulbo.

Il seguente codice, a partire dai valori dati delle variabili
interessate, calcola il valore di \(\tau\) attraverso la legge di
raffreddamento di Newton.

\hypertarget{dati}{%
\paragraph{Dati}\label{dati}}

\texttt{c}: Capacità termica del mercurio {[}\(\frac{J}{Kg K}\){]};\\
\texttt{rho}: Densità del merucrio {[}$\frac{Kg}{m^3} $ {]};\\
\texttt{l}: altezza del bulbo {[}\(m\){]};\\
\texttt{d}: diametro del bulco {[}\(m\){]}, per cui \texttt{r} sarà il
raggio;\\
\texttt{h}: coefficiente di convezione in acqua statica
{[}$\frac{W}{m^2 K} ${]}

    \begin{tcolorbox}[breakable, size=fbox, boxrule=1pt, pad at break*=1mm,colback=cellbackground, colframe=cellborder]
\prompt{In}{incolor}{14}{\boxspacing}
\begin{Verbatim}[commandchars=\\\{\}]
\PY{n}{c} \PY{o}{=} \PY{l+m+mi}{140}
\PY{n}{rho} \PY{o}{=} \PY{l+m+mf}{1.3e4}
\PY{n}{l} \PY{o}{=} \PY{l+m+mf}{0.011}
\PY{n}{r} \PY{o}{=} \PY{l+m+mf}{0.003}
\PY{n}{h} \PY{o}{=} \PY{l+m+mi}{750}
\PY{n}{tau} \PY{o}{=} \PY{n}{c}\PY{o}{*}\PY{n}{r}\PY{o}{*}\PY{o}{*}\PY{l+m+mi}{2}\PY{o}{*}\PY{n}{np}\PY{o}{.}\PY{n}{pi}\PY{o}{*}\PY{n}{l}\PY{o}{*}\PY{n}{rho}\PY{o}{/}\PY{p}{(}\PY{n}{h}\PY{o}{*}\PY{l+m+mi}{2}\PY{o}{*}\PY{n}{r}\PY{o}{*}\PY{n}{np}\PY{o}{.}\PY{n}{pi}\PY{o}{*}\PY{n}{l}\PY{p}{)}
\PY{n+nb}{print}\PY{p}{(}\PY{l+s+s2}{\PYZdq{}}\PY{l+s+s2}{tau = }\PY{l+s+si}{\PYZpc{}.1f}\PY{l+s+s2}{ s }\PY{l+s+s2}{\PYZdq{}} \PY{o}{\PYZpc{}} \PY{p}{(}\PY{n}{tau}\PY{p}{)}\PY{p}{)}
\end{Verbatim}
\end{tcolorbox}

    \begin{Verbatim}[commandchars=\\\{\}]
tau = 3.6 s
    \end{Verbatim}

    \hypertarget{conclusioni}{%
\subsubsection{Conclusioni}\label{conclusioni}}

\begin{longtable}[]{@{}cc@{}}
\toprule
\(\tau\) stimato {[}s{]} & $\tau $ analitico {[}s{]}\tabularnewline
\midrule
\endhead
9.83 \(\pm\) 0.02 & 3.6\tabularnewline
\bottomrule
\end{longtable}

Confrontando i valori ottenuti dalle misure con il valore di \(\tau\)
calcolato analiticamente notiamo che questi valori sono decisamente poco
consistenti. Alla luce di questa inconsistenza ipotiziamo che il
termometro con il quale abbiamo effettuato le misure non sia un
termometro a mercurio.

    \hypertarget{calcolo-calore-specifico}{%
\subsection{1.2 Calcolo calore
specifico}\label{calcolo-calore-specifico}}

    \hypertarget{relazione-di-base}{%
\subsubsection{Relazione di base}\label{relazione-di-base}}

Si considerino due sistemi termodinamici, un costituito da un solido,
con temperatura \(T_{s}\) e di massa \(m_{m}\), e uno costituito da un
recipiente chiuso e dotato di pareti adiabatiche, riempito con acqua a
temperatura \(T_{a}\) e massa \(M_{a}\). Inserendo il sistema solido
nell'acqua e misurando la temperatura con un termometro, si osserva una
variazione del valore di questa grandezza che si porta ad un livello
\(T_{e}\) intermedio tra la tempretura dell'acqua e la temperatura del
solido. Come conseguenza di tale variazione cambierà anche il valore
della funzione di stato energia interna dei due sistemi, costituenti un
sistema unico dopo il mescolamento. Nello specifico, definendo
\(\Delta U_{s}\) e \(\Delta U_{a}\) le variazioni di energia interna del
sistema solido e dell'acqua, per il primo principio della termodinamica
si avrà

    \begin{itemize}
\tightlist
\item
  \textbf{Sistema Solido} \(\Delta U_{s} = Q_{s} - L_{s}\)\\
\item
  \textbf{Acqua} \(\Delta U_{a} = Q_{a} - L_{a}\)\\
\item
  \textbf{Solido \(+\) acqua} $ \Delta U = \Delta U\_\{s\} +
  \Delta U\_\{a\} = (Q\_\{s\} + Q\_\{a\}) - (L\_\{s\} + L\_\{a\}) =
  0$\\
  Essendo rigide le pareti del sistema, non viene compiuto lavoro e
  \[(L_{s} + L_{a}) = 0 \] per cui \[ (Q_{s} = - Q_{a})\]
\end{itemize}

Definendo - \(c_{m}\) il calore specifico del materiale ; - \(c{a}\) il
calore specifico dell'acqua;\\
si può riscrivere la precedente formula come
\[  m_{m}c_{m}(T_{e} - T_{s}) = M_{a}c_{a}(T_{a} - T_{e})\] In realtà,
poichè il calorimetro non partecipa in egual misura alla trasformazione,
si introduce una grandezza \(M^*\), detta massa equivalente del
calorimetro, il cui valore è \(25 \pm 5 g\) (Per una discussione circa
questa grandezza, si veda
Section \ref{20verifica20dellequivalente20in20acqua20del20calorimetro};
La formula utilizzata per la stima del calore specifico è dunque

    \[ c_{m} = c_{a} \frac{(M^* + M_{a})(T_{a}-T_{e})}{m_{m}(T_{e}-T_{m})}\]

\hypertarget{procedimento-di-misura}{%
\subsubsection{Procedimento di misura}\label{procedimento-di-misura}}

\begin{itemize}
\tightlist
\item
  Selezionato uno dei materiali proposti, ne è stata misurata la massa;
\item
  Il materiale selezionao è stato inserito in un bagno di acqua a
  temperatura ambiente. Successivamente ne è stata registrata la
  temperatura di equilibrio a termalizzazione avvenuta;
\item
  E' stato preparato un bagno di acqua calda all'interno del calorimetro
  delle mescolanze, e dopo averne registrato la temperatura è stato
  immerso il materiale all'interno del calorimetro;
\item
  A termalizzazione avvenuta, è stata registrata la temperatura di
  equilobrio.\\
  Durante la misura l'apertura del calorimetro è stata tenuta chiusa con
  un tappo per ridurre la dispersione di calore con l'ambiente.\\
  Questa sequenza di operazioni è stata ripetuta una volta per ciascun
  materiale.
\end{itemize}

\hypertarget{dati}{%
\subsubsection{Dati}\label{dati}}

\texttt{ca} Calore specifico dell'acqua (\(\frac{cal}{g ^\circ C}\))\\
\texttt{Ma} Massa di acqua calda versata nel calorimetro (\(g\) )\\
\texttt{DMa} Incertezza associata alla misura (\(g\) )\\
\texttt{Mm} Massa materiale (\(g\))\\
\texttt{DMm} Incertezza associata alla misura (\(g\))\\
\texttt{Meq} Equivalente in acqua delcalorimetro (\(g\))\\
\texttt{DMeq} Incertezza associata alla misura (\(g\))\\
\texttt{Ta} Temperatura dell'acqua calda ( $ ^{\circ}$ C )\\
\texttt{DTa} Incertezza associata alla misura ( $ ^{\circ}$ C )\\
\texttt{Tm} Temperatura all'equilibrio ( $ ^{\circ}$ C )\\
\texttt{DTm} Incertezza associata alla misura  ( $ ^{\circ}$ C )\\
\texttt{Teq} Temperatura all'equilibrio ( $ ^{\circ}$ C )\\
\texttt{DTeq} Incertezza associata alla misura  ( $ ^{\circ}$ C )


Per la stima delle incertezze sulle misure è stato preso l'errore
casuale dovuto alla risoluzione finita dello strumento. Tale incertezza
viene valutata come \[ \frac{I_{s}}{\sqrt{12}}\] dove \(I_{s}\)
rappresenta l'intervallo delle divisioni o della risoluzione dello
strumento.\\
Le variabili utilizzate nel seguente codice sono degli array di tre
elementi ciascuno, il primo dei quali contiene le misure del materiale
1, il secondo quelle del materiale 2, il terzo quelle del materiale 3.

    \begin{tcolorbox}[breakable, size=fbox, boxrule=1pt, pad at break*=1mm,colback=cellbackground, colframe=cellborder]
\prompt{In}{incolor}{27}{\boxspacing}
\begin{Verbatim}[commandchars=\\\{\}]
\PY{n}{Ma} \PY{o}{=} \PY{n}{np}\PY{o}{.}\PY{n}{array}\PY{p}{(}\PY{p}{[}\PY{l+m+mf}{202.}\PY{p}{,} \PY{l+m+mf}{202.8}\PY{p}{,} \PY{l+m+mf}{200.7}\PY{p}{]}\PY{p}{)}
\PY{n}{DMa} \PY{o}{=} \PY{l+m+mf}{0.1}\PY{o}{/}\PY{n}{np}\PY{o}{.}\PY{n}{sqrt}\PY{p}{(}\PY{l+m+mi}{12}\PY{p}{)}\PY{o}{*}\PY{n}{np}\PY{o}{.}\PY{n}{ones\PYZus{}like}\PY{p}{(}\PY{n}{Ma}\PY{p}{)}
\PY{n}{Mm} \PY{o}{=} \PY{n}{np}\PY{o}{.}\PY{n}{array}\PY{p}{(}\PY{p}{[}\PY{l+m+mf}{79.}\PY{p}{,} \PY{l+m+mf}{73.1}\PY{p}{,} \PY{l+m+mf}{194.4}\PY{p}{]}\PY{p}{)}
\PY{n}{DMm} \PY{o}{=} \PY{l+m+mf}{0.1}\PY{o}{/}\PY{n}{np}\PY{o}{.}\PY{n}{sqrt}\PY{p}{(}\PY{l+m+mi}{12}\PY{p}{)}\PY{o}{*}\PY{n}{np}\PY{o}{.}\PY{n}{ones\PYZus{}like}\PY{p}{(}\PY{n}{Mm}\PY{p}{)}
\PY{n}{Meq} \PY{o}{=} \PY{l+m+mf}{25.} 
\PY{n}{DMeq} \PY{o}{=} \PY{l+m+mf}{5.}
\PY{n}{Ta} \PY{o}{=} \PY{n}{np}\PY{o}{.}\PY{n}{array}\PY{p}{(}\PY{p}{[}\PY{l+m+mf}{42.2}\PY{p}{,} \PY{l+m+mf}{40.8}\PY{p}{,} \PY{l+m+mf}{39.6}\PY{p}{]}\PY{p}{)}
\PY{n}{DTa} \PY{o}{=} \PY{l+m+mf}{0.2}\PY{o}{/}\PY{n}{np}\PY{o}{.}\PY{n}{sqrt}\PY{p}{(}\PY{l+m+mi}{12}\PY{p}{)}\PY{o}{*}\PY{n}{np}\PY{o}{.}\PY{n}{ones\PYZus{}like}\PY{p}{(}\PY{n}{Ta}\PY{p}{)}
\PY{n}{Tm} \PY{o}{=} \PY{l+m+mf}{28.2}\PY{o}{*}\PY{n}{np}\PY{o}{.}\PY{n}{ones\PYZus{}like}\PY{p}{(}\PY{n}{Ta}\PY{p}{)}
\PY{n}{DTm} \PY{o}{=} \PY{l+m+mf}{0.2}\PY{o}{/}\PY{n}{np}\PY{o}{.}\PY{n}{sqrt}\PY{p}{(}\PY{l+m+mi}{12}\PY{p}{)}\PY{o}{*}\PY{n}{np}\PY{o}{.}\PY{n}{ones\PYZus{}like}\PY{p}{(}\PY{n}{Tm}\PY{p}{)}
\PY{n}{Teq} \PY{o}{=} \PY{n}{np}\PY{o}{.}\PY{n}{array}\PY{p}{(}\PY{p}{[}\PY{l+m+mf}{40.8}\PY{p}{,} \PY{l+m+mf}{39.}\PY{p}{,} \PY{l+m+mf}{38.2}\PY{p}{]}\PY{p}{)}
\PY{n}{DTeq} \PY{o}{=} \PY{l+m+mf}{0.2}\PY{o}{/}\PY{n}{np}\PY{o}{.}\PY{n}{sqrt}\PY{p}{(}\PY{l+m+mi}{12}\PY{p}{)}\PY{o}{*}\PY{n}{np}\PY{o}{.}\PY{n}{ones\PYZus{}like}\PY{p}{(}\PY{n}{Teq}\PY{p}{)}
\PY{n+nb}{print}\PY{p}{(}\PY{l+s+s2}{\PYZdq{}}\PY{l+s+s2}{Solido n°    Ma +/\PYZhy{} DMa [g]: }\PY{l+s+s2}{\PYZdq{}}\PY{p}{)}
\PY{k}{for} \PY{n}{i} \PY{o+ow}{in} \PY{n+nb}{range}\PY{p}{(}\PY{l+m+mi}{0}\PY{p}{,} \PY{l+m+mi}{3}\PY{p}{)}\PY{p}{:}
    \PY{n+nb}{print}\PY{p}{(}\PY{l+s+s2}{\PYZdq{}}\PY{l+s+si}{\PYZpc{}i}\PY{l+s+s2}{            }\PY{l+s+si}{\PYZpc{}.2f}\PY{l+s+s2}{ +/\PYZhy{} }\PY{l+s+si}{\PYZpc{}.2f}\PY{l+s+s2}{\PYZdq{}} \PY{o}{\PYZpc{}}\PY{p}{(}\PY{n}{i}\PY{o}{+}\PY{l+m+mi}{1}\PY{p}{,} \PY{n}{Ma}\PY{p}{[}\PY{n}{i}\PY{p}{]}\PY{p}{,} \PY{n}{DMa}\PY{p}{[}\PY{n}{i}\PY{p}{]}\PY{p}{)}\PY{p}{)}
\PY{n+nb}{print}\PY{p}{(}\PY{l+s+s2}{\PYZdq{}}\PY{l+s+s2}{Solido n°    Mm +/\PYZhy{} DMm [g]: }\PY{l+s+s2}{\PYZdq{}}\PY{p}{)}
\PY{k}{for} \PY{n}{i} \PY{o+ow}{in} \PY{n+nb}{range}\PY{p}{(}\PY{l+m+mi}{0}\PY{p}{,} \PY{l+m+mi}{3}\PY{p}{)}\PY{p}{:}
    \PY{n+nb}{print}\PY{p}{(}\PY{l+s+s2}{\PYZdq{}}\PY{l+s+si}{\PYZpc{}i}\PY{l+s+s2}{            }\PY{l+s+si}{\PYZpc{}.2f}\PY{l+s+s2}{ +/\PYZhy{} }\PY{l+s+si}{\PYZpc{}.2f}\PY{l+s+s2}{\PYZdq{}} \PY{o}{\PYZpc{}}\PY{p}{(}\PY{n}{i}\PY{o}{+}\PY{l+m+mi}{1}\PY{p}{,} \PY{n}{Mm}\PY{p}{[}\PY{n}{i}\PY{p}{]}\PY{p}{,} \PY{n}{DMm}\PY{p}{[}\PY{n}{i}\PY{p}{]}\PY{p}{)}\PY{p}{)}
\PY{n+nb}{print}\PY{p}{(}\PY{l+s+s2}{\PYZdq{}}\PY{l+s+s2}{Solido n°    Ta +/\PYZhy{} DTa [°C]: }\PY{l+s+s2}{\PYZdq{}}\PY{p}{)}
\PY{k}{for} \PY{n}{i} \PY{o+ow}{in} \PY{n+nb}{range}\PY{p}{(}\PY{l+m+mi}{0}\PY{p}{,} \PY{l+m+mi}{3}\PY{p}{)}\PY{p}{:}
    \PY{n+nb}{print}\PY{p}{(}\PY{l+s+s2}{\PYZdq{}}\PY{l+s+si}{\PYZpc{}i}\PY{l+s+s2}{            }\PY{l+s+si}{\PYZpc{}.2f}\PY{l+s+s2}{ +/\PYZhy{} }\PY{l+s+si}{\PYZpc{}.2f}\PY{l+s+s2}{\PYZdq{}} \PY{o}{\PYZpc{}}\PY{p}{(}\PY{n}{i}\PY{o}{+}\PY{l+m+mi}{1}\PY{p}{,} \PY{n}{Ta}\PY{p}{[}\PY{n}{i}\PY{p}{]}\PY{p}{,} \PY{n}{DTa}\PY{p}{[}\PY{n}{i}\PY{p}{]}\PY{p}{)}\PY{p}{)}
\PY{n+nb}{print}\PY{p}{(}\PY{l+s+s2}{\PYZdq{}}\PY{l+s+s2}{Solido n°    Tm +/\PYZhy{} DTm [°C]: }\PY{l+s+s2}{\PYZdq{}}\PY{p}{)}
\PY{k}{for} \PY{n}{i} \PY{o+ow}{in} \PY{n+nb}{range}\PY{p}{(}\PY{l+m+mi}{0}\PY{p}{,} \PY{l+m+mi}{3}\PY{p}{)}\PY{p}{:}
    \PY{n+nb}{print}\PY{p}{(}\PY{l+s+s2}{\PYZdq{}}\PY{l+s+si}{\PYZpc{}i}\PY{l+s+s2}{            }\PY{l+s+si}{\PYZpc{}.2f}\PY{l+s+s2}{ +/\PYZhy{} }\PY{l+s+si}{\PYZpc{}.2f}\PY{l+s+s2}{\PYZdq{}} \PY{o}{\PYZpc{}}\PY{p}{(}\PY{n}{i}\PY{o}{+}\PY{l+m+mi}{1}\PY{p}{,} \PY{n}{Tm}\PY{p}{[}\PY{n}{i}\PY{p}{]}\PY{p}{,} \PY{n}{DTm}\PY{p}{[}\PY{n}{i}\PY{p}{]}\PY{p}{)}\PY{p}{)}
\PY{n+nb}{print}\PY{p}{(}\PY{l+s+s2}{\PYZdq{}}\PY{l+s+s2}{Solido n°    Teq +/\PYZhy{} DTeq [°C]: }\PY{l+s+s2}{\PYZdq{}}\PY{p}{)}
\PY{k}{for} \PY{n}{i} \PY{o+ow}{in} \PY{n+nb}{range}\PY{p}{(}\PY{l+m+mi}{0}\PY{p}{,} \PY{l+m+mi}{3}\PY{p}{)}\PY{p}{:}
    \PY{n+nb}{print}\PY{p}{(}\PY{l+s+s2}{\PYZdq{}}\PY{l+s+si}{\PYZpc{}i}\PY{l+s+s2}{            }\PY{l+s+si}{\PYZpc{}.2f}\PY{l+s+s2}{ +/\PYZhy{} }\PY{l+s+si}{\PYZpc{}.2f}\PY{l+s+s2}{\PYZdq{}} \PY{o}{\PYZpc{}}\PY{p}{(}\PY{n}{i}\PY{o}{+}\PY{l+m+mi}{1}\PY{p}{,} \PY{n}{Teq}\PY{p}{[}\PY{n}{i}\PY{p}{]}\PY{p}{,} \PY{n}{DTeq}\PY{p}{[}\PY{n}{i}\PY{p}{]}\PY{p}{)}\PY{p}{)}
\end{Verbatim}
\end{tcolorbox}

    \begin{Verbatim}[commandchars=\\\{\}]
Solido n°    Ma +/- DMa [g]:
1            202.00 +/- 0.03
2            202.80 +/- 0.03
3            200.70 +/- 0.03
Solido n°    Mm +/- DMm [g]:
1            79.00 +/- 0.03
2            73.10 +/- 0.03
3            194.40 +/- 0.03
Solido n°    Ta +/- DTa [°C]:
1            42.20 +/- 0.06
2            40.80 +/- 0.06
3            39.60 +/- 0.06
Solido n°    Tm +/- DTm [°C]:
1            28.20 +/- 0.06
2            28.20 +/- 0.06
3            28.20 +/- 0.06
Solido n°    Teq +/- DTeq [°C]:
1            40.80 +/- 0.06
2            39.00 +/- 0.06
3            38.20 +/- 0.06
    \end{Verbatim}

    Calore specifico acqua

Con $4.18 J = 1 ca $

    \begin{tcolorbox}[breakable, size=fbox, boxrule=1pt, pad at break*=1mm,colback=cellbackground, colframe=cellborder]
\prompt{In}{incolor}{16}{\boxspacing}
\begin{Verbatim}[commandchars=\\\{\}]
\PY{n}{ca} \PY{o}{=} \PY{l+m+mf}{1.}
\end{Verbatim}
\end{tcolorbox}

    \hypertarget{calcolo}{%
\paragraph{Calcolo}\label{calcolo}}

Per il calcolo delle incertezze \(Dcm\) è stata utilizzata la formula di
propagazione delle incertezze (ricavata sviluppando in serie di Taylor
la funzione che lega \(cm\) alle varaibili \(Meq\), \(Ma\), \(ca\),
\(Ta\), \(Teq\), \(Mm\), \(Tm\) intorno al valore medio:
\[ \delta_{cm} = \sqrt{(\frac{\partial cm}{\partial Meq}\delta Meq)^2 + (\frac{\partial cm}{\partial Ma}\delta Ma)^2 + (\frac{\partial cm}{\partial Ta}\delta Ta)^2 + (\frac{\partial cm}{\partial Teq}\delta Teq)^2 + (\frac{\partial cm}{\partial Mm}\delta Mm)^2 + (\frac{\partial cm}{\partial Tm}\delta Tm)^2}\]

    \begin{tcolorbox}[breakable, size=fbox, boxrule=1pt, pad at break*=1mm,colback=cellbackground, colframe=cellborder]
\prompt{In}{incolor}{17}{\boxspacing}
\begin{Verbatim}[commandchars=\\\{\}]
\PY{n}{cm} \PY{o}{=} \PY{p}{(}\PY{n}{Meq} \PY{o}{+} \PY{n}{Ma}\PY{p}{)}\PY{o}{*}\PY{n}{ca}\PY{o}{*}\PY{p}{(}\PY{n}{Ta}\PY{o}{\PYZhy{}}\PY{n}{Teq}\PY{p}{)}\PY{o}{/}\PY{p}{(}\PY{n}{Mm}\PY{o}{*}\PY{p}{(}\PY{n}{Teq}\PY{o}{\PYZhy{}}\PY{n}{Tm}\PY{p}{)}\PY{p}{)}
\PY{n}{Dcm\PYZus{}Meq} \PY{o}{=} \PY{n}{ca}\PY{o}{*}\PY{p}{(}\PY{n}{Ta} \PY{o}{\PYZhy{}} \PY{n}{Teq}\PY{p}{)}\PY{o}{/}\PY{p}{(}\PY{n}{Mm}\PY{o}{*}\PY{p}{(}\PY{n}{Teq}\PY{o}{\PYZhy{}}\PY{n}{Tm}\PY{p}{)}\PY{p}{)}
\PY{n}{Dcm\PYZus{}Ma} \PY{o}{=} \PY{n}{ca}\PY{o}{*}\PY{p}{(}\PY{n}{Ta} \PY{o}{\PYZhy{}} \PY{n}{Teq}\PY{p}{)}\PY{o}{/}\PY{p}{(}\PY{n}{Mm}\PY{o}{*}\PY{p}{(}\PY{n}{Teq}\PY{o}{\PYZhy{}}\PY{n}{Tm}\PY{p}{)}\PY{p}{)}
\PY{n}{Dcm\PYZus{}Ta} \PY{o}{=} \PY{p}{(}\PY{n}{Meq}\PY{o}{+}\PY{n}{Ma}\PY{p}{)}\PY{o}{*}\PY{n}{ca}\PY{o}{/}\PY{p}{(}\PY{n}{Mm}\PY{o}{*}\PY{p}{(}\PY{n}{Teq}\PY{o}{\PYZhy{}}\PY{n}{Tm}\PY{p}{)}\PY{p}{)}
\PY{n}{Dcm\PYZus{}Teq} \PY{o}{=} \PY{p}{(}\PY{n}{Meq} \PY{o}{+} \PY{n}{Ma}\PY{p}{)}\PY{o}{*}\PY{p}{(}\PY{n}{Tm}\PY{o}{\PYZhy{}}\PY{n}{Ta}\PY{p}{)}\PY{o}{*}\PY{n}{ca}\PY{o}{/}\PY{p}{(}\PY{n}{Mm}\PY{o}{*}\PY{p}{(}\PY{n}{Teq}\PY{o}{\PYZhy{}}\PY{n}{Tm}\PY{p}{)}\PY{o}{*}\PY{o}{*}\PY{l+m+mi}{2}\PY{p}{)}
\PY{n}{Dcm\PYZus{}Teqq} \PY{o}{=} \PY{p}{(}\PY{n}{Meq} \PY{o}{+} \PY{n}{Ma}\PY{p}{)}\PY{o}{*}\PY{n}{ca}\PY{o}{/}\PY{p}{(}\PY{n}{Mm}\PY{p}{)}\PY{o}{*}\PY{p}{(}\PY{n}{Tm} \PY{o}{+} \PY{n}{Ta}\PY{p}{)}\PY{o}{/}\PY{p}{(}\PY{p}{(}\PY{n}{Teq}\PY{o}{\PYZhy{}} \PY{n}{Tm}\PY{p}{)}\PY{o}{*}\PY{o}{*}\PY{l+m+mi}{2}\PY{p}{)}
\PY{n}{Dcm\PYZus{}Mm} \PY{o}{=} \PY{o}{\PYZhy{}} \PY{p}{(}\PY{n}{Meq} \PY{o}{+} \PY{n}{Ma}\PY{p}{)}\PY{o}{*}\PY{n}{ca}\PY{o}{*}\PY{p}{(}\PY{n}{Ta}\PY{o}{\PYZhy{}}\PY{n}{Teq}\PY{p}{)}\PY{o}{/}\PY{p}{(}\PY{p}{(}\PY{n}{Teq}\PY{o}{\PYZhy{}}\PY{n}{Tm}\PY{p}{)}\PY{o}{*}\PY{n}{Mm}\PY{o}{*}\PY{o}{*}\PY{l+m+mi}{2}\PY{p}{)}
\PY{n}{Dcm\PYZus{}Tm} \PY{o}{=} \PY{p}{(}\PY{n}{Meq} \PY{o}{+} \PY{n}{Ma}\PY{p}{)}\PY{o}{*}\PY{n}{ca}\PY{o}{*}\PY{p}{(}\PY{n}{Ta}\PY{o}{\PYZhy{}}\PY{n}{Teq}\PY{p}{)}\PY{o}{/}\PY{p}{(}\PY{n}{Mm}\PY{o}{*}\PY{p}{(}\PY{n}{Teq} \PY{o}{\PYZhy{}} \PY{n}{Tm}\PY{p}{)}\PY{o}{*}\PY{o}{*}\PY{l+m+mi}{2}\PY{p}{)}
\PY{n}{Dcm} \PY{o}{=} \PY{n}{np}\PY{o}{.}\PY{n}{sqrt}\PY{p}{(}\PY{n}{Dcm\PYZus{}Meq}\PY{o}{*}\PY{o}{*}\PY{l+m+mi}{2}\PY{o}{*}\PY{n}{DMeq}\PY{o}{*}\PY{o}{*}\PY{l+m+mi}{2} \PY{o}{+} \PY{n}{Dcm\PYZus{}Ma}\PY{o}{*}\PY{o}{*}\PY{l+m+mi}{2}\PY{o}{*}\PY{n}{DMa}\PY{o}{*}\PY{o}{*}\PY{l+m+mi}{2} \PY{o}{+} \PY{n}{Dcm\PYZus{}Ta}\PY{o}{*}\PY{o}{*}\PY{l+m+mi}{2}\PY{o}{*}\PY{n}{DTa}\PY{o}{*}\PY{o}{*}\PY{l+m+mi}{2} \PY{o}{+} \PY{n}{Dcm\PYZus{}Teq}\PY{o}{*}\PY{o}{*}\PY{l+m+mi}{2}\PY{o}{*}\PY{n}{DTeq}\PY{o}{*}\PY{o}{*}\PY{l+m+mi}{2} \PY{o}{+} \PY{n}{Dcm\PYZus{}Mm}\PY{o}{*}\PY{o}{*}\PY{l+m+mi}{2}\PY{o}{*}\PY{n}{DMm}\PY{o}{*}\PY{o}{*}\PY{l+m+mi}{2} \PY{o}{+} \PY{n}{Dcm\PYZus{}Tm}\PY{o}{*}\PY{o}{*}\PY{l+m+mi}{2}\PY{o}{*}\PY{n}{DTm}\PY{o}{*}\PY{o}{*}\PY{l+m+mi}{2}\PY{p}{)}
\PY{n+nb}{print}\PY{p}{(}\PY{l+s+s2}{\PYZdq{}}\PY{l+s+s2}{Solido n°    cm [°C]: }\PY{l+s+s2}{\PYZdq{}}\PY{p}{)}
\PY{k}{for} \PY{n}{i} \PY{o+ow}{in} \PY{n+nb}{range}\PY{p}{(}\PY{l+m+mi}{0}\PY{p}{,} \PY{l+m+mi}{3}\PY{p}{)}\PY{p}{:}
    \PY{n+nb}{print}\PY{p}{(}\PY{l+s+s2}{\PYZdq{}}\PY{l+s+si}{\PYZpc{}i}\PY{l+s+s2}{            }\PY{l+s+si}{\PYZpc{}.2f}\PY{l+s+s2}{\PYZdq{}} \PY{o}{\PYZpc{}}\PY{p}{(}\PY{n}{i}\PY{o}{+}\PY{l+m+mi}{1}\PY{p}{,} \PY{n}{cm}\PY{p}{[}\PY{n}{i}\PY{p}{]}\PY{p}{)}\PY{p}{)}
\PY{n+nb}{print}\PY{p}{(}\PY{l+s+s2}{\PYZdq{}}\PY{l+s+s2}{Solido n°    Dcm [°C]: }\PY{l+s+s2}{\PYZdq{}}\PY{p}{)}
\PY{k}{for} \PY{n}{i} \PY{o+ow}{in} \PY{n+nb}{range}\PY{p}{(}\PY{l+m+mi}{0}\PY{p}{,} \PY{l+m+mi}{3}\PY{p}{)}\PY{p}{:}
    \PY{n+nb}{print}\PY{p}{(}\PY{l+s+s2}{\PYZdq{}}\PY{l+s+si}{\PYZpc{}i}\PY{l+s+s2}{            }\PY{l+s+si}{\PYZpc{}.2f}\PY{l+s+s2}{\PYZdq{}} \PY{o}{\PYZpc{}}\PY{p}{(}\PY{n}{i}\PY{o}{+}\PY{l+m+mi}{1}\PY{p}{,} \PY{n}{Dcm}\PY{p}{[}\PY{n}{i}\PY{p}{]}\PY{p}{)}\PY{p}{)}
\PY{n+nb}{print}\PY{p}{(}\PY{l+s+s2}{\PYZdq{}}\PY{l+s+s2}{Solido n°    Dcm/cm [°C]: }\PY{l+s+s2}{\PYZdq{}}\PY{p}{)}
\PY{k}{for} \PY{n}{i} \PY{o+ow}{in} \PY{n+nb}{range}\PY{p}{(}\PY{l+m+mi}{0}\PY{p}{,} \PY{l+m+mi}{3}\PY{p}{)}\PY{p}{:}
    \PY{n+nb}{print}\PY{p}{(}\PY{l+s+s2}{\PYZdq{}}\PY{l+s+si}{\PYZpc{}i}\PY{l+s+s2}{            }\PY{l+s+si}{\PYZpc{}.2f}\PY{l+s+s2}{\PYZdq{}} \PY{o}{\PYZpc{}}\PY{p}{(}\PY{n}{i}\PY{o}{+}\PY{l+m+mi}{1}\PY{p}{,} \PY{p}{(}\PY{n}{Dcm}\PY{p}{[}\PY{n}{i}\PY{p}{]}\PY{o}{/}\PY{n}{cm}\PY{p}{[}\PY{n}{i}\PY{p}{]}\PY{p}{)}\PY{p}{)}\PY{p}{)}
\end{Verbatim}
\end{tcolorbox}

    \begin{Verbatim}[commandchars=\\\{\}]
Solido n°    cm [°C]:
1            0.32
2            0.52
3            0.16
Solido n°    Dcm [°C]:
1            0.02
2            0.03
3            0.01
Solido n°    Dcm/cm [°C]:
1            0.07
2            0.05
3            0.07
    \end{Verbatim}

    \hypertarget{incertezze-con-metodo-2}{%
\subsubsection{Incertezze con metodo 2}\label{incertezze-con-metodo-2}}

commentrae, spiegare perchè viene diverso da metodo 1

    \begin{tcolorbox}[breakable, size=fbox, boxrule=1pt, pad at break*=1mm,colback=cellbackground, colframe=cellborder]
\prompt{In}{incolor}{18}{\boxspacing}
\begin{Verbatim}[commandchars=\\\{\}]
\PY{k}{for} \PY{n}{x} \PY{o+ow}{in} \PY{n+nb}{range}\PY{p}{(}\PY{l+m+mi}{0}\PY{p}{,}\PY{l+m+mi}{3}\PY{p}{)}\PY{p}{:}
    \PY{n}{Tsa} \PY{o}{=} \PY{n}{Ta}\PY{p}{[}\PY{n}{x}\PY{p}{]} \PY{o}{+} \PY{l+m+mf}{0.2}\PY{o}{*}\PY{p}{(}\PY{n}{rand}\PY{p}{(}\PY{l+m+mi}{100000}\PY{p}{)}\PY{o}{\PYZhy{}}\PY{l+m+mf}{0.5}\PY{p}{)}
    \PY{n}{Tse} \PY{o}{=} \PY{n}{Teq}\PY{p}{[}\PY{n}{x}\PY{p}{]} \PY{o}{+} \PY{l+m+mf}{0.2}\PY{o}{*}\PY{p}{(}\PY{n}{rand}\PY{p}{(}\PY{l+m+mi}{100000}\PY{p}{)}\PY{o}{\PYZhy{}}\PY{l+m+mf}{0.5}\PY{p}{)}
    \PY{n}{Tsm} \PY{o}{=} \PY{n}{Tm}\PY{p}{[}\PY{n}{x}\PY{p}{]} \PY{o}{+} \PY{l+m+mf}{0.2}\PY{o}{*}\PY{p}{(}\PY{n}{rand}\PY{p}{(}\PY{l+m+mi}{100000}\PY{p}{)}\PY{o}{\PYZhy{}}\PY{l+m+mf}{0.5}\PY{p}{)}
    \PY{n}{Msa} \PY{o}{=} \PY{n}{Ma}\PY{p}{[}\PY{n}{x}\PY{p}{]} \PY{o}{+} \PY{l+m+mf}{0.2}\PY{o}{*}\PY{p}{(}\PY{n}{rand}\PY{p}{(}\PY{l+m+mi}{100000}\PY{p}{)}\PY{o}{\PYZhy{}}\PY{l+m+mf}{0.5}\PY{p}{)}
    \PY{n}{Mse} \PY{o}{=} \PY{n}{Meq} \PY{o}{+} \PY{l+m+mf}{10.}\PY{o}{*}\PY{p}{(}\PY{n}{randn}\PY{p}{(}\PY{l+m+mi}{100000}\PY{p}{)}\PY{o}{\PYZhy{}}\PY{l+m+mf}{0.5}\PY{p}{)}
    \PY{n}{Msm} \PY{o}{=} \PY{n}{Mm}\PY{p}{[}\PY{n}{x}\PY{p}{]} \PY{o}{+} \PY{l+m+mf}{0.2}\PY{o}{*}\PY{p}{(}\PY{n}{rand}\PY{p}{(}\PY{l+m+mi}{100000}\PY{p}{)}\PY{o}{\PYZhy{}}\PY{l+m+mf}{0.5}\PY{p}{)}
    \PY{n}{R} \PY{o}{=} \PY{p}{(}\PY{n}{Tsa} \PY{o}{\PYZhy{}} \PY{n}{Tse}\PY{p}{)}\PY{o}{/}\PY{p}{(}\PY{n}{Tse}\PY{o}{\PYZhy{}}\PY{n}{Tsm}\PY{p}{)}
    \PY{n}{Mtot} \PY{o}{=} \PY{n}{Mse} \PY{o}{+} \PY{n}{Msa}
    \PY{n}{csm} \PY{o}{=} \PY{n}{Mtot}\PY{o}{*}\PY{n}{R}\PY{o}{/}\PY{n}{Msm}
    \PY{n+nb}{print}\PY{p}{(}\PY{l+s+s2}{\PYZdq{}}\PY{l+s+s2}{sigma [}\PY{l+s+si}{\PYZpc{}.2f}\PY{l+s+s2}{]}\PY{l+s+s2}{\PYZdq{}} \PY{o}{\PYZpc{}} \PY{p}{(}\PY{n}{stdev}\PY{p}{(}\PY{n}{csm}\PY{p}{)}\PY{p}{)}\PY{p}{)}
    \PY{n}{hist}\PY{p}{(}\PY{n}{csm}\PY{p}{,} \PY{n}{bins}\PY{o}{=}\PY{l+m+mi}{100}\PY{p}{)}\PY{p}{;}
    \PY{n}{xlabel}\PY{p}{(}\PY{l+s+s2}{\PYZdq{}}\PY{l+s+s2}{T [\PYZdl{}\PYZca{}}\PY{l+s+s2}{\PYZbs{}}\PY{l+s+s2}{circ C\PYZdl{}]}\PY{l+s+s2}{\PYZdq{}}\PY{p}{)}
    \PY{n}{ylabel}\PY{p}{(}\PY{l+s+s2}{\PYZdq{}}\PY{l+s+s2}{Frequenza}\PY{l+s+s2}{\PYZdq{}}\PY{p}{)}
\end{Verbatim}
\end{tcolorbox}

    \begin{Verbatim}[commandchars=\\\{\}]
sigma [0.02]
sigma [0.03]
sigma [0.01]
    \end{Verbatim}

    \begin{center}
    \adjustimage{max size={0.9\linewidth}{0.9\paperheight}}{./Esperienza1_files/./Esperienza1_29_1.pdf}
    \end{center}
    { \hspace*{\fill} \\}
    
    \hypertarget{conclusioni}{%
\subsubsection{Conclusioni}\label{conclusioni}}

Riportiamo in una tabella i calori specifici stimati con l'incertezza,
nelle due unità di misura \(\frac{cal}{g^\circ C}\) e \(\frac{J}{kgK}\),
con il valore dell'incertezza relativa definita come \(\frac{J}{kgK}\)

\begin{longtable}[]{@{}rccl@{}}
\toprule
\begin{minipage}[b]{0.09\columnwidth}\raggedleft
Materiali\strut
\end{minipage} & \begin{minipage}[b]{0.41\columnwidth}\centering
Calore specifico \(\pm\) incertezza \([\frac{cal}{g^\circ C}]\)\strut
\end{minipage} & \begin{minipage}[b]{0.30\columnwidth}\centering
Calore specifico \(\pm\) incertezza \([\frac{J}{kgK}]\)\strut
\end{minipage} & \begin{minipage}[b]{0.09\columnwidth}\raggedright
Errore relativo \(\frac{\delta x}{x}\)\strut
\end{minipage}\tabularnewline
\midrule
\endhead
\begin{minipage}[t]{0.09\columnwidth}\raggedleft
Parallelepipedo\strut
\end{minipage} & \begin{minipage}[t]{0.41\columnwidth}\centering
\(0.32\pm0.02\)\strut
\end{minipage} & \begin{minipage}[t]{0.30\columnwidth}\centering
\(1.33 \pm 0.08\)\strut
\end{minipage} & \begin{minipage}[t]{0.09\columnwidth}\raggedright
\(7 \%\)\strut
\end{minipage}\tabularnewline
\begin{minipage}[t]{0.09\columnwidth}\raggedleft
Pietra\strut
\end{minipage} & \begin{minipage}[t]{0.41\columnwidth}\centering
\(0.52 \pm 0.03\)\strut
\end{minipage} & \begin{minipage}[t]{0.30\columnwidth}\centering
\(2.17 \pm 0.13\)\strut
\end{minipage} & \begin{minipage}[t]{0.09\columnwidth}\raggedright
\(5 \%\)\strut
\end{minipage}\tabularnewline
\begin{minipage}[t]{0.09\columnwidth}\raggedleft
Cilindro\strut
\end{minipage} & \begin{minipage}[t]{0.41\columnwidth}\centering
\(0.16 \pm 0.01\)\strut
\end{minipage} & \begin{minipage}[t]{0.30\columnwidth}\centering
\(0.66 \pm 0.04\)\strut
\end{minipage} & \begin{minipage}[t]{0.09\columnwidth}\raggedright
\(7 \%\)\strut
\end{minipage}\tabularnewline
\bottomrule
\end{longtable}

In base a considerazione qualitative, prima delle misure stesse avevamo
supposto che il materiale del parallelepipedo fosse l'alluminio, mentre
per quello del cilindro l'ottone. Per il secodno corpo, abbiamo
consideraro che si trattasse di un blocco di pietra. Nella seguente
tabella sono riportati i calori specifici di questi materiali. Per la
pietra riportiamo una media dei calori specifici delle varie tipologie
di pietre (non essendo in grado di riconoscerne una come quella
costituente il solido utilizzato in laboratorio):

\begin{longtable}[]{@{}rc@{}}
\toprule
Materiali & Calori specifici reali
\([\frac{cal}{g^\circ C}]\)\tabularnewline
\midrule
\endhead
Alluminio & \(0.215\)\tabularnewline
Pietra & \(0.210\)\tabularnewline
Ottone & \(0.09\)\tabularnewline
\bottomrule
\end{longtable}

Confrontando questi valori con quelli misurati in laboratorio, notiamo
che le misure, per quanto abbastanza precise (nel senso che l'incertezza
relativa è abbastanza piccola ), non sono molto accurate, ma risultano,
più o meno significativamente, sovrastimate. Riteniamo che la causa di
questi risultati sia l'effetto di errori sistematici di cui sono affette
le misure a causa della procedura scelta: infatti, consideriamo come
fattori che possono costituire una fonte di errore: 1. Scambi di calore
tra il solido e l'ambiente nel momento in cui questo viene portato fuori
dal calorimetro in cui era stato inizialmente inserito;\\
2. Scambi di calore tra il calorimetro e l'ambiente nel momento in cui
viene tolto il tappo per immergere il solido all'interno; 3. Scambi di
calore tra il calorimetro e l'ambiente anche a calorimetro chiuso (
ovvero, non perfetta adiabadicità del calorimetro); 4. Taratura degli
strumenti.

Quella che proponiamo di seguito è una procedura alternativa che
permetterebbe di ridurre, almeno, gli effetti dovuti agli errori (1) e
(2):

\begin{itemize}
\tightlist
\item
  Si immerge il materiale nel calorimetro riempito con acqua calda a
  temperatura \(T_{1}\);\\
\item
  Si versa dell'acqua fredda, a temperatura \(T_{2}\) attraverso uno dei
  buchi presenti sul coperchio del caloriemtro, mediante l'uso di un
  imbuto;
\item
  Si smette di versare acqua quando la temperatura raggiunta dal bagno è
  una temperatura intermedia tra \(T_{1}\); Ancora meglio sarebbe
  utilizzare per la massa di acqua da versare nel calorimetro delle
  mescolenze, un recipiente con un migliore isolamento termico.
\end{itemize}

    \hypertarget{calcolo-del-calore-latente-di-fusione-del-ghiaccio}{%
\subsection{1.3 Calcolo del calore latente di fusione del
ghiaccio}\label{calcolo-del-calore-latente-di-fusione-del-ghiaccio}}

\hypertarget{relazione-di-base}{%
\subsubsection{Relazione di base}\label{relazione-di-base}}

Come spiegato per l'esperimento 1.2.

\hypertarget{procedimento-di-misura}{%
\subsubsection{Procedimento di misura}\label{procedimento-di-misura}}

\ldots{}.

\hypertarget{dati}{%
\subsubsection{Dati}\label{dati}}

    \begin{tcolorbox}[breakable, size=fbox, boxrule=1pt, pad at break*=1mm,colback=cellbackground, colframe=cellborder]
\prompt{In}{incolor}{24}{\boxspacing}
\begin{Verbatim}[commandchars=\\\{\}]
\PY{n}{Ma} \PY{o}{=} \PY{l+m+mf}{200.5}
\PY{n}{DMa} \PY{o}{=} \PY{l+m+mf}{0.1}\PY{o}{/}\PY{n}{np}\PY{o}{.}\PY{n}{sqrt}\PY{p}{(}\PY{l+m+mi}{12}\PY{p}{)}
\PY{n}{Mg} \PY{o}{=} \PY{l+m+mf}{70.7}
\PY{n}{DMg} \PY{o}{=} \PY{l+m+mf}{0.1}\PY{o}{/}\PY{n}{np}\PY{o}{.}\PY{n}{sqrt}\PY{p}{(}\PY{l+m+mi}{12}\PY{p}{)}
\PY{n}{Ta} \PY{o}{=} \PY{l+m+mf}{40.}
\PY{n}{DTa} \PY{o}{=} \PY{l+m+mf}{0.2}\PY{o}{/}\PY{n}{np}\PY{o}{.}\PY{n}{sqrt}\PY{p}{(}\PY{l+m+mi}{12}\PY{p}{)}
\PY{n}{Teq}\PY{o}{=} \PY{l+m+mf}{13.4}
\PY{n}{DTeq} \PY{o}{=} \PY{l+m+mf}{0.2}\PY{o}{/}\PY{n}{np}\PY{o}{.}\PY{n}{sqrt}\PY{p}{(}\PY{l+m+mi}{12}\PY{p}{)}
\PY{n}{Tf}\PY{o}{=} \PY{l+m+mi}{0}
\end{Verbatim}
\end{tcolorbox}

    \hypertarget{calcolo}{%
\paragraph{Calcolo}\label{calcolo}}

    \begin{tcolorbox}[breakable, size=fbox, boxrule=1pt, pad at break*=1mm,colback=cellbackground, colframe=cellborder]
\prompt{In}{incolor}{25}{\boxspacing}
\begin{Verbatim}[commandchars=\\\{\}]
\PY{n}{L} \PY{o}{=} \PY{p}{(}\PY{p}{(}\PY{n}{Meq}\PY{o}{+}\PY{n}{Ma}\PY{p}{)}\PY{o}{*}\PY{n}{ca}\PY{o}{*}\PY{p}{(}\PY{n}{Ta}\PY{o}{\PYZhy{}}\PY{n}{Teq}\PY{p}{)}\PY{o}{\PYZhy{}}\PY{n}{ca}\PY{o}{*}\PY{n}{Mg}\PY{o}{*}\PY{p}{(}\PY{n}{Teq}\PY{o}{\PYZhy{}}\PY{n}{Tf}\PY{p}{)}\PY{p}{)}\PY{o}{/}\PY{n}{Mg}
\PY{n}{DL\PYZus{}Meq} \PY{o}{=} \PY{n}{ca}\PY{o}{*}\PY{p}{(}\PY{n}{Ta}\PY{o}{\PYZhy{}}\PY{n}{Teq}\PY{p}{)}\PY{o}{/}\PY{n}{Mg}
\PY{n}{DL\PYZus{}Ma} \PY{o}{=} \PY{n}{ca}\PY{o}{*}\PY{p}{(}\PY{n}{Ta}\PY{o}{\PYZhy{}}\PY{n}{Teq}\PY{p}{)}\PY{o}{/}\PY{n}{Mg}
\PY{n}{DL\PYZus{}Ta} \PY{o}{=} \PY{n}{ca}\PY{o}{*}\PY{p}{(}\PY{n}{Meq}\PY{o}{+}\PY{n}{Ma}\PY{p}{)}\PY{o}{/}\PY{n}{Mg}
\PY{n}{DL\PYZus{}Teq} \PY{o}{=} \PY{o}{\PYZhy{}} \PY{n}{ca}\PY{o}{*}\PY{p}{(}\PY{n}{Meq} \PY{o}{+} \PY{n}{Ma} \PY{o}{+} \PY{n}{Mg}\PY{p}{)}\PY{o}{/}\PY{n}{Mg}
\PY{n}{DL\PYZus{}Mg} \PY{o}{=} \PY{o}{\PYZhy{}}\PY{p}{(}\PY{p}{(}\PY{n}{Meq}\PY{o}{+}\PY{n}{Ma}\PY{p}{)}\PY{o}{*}\PY{n}{ca}\PY{o}{*}\PY{p}{(}\PY{n}{Ta}\PY{o}{\PYZhy{}}\PY{n}{Teq}\PY{p}{)}\PY{p}{)}\PY{o}{/}\PY{n}{Mg}\PY{o}{*}\PY{o}{*}\PY{l+m+mi}{2}
\PY{n}{DL} \PY{o}{=} \PY{n}{np}\PY{o}{.}\PY{n}{sqrt}\PY{p}{(}\PY{n}{DL\PYZus{}Meq}\PY{o}{*}\PY{o}{*}\PY{l+m+mi}{2}\PY{o}{*}\PY{n}{DMeq}\PY{o}{*}\PY{o}{*}\PY{l+m+mi}{2}\PY{o}{+}\PY{n}{DL\PYZus{}Ma}\PY{o}{*}\PY{o}{*}\PY{l+m+mi}{2}\PY{o}{*}\PY{n}{DMa}\PY{o}{*}\PY{o}{*}\PY{l+m+mi}{2}\PY{o}{+}\PY{n}{DL\PYZus{}Ta}\PY{o}{*}\PY{o}{*}\PY{l+m+mi}{2}\PY{o}{*}\PY{n}{DTa}\PY{o}{*}\PY{o}{*}\PY{l+m+mi}{2}\PY{o}{+}\PY{n}{DL\PYZus{}Teq}\PY{o}{*}\PY{o}{*}\PY{l+m+mi}{2}\PY{o}{*}\PY{n}{DTeq}\PY{o}{*}\PY{o}{*}\PY{l+m+mi}{2}\PY{o}{+}\PY{n}{DL\PYZus{}Mg}\PY{o}{*}\PY{o}{*}\PY{l+m+mi}{2}\PY{o}{*}\PY{n}{DMg}\PY{o}{*}\PY{o}{*}\PY{l+m+mi}{2}\PY{p}{)}
\PY{n+nb}{print}\PY{p}{(}\PY{l+s+s2}{\PYZdq{}}\PY{l+s+s2}{Lambda = }\PY{l+s+si}{\PYZpc{}.1f}\PY{l+s+s2}{ cal/g}\PY{l+s+s2}{\PYZdq{}} \PY{o}{\PYZpc{}}\PY{p}{(}\PY{n}{L}\PY{p}{)}\PY{p}{)}
\PY{n+nb}{print}\PY{p}{(}\PY{l+s+s2}{\PYZdq{}}\PY{l+s+s2}{Incertezza = }\PY{l+s+si}{\PYZpc{}.1f}\PY{l+s+s2}{ cal/g}\PY{l+s+s2}{\PYZdq{}} \PY{o}{\PYZpc{}}\PY{p}{(}\PY{n}{DL}\PY{p}{)}\PY{p}{)}
\end{Verbatim}
\end{tcolorbox}

    \begin{Verbatim}[commandchars=\\\{\}]
Lambda = 73.2 cal/g
Incertezza = 1.9 cal/g
    \end{Verbatim}

    \hypertarget{incertezze-con-il-metodo-2}{%
\subsubsection{Incertezze con il metodo
2}\label{incertezze-con-il-metodo-2}}

    \begin{tcolorbox}[breakable, size=fbox, boxrule=1pt, pad at break*=1mm,colback=cellbackground, colframe=cellborder]
\prompt{In}{incolor}{26}{\boxspacing}
\begin{Verbatim}[commandchars=\\\{\}]
\PY{n}{Msa} \PY{o}{=} \PY{n}{Ma} \PY{o}{+} \PY{l+m+mf}{0.2}\PY{o}{*}\PY{p}{(}\PY{n}{rand}\PY{p}{(}\PY{l+m+mi}{100000}\PY{p}{)}\PY{o}{\PYZhy{}}\PY{l+m+mf}{0.5}\PY{p}{)}
\PY{n}{Msg} \PY{o}{=} \PY{n}{Mg} \PY{o}{+} \PY{l+m+mf}{0.2}\PY{o}{*}\PY{p}{(}\PY{n}{rand}\PY{p}{(}\PY{l+m+mi}{100000}\PY{p}{)}\PY{o}{\PYZhy{}}\PY{l+m+mf}{0.5}\PY{p}{)}
\PY{n}{Tsa} \PY{o}{=} \PY{n}{Ta} \PY{o}{+} \PY{l+m+mf}{0.2}\PY{o}{*}\PY{p}{(}\PY{n}{rand}\PY{p}{(}\PY{l+m+mi}{100000}\PY{p}{)}\PY{o}{\PYZhy{}}\PY{l+m+mf}{0.5}\PY{p}{)}
\PY{n}{Tse} \PY{o}{=} \PY{n}{Teq} \PY{o}{+} \PY{l+m+mf}{0.2}\PY{o}{*}\PY{p}{(}\PY{n}{rand}\PY{p}{(}\PY{l+m+mi}{100000}\PY{p}{)}\PY{o}{\PYZhy{}}\PY{l+m+mf}{0.5}\PY{p}{)}
\PY{n}{Mse} \PY{o}{=} \PY{n}{Meq} \PY{o}{+} \PY{l+m+mf}{10.}\PY{o}{*}\PY{p}{(}\PY{n}{randn}\PY{p}{(}\PY{l+m+mi}{100000}\PY{p}{)}\PY{o}{\PYZhy{}}\PY{l+m+mf}{0.5}\PY{p}{)}
\PY{n}{L} \PY{o}{=} \PY{p}{(}\PY{p}{(}\PY{n}{Mse}\PY{o}{+}\PY{n}{Msa}\PY{p}{)}\PY{o}{*}\PY{n}{ca}\PY{o}{*}\PY{p}{(}\PY{n}{Tsa}\PY{o}{\PYZhy{}}\PY{n}{Tse}\PY{p}{)}\PY{o}{\PYZhy{}}\PY{n}{ca}\PY{o}{*}\PY{n}{Msg}\PY{o}{*}\PY{p}{(}\PY{n}{Tse}\PY{o}{\PYZhy{}}\PY{n}{Tf}\PY{p}{)}\PY{p}{)}\PY{o}{/}\PY{n}{Msg}
\PY{n}{hist}\PY{p}{(}\PY{n}{L}\PY{p}{,} \PY{n}{bins}\PY{o}{=}\PY{l+m+mi}{100}\PY{p}{)}\PY{p}{;}
\PY{n}{xlabel}\PY{p}{(}\PY{l+s+s2}{\PYZdq{}}\PY{l+s+s2}{Calore latente di fusione [Cal/g]}\PY{l+s+s2}{\PYZdq{}}\PY{p}{)}
\PY{n}{ylabel}\PY{p}{(}\PY{l+s+s2}{\PYZdq{}}\PY{l+s+s2}{Frequenza}\PY{l+s+s2}{\PYZdq{}}\PY{p}{)}
\PY{n+nb}{print}\PY{p}{(}\PY{l+s+s2}{\PYZdq{}}\PY{l+s+s2}{sigma [}\PY{l+s+si}{\PYZpc{}.2f}\PY{l+s+s2}{]}\PY{l+s+s2}{\PYZdq{}} \PY{o}{\PYZpc{}} \PY{p}{(}\PY{n}{stdev}\PY{p}{(}\PY{n}{L}\PY{p}{)}\PY{p}{)}\PY{p}{)}
\end{Verbatim}
\end{tcolorbox}

    \begin{Verbatim}[commandchars=\\\{\}]
sigma [3.76]
    \end{Verbatim}

    \begin{center}
    \adjustimage{max size={0.9\linewidth}{0.9\paperheight}}{./Esperienza1_files/./Esperienza1_36_1.pdf}
    \end{center}
    { \hspace*{\fill} \\}
    
    \hypertarget{concluzione}{%
\subsubsection{Concluzione}\label{concluzione}}

Migliorare la stima: bisognerebb tenere conto che la temperatura da cui
parte il ghiaccio non è 0 ma: secondo spiegazine professore minore
perchè sta nel polistirolo secondo i dati, maggiore (forse perche nel
trasportarla fuori dal polistirolo e pesarla c'è stato qualche scambio
di calore.

    \hypertarget{verifica-dellequivalente-in-acqua-del-calorimetro}{%
\subsection{1.4 Verifica dell'equivalente in acqua del
calorimetro}\label{verifica-dellequivalente-in-acqua-del-calorimetro}}

\hypertarget{relazione-di-base}{%
\subsubsection{Relazione di base}\label{relazione-di-base}}

Negli esperimenti precedenti per calolare il calore specifico dei tre
materiali (esperimento 1.2) e il calore latente di fusione del ghiaccio,
per una stima più corretta si è tenuto conto del fatto che il sistema
termodinamico costituito dall'oggetto (oppure dal ghiaccio) e l'acqua
non sono gli unici a scambiare calore; il calorimetro stesso, (o meglio
una sua frazione) infatti, partecipa a questi scambi termici, ed è
dunque necessario considerare anche la capacità termica del calorimetro
\(C_{cal}\). Per questo motivo, la formula corretta per il calcolo del
calore specifico di un materiale con capacità termica \(C_{m}\) e
temperatura \(T_{m}\), che immerso in un bagno di acqua a temperatura
\(T_{a}\) raggiunge una temperatura di equilibrio \(T_{e}\), è data da
\[ (C_{cal} + C_{a})(T_{a}-T_{e}) = C_{m}(T_{e} - T_{m})\] Invece di
considerare la capacità termica \(C_{cal}\), moltiplicando e dividendo
il primo membro per il calore specifico dell'acqua \(c_{a}\) il
risultato dell'operazione \(\frac{C_{cal}}{c_{a}}\) è una grandezza che
ha le dimensioni di una massa, che chiamiamo massa equivalente \(M^*\).
Tale massa rappresenta la massa di acqua che ha la stessa capacità
termica di quella parte di calorimetro che partecipa agli scambi
termici. Si arriva dunque alla formula utilizzata nell'esperimeto 1.2
\[ (M^*+ M_{a})c_{a}(T_{a}-T_{e}) = M_{m}c_{m}(T_{e} - T_{m})\] Una
possibile strategia per misurare il termine \(M^*\) è versare
all'interno del calorimetro un materiale il cui calore specifico sia
noto, come, per esempio, una massa di acqua \(M_{2}\) a temperatura
\(T_{2}\); chiamando \(M_{1}\) la massa dell'acqua contenuta nel
calorimetro con temperatura T\_\{1\}, si ottiene la formula che
corrisponde a quella da noi utilizzata nel codice
\ldots{}seguente\ldots{}
\[ (M^*+ M_{1})c_{a}(T_{1}-T_{e}) = M_{2}c_{m}(T_{e} - T_{2})\]

\hypertarget{procedimento-di-misura}{%
\subsubsection{Procedimento di misura}\label{procedimento-di-misura}}

\begin{itemize}
\tightlist
\item
  Si riempie il calorimetro delle mescolanze con una massa \(M_1\) di
  acqua a temperatura \(T_{1}\);
\item
  Si prepara in un contenitore (nel nostro caso, il secondo
  calorimetro), una massa di acqua \(M_{2}\) a temperatura \(T_{2}\);
\item
  Si versa l'acqua a temperatura \(T_{2}\) nel calorimetro delle
  mescolanze e si registra la temperatura di equilibrio \(T_{e}\) che
  viene raggiunta dal sistema. \#\#\# Dati \texttt{Ma1} massa acqua nel
  calorimetro delle mescolanze (\(g\));\\
  \texttt{DM\_Ma1} incertezza massa acqua nel calorimetro delle
  mescolande (\(g\));\\
  \texttt{Ma2} massa acqua nel secondo calorimetro (\(g\));\\
  \texttt{DMa2} incertezza massa acqua nel secondo calorimetro
  (\(g\));\\
  \texttt{Ta1} Temperatura dell'acqua nel primo calorimetro
  (\(^\circ C\) );\\
  \texttt{DTa1} Incertezza temperatura dell'acqua nel primo calorimetro
  (\(^\circ C\) );\\
  \texttt{Ta2} Temperatura dell'acqua nel secodno calorimetro
  (\(^\circ C\) );\\
  \texttt{DTa2} Incertezza temperatura dell'acqua nel secondo
  calorimetro (\(^\circ C\) );\\
  \texttt{Teq} Temperatura dell'acqua all'equilibrio (\(^\circ C\) );\\
  \texttt{DTa2} Incertezza temperatura dell'acqua all'equilibrio
  (\(^\circ C\) );
\end{itemize}

    \begin{tcolorbox}[breakable, size=fbox, boxrule=1pt, pad at break*=1mm,colback=cellbackground, colframe=cellborder]
\prompt{In}{incolor}{22}{\boxspacing}
\begin{Verbatim}[commandchars=\\\{\}]
\PY{n}{Ma1} \PY{o}{=} \PY{l+m+mf}{201.2}
\PY{n}{DMa1}\PY{o}{=} \PY{l+m+mf}{0.1}\PY{o}{/}\PY{n}{np}\PY{o}{.}\PY{n}{sqrt}\PY{p}{(}\PY{l+m+mi}{12}\PY{p}{)}
\PY{n}{Ma2} \PY{o}{=} \PY{l+m+mf}{31.2}
\PY{n}{DMa2}\PY{o}{=} \PY{l+m+mf}{0.1}\PY{o}{/}\PY{n}{np}\PY{o}{.}\PY{n}{sqrt}\PY{p}{(}\PY{l+m+mi}{12}\PY{p}{)}
\PY{n}{Ta1} \PY{o}{=} \PY{l+m+mf}{40.4}
\PY{n}{DTa1}\PY{o}{=} \PY{l+m+mf}{0.2}\PY{o}{/}\PY{n}{np}\PY{o}{.}\PY{n}{sqrt}\PY{p}{(}\PY{l+m+mi}{12}\PY{p}{)}
\PY{n}{Ta2} \PY{o}{=} \PY{l+m+mf}{23.6}
\PY{n}{DTa2}\PY{o}{=} \PY{l+m+mf}{0.2}\PY{o}{/}\PY{n}{np}\PY{o}{.}\PY{n}{sqrt}\PY{p}{(}\PY{l+m+mi}{12}\PY{p}{)}
\PY{n}{Teq} \PY{o}{=} \PY{l+m+mf}{38.4}
\PY{n}{DTeq}\PY{o}{=} \PY{l+m+mf}{0.2}\PY{o}{/}\PY{n}{np}\PY{o}{.}\PY{n}{sqrt}\PY{p}{(}\PY{l+m+mi}{12}\PY{p}{)}
\end{Verbatim}
\end{tcolorbox}

    \hypertarget{calcoli}{%
\paragraph{Calcoli}\label{calcoli}}

    \begin{tcolorbox}[breakable, size=fbox, boxrule=1pt, pad at break*=1mm,colback=cellbackground, colframe=cellborder]
\prompt{In}{incolor}{23}{\boxspacing}
\begin{Verbatim}[commandchars=\\\{\}]
\PY{n}{Meq} \PY{o}{=} \PY{n}{Ma2}\PY{o}{*}\PY{p}{(}\PY{n}{Teq} \PY{o}{\PYZhy{}} \PY{n}{Ta2}\PY{p}{)}\PY{o}{/}\PY{p}{(}\PY{n}{Ta1} \PY{o}{\PYZhy{}} \PY{n}{Teq}\PY{p}{)} \PY{o}{\PYZhy{}} \PY{n}{Ma1}
\PY{n}{DM\PYZus{}Ma1} \PY{o}{=} \PY{o}{\PYZhy{}}\PY{l+m+mi}{1}
\PY{n}{DM\PYZus{}Ma2} \PY{o}{=} \PY{p}{(}\PY{n}{Teq} \PY{o}{\PYZhy{}} \PY{n}{Ta2}\PY{p}{)}\PY{o}{/}\PY{p}{(}\PY{n}{Ta1} \PY{o}{\PYZhy{}} \PY{n}{Teq}\PY{p}{)}
\PY{n}{DM\PYZus{}Ta1} \PY{o}{=} \PY{o}{\PYZhy{}}\PY{n}{Ma2}\PY{o}{*}\PY{p}{(}\PY{n}{Teq} \PY{o}{\PYZhy{}} \PY{n}{Ta2}\PY{p}{)}\PY{o}{/}\PY{p}{(}\PY{n}{Ta1} \PY{o}{\PYZhy{}} \PY{n}{Teq}\PY{p}{)}\PY{o}{*}\PY{o}{*}\PY{l+m+mi}{2}
\PY{n}{DM\PYZus{}Ta2} \PY{o}{=} \PY{o}{\PYZhy{}}\PY{n}{Ma2}\PY{o}{/}\PY{p}{(}\PY{n}{Ta1} \PY{o}{\PYZhy{}} \PY{n}{Teq}\PY{p}{)}
\PY{n}{DM\PYZus{}Teq} \PY{o}{=}  \PY{n}{Ma2}\PY{o}{/}\PY{p}{(}\PY{n}{Ta1} \PY{o}{\PYZhy{}} \PY{n}{Teq}\PY{p}{)}
\PY{n}{DM} \PY{o}{=} \PY{n}{np}\PY{o}{.}\PY{n}{sqrt}\PY{p}{(}\PY{n}{DM\PYZus{}Teq}\PY{o}{*}\PY{o}{*}\PY{l+m+mi}{2}\PY{o}{*}\PY{n}{DTeq}\PY{o}{*}\PY{o}{*}\PY{l+m+mi}{2}\PY{o}{+}\PY{n}{DM\PYZus{}Ma1}\PY{o}{*}\PY{o}{*}\PY{l+m+mi}{2}\PY{o}{*}\PY{n}{DMa1}\PY{o}{*}\PY{o}{*}\PY{l+m+mi}{2}\PY{o}{+}\PY{n}{DM\PYZus{}Ta1}\PY{o}{*}\PY{o}{*}\PY{l+m+mi}{2}\PY{o}{*}\PY{n}{DTa1}\PY{o}{*}\PY{o}{*}\PY{l+m+mi}{2}\PY{o}{+}\PY{n}{DM\PYZus{}Ta2}\PY{o}{*}\PY{o}{*}\PY{l+m+mi}{2}\PY{o}{*}\PY{n}{DTa2}\PY{o}{*}\PY{o}{*}\PY{l+m+mi}{2}\PY{o}{+}\PY{n}{DM\PYZus{}Ma2}\PY{o}{*}\PY{o}{*}\PY{l+m+mi}{2}\PY{o}{*}\PY{n}{DMa2}\PY{o}{*}\PY{o}{*}\PY{l+m+mi}{2}\PY{p}{)}
\PY{n+nb}{print}\PY{p}{(}\PY{n}{Meq}\PY{p}{)}
\PY{n+nb}{print}\PY{p}{(}\PY{n}{DM}\PY{p}{)}
\end{Verbatim}
\end{tcolorbox}

    \begin{Verbatim}[commandchars=\\\{\}]
29.67999999999995
6.788974787599866
    \end{Verbatim}

    Possibile spiegazione: siccome la massa equivalente è definita coma la
massa di acqua che ha la stessa capacità termica di quella parte di
calorimetro che partecipa agli scambi, se la massa mi viene negativa
potrebbe volere dire che, per i valori inseriti, la massa di calorimetro
che partecipa agli scambi è praticamente inconsistente, cioè ci sta una
soglia minima di volume che deve essere riempito affinchè il calorimetro
partecipi agli scambi.

    \hypertarget{appendice}{%
\subsection{Appendice}\label{appendice}}

    \hypertarget{tabella-dati-esperimento-1}{%
\subsubsection{Tabella dati esperimento
1}\label{tabella-dati-esperimento-1}}

    \hypertarget{video-1-1s}{%
\subsection{Video 1, 1s}\label{video-1-1s}}

\begin{longtable}[]{@{}rc@{}}
\toprule
Tempo & Temperatura\tabularnewline
\midrule
\endhead
0 & 28.2\tabularnewline
1 & --\tabularnewline
2 & 34.2\tabularnewline
3 & 36.2\tabularnewline
4 & 38\tabularnewline
5 & 39.6\tabularnewline
6 & 41\tabularnewline
7 & 42.2\tabularnewline
8 & 43.4\tabularnewline
9 & 44.4\tabularnewline
10 & 45.4\tabularnewline
11 & 46.2\tabularnewline
12 & 47\tabularnewline
13 & 47.6\tabularnewline
14 & 48.2\tabularnewline
15 & 48.8\tabularnewline
16 & 49.4\tabularnewline
17 & 49.8\tabularnewline
18 & 50.2\tabularnewline
19 & 50.6\tabularnewline
20 & 51\tabularnewline
21 & 51.2\tabularnewline
22 & 51.6\tabularnewline
23 & 51.8\tabularnewline
24 & 52.2\tabularnewline
25 & 52.4\tabularnewline
26 & 52.6\tabularnewline
27 & 52.6\tabularnewline
28 & 52.8\tabularnewline
29 & 53\tabularnewline
30 & 53\tabularnewline
31 & 53.2\tabularnewline
32 & 53.4\tabularnewline
33 & 53.4\tabularnewline
34 & 53.6\tabularnewline
35 & 53.6\tabularnewline
36 & 53.6\tabularnewline
37 & 53.8\tabularnewline
38 & 53.8\tabularnewline
39 & 53.8\tabularnewline
40 & 54\tabularnewline
41 & 54\tabularnewline
42 & 54\tabularnewline
43 & 54\tabularnewline
44 & 54\tabularnewline
45 & 54.2\tabularnewline
46 & 54.2\tabularnewline
47 & 54.2\tabularnewline
48 & 54.2\tabularnewline
49 & 54.2\tabularnewline
50 & 54.4\tabularnewline
51 & 54.4\tabularnewline
52 & 54.4\tabularnewline
\bottomrule
\end{longtable}

    \hypertarget{video-2-1s}{%
\subsection{Video 2 , 1s}\label{video-2-1s}}

\begin{longtable}[]{@{}rc@{}}
\toprule
Tempo & Temperatura\tabularnewline
\midrule
\endhead
0 & 28.2\tabularnewline
1 & 30.8\tabularnewline
2 & 32.8\tabularnewline
3 & 35.2\tabularnewline
4 & 37.6\tabularnewline
5 & 39.2\tabularnewline
6 & 40.6\tabularnewline
7 & 41.8\tabularnewline
8 & 43\tabularnewline
9 & 44\tabularnewline
10 & 45\tabularnewline
11 & 46\tabularnewline
12 & 46.6\tabularnewline
13 & 47.4\tabularnewline
14 & 48\tabularnewline
15 & 48.6\tabularnewline
16 & 49.2\tabularnewline
17 & 49.6\tabularnewline
18 & 50\tabularnewline
19 & 50.4\tabularnewline
20 & 50.6\tabularnewline
21 & 51\tabularnewline
22 & 51.2\tabularnewline
23 & 51.6\tabularnewline
24 & 51.8\tabularnewline
25 & 52\tabularnewline
26 & 52\tabularnewline
27 & 52.2\tabularnewline
28 & 52.4\tabularnewline
29 & 52.6\tabularnewline
30 & 52.8\tabularnewline
31 & 52.8\tabularnewline
32 & 53\tabularnewline
33 & 53\tabularnewline
34 & 53.2\tabularnewline
35 & 53.2\tabularnewline
36 & 53.4\tabularnewline
37 & 53.4\tabularnewline
38 & 53.4\tabularnewline
39 & 53.6\tabularnewline
40 & 53.6\tabularnewline
41 & 53.6\tabularnewline
42 & 53.6\tabularnewline
43 & 53.8\tabularnewline
44 & 53.8\tabularnewline
45 & 53.8\tabularnewline
46 & 53.8\tabularnewline
47 & 53.8\tabularnewline
48 & 54\tabularnewline
49 & 54\tabularnewline
50 & 54\tabularnewline
51 & 54\tabularnewline
52 & 54\tabularnewline
53 & 54\tabularnewline
54 & 54\tabularnewline
55 & 54\tabularnewline
56 & 54\tabularnewline
57 & 54.2\tabularnewline
58 & 54.2\tabularnewline
59 & 54.2\tabularnewline
60 & 54.2\tabularnewline
61 & 54.2\tabularnewline
62 & 54.2\tabularnewline
63 & 54.2\tabularnewline
64 & 54.2\tabularnewline
65 & 54.2\tabularnewline
66 & 54.4\tabularnewline
\bottomrule
\end{longtable}


    % Add a bibliography block to the postdoc
    
    
    
\end{document}
